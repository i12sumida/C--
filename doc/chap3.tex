% 
% $Id$
%
\chapter{\cmml の仕様}

システム記述言語として実績のある \cl を参考に、\cmml は設計されました。
しかし、\cl は設計が古く分かりにくい仕様が多い言語でもあります。
そこで、設計の新しい \javal を参考に \cl の問題点を避けるようにしました。
最終的に \cmml は、
{\bf 「\javal の特徴を取り入れた簡易\cl」} になりました。

%\section{プログラムの構造}

\section{コメント}

\cmml では、2種類のコメントを使用できます。
一つは \verb!/* 〜 */! 形式のコメント、
もう一つは \verb!// 〜 行末! 形式のコメントです。
以下に例を示します。
コメントは、空白を挿入できるところならどこにでも書くことができます。

\begin{mylist}
\begin{verbatim}
/*
 *  コメントの例を示すプログラム
 */
#include "util.h"             // printf を使用するために必要

public void main( /* 引数なし */ ) {
  printf("%d\n",10);
}

\end{verbatim}
\end{mylist}


\section{データ型}

\cmml のデータ型は、\javal と同様に基本型と参照型に大別できます。
基本型には、 int 型、 char 型、 boolean 型の3種類があります。
char 型は Unicode の 1 文字を表現することを想定していますが、
2009年4月現在、コンパイラ、アセンブラ、\tac のハードウェアとも、
ASCII 文字にしか対応していません。

参照型には、配列型と構造体型があります。
参照型は \javal の参照型と良く似た型です。

\subsection{基本型}

int 型は整数値の表現に使用します。
char 型は文字の表現に使用します。
実は、int 型と char 型は全く同じ型です。
相互の代入等が自由にできます。
boolean 型は true と false のどちらかの値を取る論理型です。
\javal の boolean 型と同じものです。

\subsubsection{int 型}
int型は16bit 符号付き2の補数表現2進数です。
-32768 から 32767 までの範囲の数値を表現することができます。
int 型変数は次のように宣言します。

\begin{mylist}
\begin{verbatim}
// int 型変数を宣言した例
int a;
int b = 10;

public void main() {
  int c;
  int d = 20;
}
\end{verbatim}
\end{mylist}

int 型の定数は、
上のプログラム例にあるように 「\verb/10/」、「\verb/20/」と書きます。
また、16進数、8進数で書くこともできます。上の例と同じ値を、
16進数では「\verb/0xa/」、「\verb/0x14/」のように、
8進数では「\verb/012/」、「\verb/024/」のように書きます。
int 型は、四則演算、ビット毎の論理演算、シフト演算等の計算に使用できます。

\subsubsection{char 型}
char 型も16bit 符号付き2の補数表現2進数です。
char 型は Unicode の基本多言語面(BMP)の1文字を格納することを想定していますが、
現在のところコンパイラ、アセンブラ、\tac ハードウェアのどれも、
ASCII 文字にしか対応していません。
char 型変数は次のように宣言します。

\begin{mylist}
\begin{verbatim}
// char 型変数を宣言した例
char a;
char b = 'A';

public void main() {
  char c;
  char d = 'D';
}
\end{verbatim}
\end{mylist}

char 型の定数は、
上のプログラム例にあるように 「\verb/'A'/」、「\verb/'D'/」と書きます。
制御文字を表現するために\tabref{chap3:escape}の
エスケープ文字が用意されています
\footnote{正確には、
コンパイラではなくアセンブラが、
エスケープ文字の処理をしています。}。


\begin{mytable}{tbhp}{エスケープ文字}{chap3:escape}
\begin{tabular}{ll}
\hline
\multicolumn{1}{c}{エスケープ文字} & \multicolumn{1}{c}{意味} \\
\hline
\verb/\n/ & 改行 \\
\verb/\r/ & 復帰 \\
\verb/\t/ & 水平タブ \\
\verb/\x/16進数 & 文字コードを16進数で直接指定 \\
\verb/\/8進数 & 文字コードを8進数で直接指定 \\
\verb/\/文字 & 印刷可能な文字を指定(\verb/\'/ や \verb/\\/等) \\
\hline
\end{tabular}
\end{mytable}

\subsubsection{boolean型}

\cl では int 型で論理値を表現しました。
そのため、条件式を書かなければならないところに間違って代入式を書くミスが
よくありました。
\cmml では論理値用の専用型 boolean を導入したので、
このようなミスをコンパイラが発見することができるようになりました。

\begin{mylist}
\begin{verbatim}
// C言語でよくある条件式と代入式の書き間違え
if (a=1) {
  ...
}
\end{verbatim}
\end{mylist}

boolean 型の定数値は、\verb/true/、\verb/false/ と書き表します。
次に boolean 型の変数を宣言して使用する例を示します。

\begin{mylist}
\begin{verbatim}
// boolean 型の使用例
boolean b = true;     // true は定数

public void main() {
  b = b && (x==10);   // 論理演算の結果も boolean 型
  if (b) {            // 論理値なので条件として使用できる
    b = false;        // false も定数
    ...
  }
  ...
}
\end{verbatim}
\end{mylist}


\subsection{参照型}
\label{chap3:ref}

参照型は \cl のポインターに似た型です。
\javal の参照型と非常に良く似ています。
参照型には配列型と構造体型があります。
参照型の値はインスタンスのアドレスです。
参照型の特別な値として何も指していない状態を表す \verb/null/ があります。
\verb/null/ の値は \verb/0x0000/ です。
ここでは、配列型、多次元配列、文字列、 \verb/void/ 配列型、
構造体型について解説します。

\subsubsection{配列型}
\label{chap3:array}
配列を表現するための型です。
配列は「\verb/型名[]/」と宣言します。
例えば、int型やchar型の配列は次のように宣言します。

\begin{mylist}
\begin{verbatim}
// 配列型の参照変数を宣言した例
int[] a;
char[] b;
\end{verbatim}
\end{mylist}

\myfigure{tbhp}{5cm}{array.pdf}{配列の構造}{chap3:array}

\figref{chap3:array}のように、
\cmml の配列は参照変数と配列インスタンスの組合せによって実現します。
上の宣言では、参照変数が生成されるだけです。
配列として使用するためには次の例のように
「\verb/array/」や「\verb/malloc/」を用いて
配列インスタンスを割り付ける必要があります。
「\verb/array/」は、配列インスタンスを静的に割り付けます。
「\verb/array/」は、関数の外で宣言される配列だけで使用できます。
「\verb/malloc/」は、配列インスタンスを実行時に動的にヒープ領域に割り付けます。
「\verb/malloc/」で割り付けた領域は、
使用後「\verb/free/」によって解放する必要があります。
「\verb/free/」し忘れが予想されるので、
配列は静的に割り付けることをおすすめします。

\begin{mylist}
\begin{verbatim}
// 配列インスタンスを割り付ける例
int[] a;
int[] b = array(10);   // 要素数10(10ワード)の領域を割り付ける

public void main() {
  a = malloc(5);       // 5ワード(要素数5)の領域を割り付ける
  ...
  free(a);             // malloc した領域は忘れず解放する
}
\end{verbatim}
\end{mylist}

配列は次のようなプログラムでアクセスできます。
添字は 0 から始まります。

\begin{mylist}
\begin{verbatim}
// 配列をアクセスする例
b[9] = 1;
a[0] = b[9] + 1;
\end{verbatim}
\end{mylist}

\subsubsection{多次元配列}

多次元配列は配列の配列として表現されます。
\figref{chap3:array2}に示すように、
1次元配列の参照を要素とした配列を使うと2次元配列になります。
下に、2次元配列を使用するプログラム例を示します。
プログラム中の \verb/a2/ は、\verb/malloc/ によって、
\figref{chap3:array2}のようなデータ構造に作り上げられます。
「\verb/malloc/」を使用する場合は、プログラムが複雑になってしまいますが、
長方形ではない配列も実現できます。
プログラム中 \verb/b2/ は、「\verb/array/」を使用して
多次元配列に必要な配列インスタンスを割り付けた例です。
このように「\verb/array/」を使用すると、
多次元配列に必要な複雑なデータ構造を簡単に割り付けることができます。

プログラムを簡単に記述するために、
多次元配列は「\verb/array/」を使用して静的に割り付けることを強くおすすめします。

\myfigure{tbhp}{6cm}{array2.pdf}{2次元配列の構造}{chap3:array2}

\begin{mylist}
\begin{verbatim}
// 2次元配列の例
int[][] a2;
int[][] b2 = array(4,3);       // 一度に2次元配列の領域を割り付ける

public void main() {
  a2 = malloc(4);              // 配列参照の配列を割り付ける
  for (int i=0; i<4; i=i+1)
    a2[i] = malloc(3);         // int 型の1次元配列を割り付ける
  ...
  for (int i=0; i<4; i=i+1)
    free(a2[i]);               // int 型の1次元配列を解放
  free(a2);                    // 参照の配列を解放
}
\end{verbatim}
\end{mylist}

次のプログラムは多次元配列をアクセスする例です。
このプログラムは、\figref{chap3:array2}の右下の要素
(最後の要素)に 1 を代入したあと、
それを使用して \verb/b2/ の要素の値を決めます。

\begin{mylist}
\begin{verbatim}
  a2[3][2] = 1;
  b2[3][2] = a2[3][2] + 1;
\end{verbatim}
\end{mylist}

\subsubsection{文字列}

\cl 同様に、文字列は char 型の配列として表現されます。
文字列は、文字コード \verb/0x0000/ の文字で終端されます。
次のプログラム例のように文字列定数を用いることができ、
また、文字列の終端を \verb/'\0'/ との比較で判断できます。

\begin{mylist}
\begin{verbatim}
// 文字列の使用例
char[] str = "0123456789";

void putstr() {
  for (int i=0; str[i]!='\0'; i=i+1)  // 文字列の終端まで
    putch(str[i]);                    // 一文字づつ出力
}
\end{verbatim}
\end{mylist}

\subsubsection{ void 型配列}

\verb/void/型の配列は、どんな参照型とも互換性がある特別な型になります。
\cmml は型チェックが厳しいので、異なる型の間での代入はできません。
しかし、次のプログラム例のように \verb/void/ 型配列を用いることにより、
異なる参照型の間で代入ができます。
また、 \verb/malloc/ 関数や \verb/free/ 関数は、
\verb/void/型配列を使って実現されています。

\begin{mylist}
\begin{verbatim}
// 型変換の例
struct XYZ { int x,y,z; };
XYZ p = { 1, 2, 3 };

void putXYZ() {
  void[] tmp = p;            // tmp にはどんな参照も代入可能
  int[]  xyz = tmp;          // tmp からはどんな参照へも代入可能
  for (int i=0; i<3; i=i+1)
    printf("%d\n",xyz[i]);   // 構造体を配列として無理矢理アクセス
}
\end{verbatim}
\end{mylist}

\subsubsection{構造体型}

\label{chap3:struct}
\cmml の構造体型は \javal のクラスからメソッドを取り除いたものに良く似ています。
構造体型は次のプログラム例のように宣言します。
この例は重連結リストを構成するための \verb/Node/ 型を宣言しています。
\cl の構造体宣言に良く似ていますが、
構造体名が型名になる点と、
構造体が参照型である点が異なります。
\cl の構造体よりむしろ \javal の class に似ていると言えます。

\begin{mylist}
\begin{verbatim}
struct Node {
  Nodo next;    // 自身と同じ型を参照
  Nodo prev;    // 自身と同じ型を参照
  int  val;     // ノードのデータ
};
\end{verbatim}
\end{mylist}

\verb/Node/ 型の変数を宣言して使用する例を下に示します。
構造体名が型名になるので \verb/struct/ を書かずに変数宣言します。
構造体フィールドは、「\verb/参照.フィールド名/」形式で参照します。
\verb/new/ の代わりに \verb/malloc/ を使用する他は
\javal と良く似た書き方になります。

\begin{mylist}
\begin{verbatim}
public void main() {
  Node start;                         // 重連結リストのルート

  start = malloc(sizeof(Node));       // 門番をリストに投げ込む
  start.next = start;                 // 門番を初期化する
  start.prev = start;
  start.val  = 0;
  ...
}
\end{verbatim}
\end{mylist}

\section{関数}
\label{chap3:func}
\cmml でも、\cl 同様に関数を宣言して使用することができます。
\cl と比較して \cmml は厳格さを求めています。
つまり、先に宣言された関数しか呼び出すことができませんし、
引数の個数や型が完全に一致しないとコンパイル時エラーになります。
ライブラリ関数を使用する場合は、
呼び出す前に必ずプロトタイプ宣言をする必要があります
\footnote{
\verb/"util.h"/
をインクルードすると自動的にプロトタイプ宣言されます。}。
また、関数の型を省略することができません。
値を返さない関数は \verb/void/、
整数を返す関数は \verb/int/、
論理値を返す関数は \verb/boolean/ と明示する必要があります。

\cl や \javal と同様に関数の仮引数は、関数の自動変数と同じように使用できます。
可変個引数の関数を宣言することもできます。
可変個引数関数の仮引数は「\verb/.../」と書きます。
可変個引数関数の内部では、
\pageref{chap4:args}ページで説明する
\ul \ul args 関数を使用して引数をアクセスします。

\begin{mylist}
\begin{verbatim}
#include "util.h"             // printf のプロトタイプ宣言が含まれる
int f() {                     // 引数の無い関数
  return 1;                   // void 型以外の関数は必ずreturnが必要
}

void g(int x) {
  x = x * x;                  // 仮引数は変数のように使用できる
  printf("%04x\n", x);        // プロトタイプ宣言が必要
}

public void main() {
  int x = f();                // 関数の呼び出し
  g(x);                       // 引数の型と個数が一致する必要がある
}
\end{verbatim}
\end{mylist}

\section{interrupt関数}
%特殊な種類の関数として interrupt 関数があります。
interrupt 関数は、
割込みハンドラを \cmml で記述するために用意しました。
%普通の関数と異なり、
CPUのコンテキスト(フラグ、レジスタ等)を、一切、破壊しません。
関数の入口でコンテキストをスタックに保存し、出口で復旧します。
割込みにより起動される関数なので、
プログラムから呼び出すことはできません。
仮引数を宣言することもできません。
次の例のように関数型の代わりに \verb/interrupt/ と書きます。
例中の \verb/main/ 関数のように、
addrof 演算子(\pageref{chap3:addrof}ページ参照)や
\ul \ul ItoA 関数(\pageref{chap4:itoa}ページ参照)を使用して、
割込みベクタに interrupt 関数を登録します。

\begin{mylist}
\begin{verbatim}
interrupt timerHdr() {         // タイマー割込みハンドラのつもり
  ...
}

public void main() {
  int[] vect = __ItoA(0xfff0); // vect は割込みベクタの配列
  vect[4] = addrof(timerHdr);  // vect[4] はタイマー割込みのベクタ
  ...
}
\end{verbatim}
\end{mylist}


\section{変数}

\cmml の変数は静的な変数と自動変数の2種類です。
関数の外部で宣言した変数は全て静的な大域変数になります。
逆に、関数の内部で宣言した変数は
全てブロック内でローカルな自動変数になります。
関数の内部では、どこでも変数宣言が可能です。
ローカル変数の有効範囲はブロックの終わりまでです。
同じ名前の変数があった場合はローカル変数が優先されます。
ローカル変数同士の名前の重複は認めません(\javal と同じ規則)。
次にプログラム例を示します。

\begin{mylist}
\begin{verbatim}
#include "util.h"
int   n = 10;                 // 静的な変数
int[] a = array(10);          // 静的な配列

public void main() {
  int i;                      // 関数内ローカル変数
  for (i=0;i<n;i=i+1) {       // 大域変数 n のこと
    int j = i * i;            // ブロック内ローカル変数
    printf("%d\n",j);
    int n = j * j;            // どこでも変数宣言可能
    printf("%d %d\n",j,n);    // ローカルな n のこと
  }                        
  printf("%d\n",n);           // 大域変数 n のこと
}
\end{verbatim}
\end{mylist}

\section{変数の初期化}

基本型の変数は、いつでも宣言と同時に初期化することができます。
参照型の変数は静的に割り付けられる場合だけ初期化できます。
構造体内部に入れ子になった参照型は\verb/null/で初期化することしかできません。
ただし名前表のようなデータ構造を作るために、
文字列での初期化だけは可能にしてあります。

\begin{mylist}
\begin{verbatim}
int     n = 10;                             // 基本型変数の初期化
int[]   a = { 1, 2, 3};                     // 基本型配列の初期化
int[][] b = {{1,2,0},{1,2,3,0}};            // いびつな配列の初期化
struct List { int  val; List next; };       // 構造体型の宣言
List    r = { 1, null };                    // 構造体変数の初期化
struct NameEntry { char[] name, int val; };
NameEntry[] weekTable = {                   // 名前表を作成する例
  {"Sun", 1}, {"Mon", 2}, {"Tue", 3}
};
public void main() {
  int i = 10;                               // 自動変数の初期化
}
\end{verbatim}
\end{mylist}

\section{public 修飾子}

関数、大域変数を他のコンパイル単位から参照できるようにします。
\verb/public/修飾子の付いていない関数や大域変数は
他のコンパイル単位からは見えないので、
重複を心配しないで自由に名前を付けることができます。
main関数はスタートアップルーチンから呼び出されるので、
必ず\verb/public/修飾をしなければなりません。

\begin{mylist}
\begin{verbatim}
int        n = 10;            // 同じ .cmm ファイル内だけで参照可
public int m = 20;            // 他の .cmm ファイルからも参照可

void f() { ... }              // 同じ .cmm ファイル内だけで参照可
public void g() { ... }       // 他の .cmm ファイルからも参照可

public void printf(char[] s, ...);  // ライブラリ関数は public

public void main() {          // main は必ず public
  f();
  g();
  printf("\n");
}
\end{verbatim}
\end{mylist}

\section{演算子}

\cmml には \cl をお手本に一通りの演算子が準備されています。
しかし、コンパイラを小さくする目的で、
レパートリーの多い代入演算子、
前置後置等の組合せが複雑なインクリメント演算子とデクリメント演算子を
省略しました。

\subsection{代入演算}

\cl や \javal には、たくさんの種類の代入演算子があり便利に使用できました。
\cmml では、コンパイラをコンパクトに実装するために代入演算子を「 \verb/=/」の
1種類だけにしました。
\cl や \javal 同様、代入演算の結果は代入した値になります。

\cmml では、代入演算子の左辺と右辺が厳密に同じ型でなければなりません
\footnote{int 型と char 型は同じ型なので代入可能です。}。
これは、コンパイル時になるべく多くのバグを発見するための仕様です。
参照型の場合も型が厳密に一致している必要があります。
ただし、\verb/void[]/ だけ例外的にどの参照型とも代入可能です。
数値型は一種類しかありませんので自動的な型変換はありません。

\begin{mylist}
\begin{verbatim}
int     a;
boolean b = true;
char    c;
struct X { int r; };
struct Y { int r; };
a = 10;            // 同じ型なので代入可能
c = a;             // char と int も同じ型なので代入可能
a = b;             // 型が異なるので代入できない
a = c = 9;         // 代入演算(c=9)の結果は代入した値(9)
                   // 代入演算の結果(9)を a に代入する
X x = { 1 };
Y y;
y = x;             // 型が異なるので代入できない
void[] p = x;      // void[] にはどんな参照型も代入可能
y = p;             // void[] はどんな参照型にも代入可能
\end{verbatim}
\end{mylist}

\subsection{数値演算}

int 型、char 型データの計算に、2項演算子の
「\verb/+/」(和)、
「\verb/-/」(差)、
「\verb/*/」(積)、
「\verb+/+」(商)、
「\verb/%/」(余)
が使用できます。
その他に、単項演算子
「\verb/+/」、
「\verb/-/」が使用できます。
演算子の優先順位は数学と同じです。
計算(数値演算)をして、計算結果を変数に代入(代入演算)する例を次に示します。

\begin{mylist}
\begin{verbatim}
x = -10 + 3 * 2;
\end{verbatim}
\end{mylist}

\subsection{比較演算}

\underline{(1)整数型の大小比較と同値の判定}、
\underline{(2)参照型、論理型の同値の判定}ができます。
大小比較の演算子は、
「\verb/>/」(より大きい)、
「\verb/>=/」(以上)、
「\verb/</」(未満)、
「\verb/<=/」(以下) の4種類です。
同値を判定する演算子は、
「\verb/==/」(等しい)、
「\verb/!=/」(異なる)
の2種類です。
比較演算の結果は論理型になるので、
比較演算の結果を整数型に代入することはできません。
論理型は if 文や while 文などの条件に使用できます。
次に、比較演算の例を示します。

\begin{mylist}
\begin{verbatim}
int x = 11;
boolean b;
b = x > 10;              // 整数の大小比較
if (b==false) { ... }    // 論理型の同値判定
\end{verbatim}
\end{mylist}

\subsection{論理演算}

論理型のデータを対象にした演算です。
演算結果も論理型になります。
単項演算子「\verb/!/」(否定)、
2項演算子
「\verb/&&/」(論理積)、
「\verb/||/」(論理和)が使用できます。
次に、論理演算の例を示します。
論理型変数 \verb/b/ に比較結果を求めた後で、
\verb/b/ の否定を if 文の条件に使用しています。

\begin{mylist}
\begin{verbatim}
int x = 11;
boolean b;
b = 10 <= x && x <= 20;  // (10<=x) と (x<=20) の論理積を b に代入
if (!b) { ... }          // 論理値の否定
\end{verbatim}
\end{mylist}

\subsection{ビット毎の論理演算}

整数値を対象にした演算です。
演算結果も整数値になります。
単項演算子「\verb/~/」(全ビットを反転)、
2項演算子
「\verb/&/」(ビット毎の論理積)、
「\verb/|/」(ビット毎の論理和)、
「\verb/^/」(ビット毎の排他的論理和)が使用できます。
次に、ビット毎の論理演算の例を示します。
マスクを使用して、変数 \verb/x/ の下位8ビットを取り出して表示します。
\verb/printf/の括弧内で下位8ビットを取り出すために
ビット毎の論理演算をしています。

\begin{mylist}
\begin{verbatim}
int x   = 0xabcd;
int msk = 0x00ff;
printf("%x", x & msk);
\end{verbatim}
\end{mylist}

\subsection{シフト演算}

整数値を対象にした演算です。
演算結果も整数値になります。
2項演算子
「\verb/>>/」(右算術シフト)、
「\verb/<</」(左算術シフト)が使用できます。
\cmml の整数型は符号付きしかありませんので、
右シフトは算術シフトになります。
次に、シフト演算の例を示します。
シフト演算とマスクを使用して、
変数 \verb/x/ の値の上位8ビットを取り出して表示します。
算術シフトですから、マスクを忘れないように注意する必要があります。
\verb/printf/の括弧内で上位8ビットを取り出す計算をしています。

\begin{mylist}
\begin{verbatim}
int x   = 0xabcd;
printf("%x", x >> 8 & 0x00ff);
\end{verbatim}
\end{mylist}

\subsection{参照演算}

配列と構造体をアクセスするための「\verb/[ 添字式 ]/」や「\verb/./」は、
参照を対象にする演算子と考えることができます。
「\verb/[ 添字式 ]/」演算子は、配列参照と添字式から配列要素の値を求めます。
「\verb/./」演算子は、構造体参照とフィールド名からフィールドの値を求めます。
配列の配列である多次元配列のアクセスは、
「\verb/[ 添字式 ]/」演算子により取り出した配列要素値が配列参照なので、
更に「\verb/[ 添字式 ]/」演算子により次の配列要素を取り出すと考えます。
実際、\cmm コンパイラの内部でもそのように考えて扱っています。
次に多次元配列や構造体を使用したプログラムの例を示します。
「\ref{chap3:ref} 参照型」に示したプログラム例も参考にしてください。

\begin{mylist}
\begin{verbatim}
// 多次元配列は配列参照の配列と考える
int[][] a = {{1,2},{3,4}};  // 2次元配列を作る
void f() {
  int[] b = a[0];           // 2次元配列の要素は1次元配列
  int   c = b[1];           // 1次元配列の要素は int 型
  int   d = a[0][1];        // c と d は同じ結果になる
  ...
}

// 構造体リスト例
struct List {               // リスト構造のノード型
  List next;                // 次のノードの参照
  int  val;                 // ノードの値
};
List a;                     // リストのルートを作る

void g() {
  a = malloc(sizeof(List));       // リストの先頭ノードを作る
  a.val  = 1;
  a.next = malloc(sizeof(List));  // リストの2番目ノードを作る
  a.next.val  = 2;
  a.next.next = null;             // 2番目ノードは参照の参照
  ...
}
\end{verbatim}
\end{mylist}

\subsection{sizeof 演算}

構造体のサイズを知るための演算子です。
「\verb/sizeof(型)/」のように使用し整数型の値を返します。
値は、ワード単位で表した「型の領域サイズ」です。
\cmml では基本型のサイズは全て1です。
配列型のサイズも1になります
\footnote{配列型のサイズが1になる根拠はありませんが...とにかく1です。}。
1以外の値になるのは構造体型の場合だけです。
これまでのプログラム例で見てきたように、
\verb/malloc/ で領域を割り付けるとき使用します。

\subsection{addrof 演算}
\label{chap3:addrof}

関数や大域変数のアドレスを知るための演算子です。
「\verb/addrof(大域名)/」のように使用し整数型の値を返します。
\verb/interrupt/関数を割込みベクタに登録したりする目的で使用します。
配列や構造体の要素や、関数のローカル変数のアドレスを求めることはできません。

\subsection{カンマ演算}

複数の式を接続して文法上一つの式にします。
例えば、
式が一つしか書けない for 文の再初期化部分に二つの式を書くために使用できます。
カンマ演算子は、最も優先順位の低い演算子です。

\begin{mylist}
\begin{verbatim}
// カンマ式を使用した例
for (i=0;i<0;i=i+1,j=j-1) { ... } // 再初期化に二つの式を書いた
\end{verbatim}
\end{mylist}

\subsection{演算子のまとめ}

関数の呼び出しも厳密には「\verb/(引数リスト)/」演算子と考えることができますが、
関数の呼び出しかたは\pageref{chap3:func}ページで説明したので省略します。
その他に、演算の順序を明確にするための「\verb/(式)/」も含めて、
演算子の優先順位を\tabref{chap3:operator}にまとめます。
表の上の方に書いてある演算子が優先順位の高い演算子です。
同じ高さにある演算子同士は同じ優先順位になります。
計算は優先順位の高いものから順に行われます.
例えば,「\verb/*/」は「\verb/+/」よりも優先順位が高いので先に計算されます.
優先順位が同じ場合は結合規則の欄で示した順に計算されます。

\begin{mytable}{tbhp}{演算子の優先順位}{chap3:operator}
\begin{tabular}{l|l}
\hline
\multicolumn{1}{c}{演算子} & \multicolumn{1}{|c}{結合規則} \\
\hline
関数呼び出し、\verb/sizeof/、\verb/addrof/、
\verb/()/、\verb/[]/、\verb/./ & 左から右 \\
\verb/+/(単項演算子)、\verb/-/(単項演算子)、\verb/!/、\verb/~/ & 右から左 \\
\verb/*/、\verb!/!、\verb/%/             & 左から右 \\
\verb/+/、\verb/-/                       & 左から右 \\
\verb/<</、\verb/>>/                     & 左から右 \\
\verb/>/、\verb/>=/、\verb/</、\verb/<=/ & 左から右 \\
\verb/==/、\verb/!=/                     & 左から右 \\
\verb/&/                                 & 左から右 \\
\verb/^/                                 & 左から右 \\
\verb/|/                                 & 左から右 \\
\verb/&&/                                & 左から右 \\
\verb/||/                                & 左から右 \\
\verb/=/                                 & 右から左 \\
\verb/,/                                 & 左から右 \\
\end{tabular}
\end{mytable}

\section{文}

関数の内部に記述されるもので、変数宣言以外の記述を文と呼びます。
文は機械語に変換されて実行されます。
文には、空文、式文、ブロック、制御文(if 文や for 文)等があります。
\cmml の制御文には switch 文がありませんが
\cl にある他の制御文は一通り揃っています。

\subsection{空文}

単独の「\verb/;/」を空文と呼び、
文法上、一つの文として扱います。
本文のない for 文等で形式的な本文として使用します。
次に、空文を用いる例を示します。

\begin{mylist}
\begin{verbatim}
for (i=0; i<n; i=i*i)  // 必要なことはこの1行で全部記述できた
  ;                    // 空文
\end{verbatim}
\end{mylist}

\subsection{式文}

式の後ろに「\verb/;/」を付けたものを式文と呼び、
文法上、一つの文として扱います。
\cmml の文法に代入文はありませんが、
代入式に「\verb/;/」を付けた「式文」が同じ役割に使用されます。

\begin{quote}
\begin{verbatim}
式 ;
\end{verbatim}
\end{quote}

\subsection{ブロック}

「\verb/{/」と「 \verb/}/」で括って複数の文をグループ化し、
文法上、一つの文にします。
if 文や while 文の「本文」は、文法的には一つの文でなければなりません。
複数の文を「本文」として実行させたい場合はブロックにします。
また、ブロックはローカル変数の有効範囲を決定します。
ブロック内部で宣言された変数の有効範囲はブロックの最後までです。

\begin{quote}
\begin{verbatim}
{ 文 または 変数宣言 ... }
\end{verbatim}
\end{quote}

\subsection{if 文}

条件によって実行の流れを変更するための文です。
「条件式」は論理値を返す式でなければなりません。
「条件式」の値が true の場合「本文1」が実行され、
false の場合「本文2」が実行されます。
なお、else 節(「else 本文2」の部分)は省略することができます。

\begin{quote}
\begin{verbatim}
if ( 条件式 ) 本文1 【 else 本文2 】 
\end{verbatim}
\end{quote}

\subsection{while 文}

条件が成立している間、while 文の「本文」を実行します。
「条件式」は論理値を返す式でなければなりません。
まず、「条件式」を計算し、値が true なら「本文」が実行されます。
これは、「条件式」の値が false になるまで繰り返されます。

\begin{quote}
\begin{verbatim}
while ( 条件式 ) 本文
\end{verbatim}
\end{quote}

\subsection{do-while 文}

条件が成立している間、do-while 文の「本文」を実行します。
「条件式」は論理値を返す式でなければなりません。
まず「本文」を実行し、次に「条件式」を計算します。
「条件式」の値が true なら、再度、「本文」の実行に戻ります。
これは、「条件式」の値が false になるまで繰り返されます。

\begin{quote}
\begin{verbatim}
do 本文 while ( 条件式 ) ;
\end{verbatim}
\end{quote}

\subsection{for 文}

便利に拡張された while 文です。
「条件式」は論理型でなければなりません。
まず、「初期化式」か「ローカル変数宣言」を実行します。
次に「条件式」を計算し、値が true なら「本文」を実行します。
最後に「再初期化式」を実行し、その後「条件式」の計算に戻ります。
これは、「条件式」の値が false になるまで繰り返されます。

「ローカル変数宣言」で宣言された変数は、「条件式」、「再初期化式」、
「本文」で使用することができますが、それ以降では参照できません。
「初期化式」、「条件式」、「再初期化式」のどれも省略可能です。
「条件式」を省略した場合は無限ループの記述になります。

\begin{quote}
\begin{verbatim}
for( 【 初期化式 | ローカル変数宣言 】 ; 【 条件式 】 ; 【 再初期化式 】 ) 本文
\end{verbatim}
\end{quote}

\begin{mylist}
\begin{verbatim}
// for 文の使用例
int i;
for (i=0; i<10; i=i+1) { ... }
n = i;                              // i をここで使用できる

for (int j=0; j<10; j=j+1) { ... }
n = j;                              // この j は未定義になる

for (;;) {                          // 無限に本文を繰り返す
  ...
}
\end{verbatim}
\end{mylist}

\subsection{return 文}

関数から戻るときに使用します。
関数の途中で使用すると、関数の途中から呼び出し側に戻ることができます。
void 型以外の関数では、関数の最後の文が return 文でなければなりません。
「式」は void 型の関数では書いてはなりません。
逆に、void 型以外の関数では書かなければなりません。
「式」の型と関数の型は一致していなければなりません。
なお、interrupt 関数には void 型の関数と同じルールが適用されます。

\begin{quote}
\begin{verbatim}
return 【 式 】 ;
\end{verbatim}
\end{quote}

\begin{mylist}
\begin{verbatim}
// void 型以外の関数
int f() {
  ...
  if (err) return 1;     // f の途中から戻る
  ...
  return 0;              // この return は省略できない
}

// void 型の関数
void g() {
  ...
  if (err) return;       // g の途中から戻る
  ...
  return;                // この return は省略しても良い
}
\end{verbatim}
\end{mylist}

\subsection{break 文}

for 文や while 文、do-while 文の繰り返しから脱出します。
多重ループから一度に脱出することはできません。

\begin{quote}
\begin{verbatim}
break;
\end{verbatim}
\end{quote}

\subsection{continue 文}

for 文や while 文、do-while 文の本文の実行をスキップします。
for 文では再初期化式に、while 文と do-while 文では条件式にジャンプします。

\begin{quote}
\begin{verbatim}
continue;
\end{verbatim}
\end{quote}
