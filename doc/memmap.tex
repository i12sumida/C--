% 
% $Id$
%
\chapter{\tac ハードウェアマニュアル}

\section{メモリ構成}

\subsection{メモリマップ}
\label{app:memmap}

\tac は 64kW (64キロワード) のメモリ空間を持ちます。
1ワードは16ビットです。
しかし、ビデオメモリだけは8ビット幅です。
次の表に、メモリマップを示します。

\begin{center}
{\bf \tac のメモリマップ} \\
\begin{tabular}{|c|c|c|c|}
\hline
アドレス    & \multicolumn{2}{|c|}{内 容}              & 種類 \\
\hline
{\tt 0000H} & \multicolumn{2}{|c|}{}                   &      \\
{\tt ...  } & \multicolumn{2}{|c|}{ユーザ領域(60kW)}   & RAM  \\
{\tt EFFFH} & \multicolumn{2}{|c|}{}                   &      \\
\hline
{\tt F000H} &                  &  ビデオ               &      \\
{\tt ...  } &    空き          &  メモリ(2kB)          & RAM  \\
{\tt F7FFH} &    (上位8ビット) &  (下位8ビット)        &(8ビット幅) \\
\hline
{\tt F800H} & \multicolumn{2}{|c|}{}                   &      \\
{\tt ...  } & \multicolumn{2}{|c|}{IPL(2016W)}         & ROM  \\
{\tt FFDFH} & \multicolumn{2}{|c|}{}                   &      \\
\hline
{\tt FFE0H} & \multicolumn{2}{|c|}{}                   &      \\
{\tt ...  } & \multicolumn{2}{|c|}{IPLワーク(16W)}     &      \\
{\tt FFEFH} & \multicolumn{2}{|c|}{}                   &      \\
\cline{1-3}
{\tt FFF0H} & \multicolumn{2}{|c|}{INT0割込みベクタ}   &      \\
\cline{1-3}
{\tt FFF1H} & \multicolumn{2}{|c|}{INT1割込みベクタ}   &      \\
\cline{1-3}
{\tt FFF2H} & \multicolumn{2}{|c|}{INT2割込みベクタ}   &      \\
\cline{1-3}
{\tt FFF3H} & \multicolumn{2}{|c|}{INT3割込みベクタ}   &      \\
\cline{1-3}
{\tt FFF4H} & \multicolumn{2}{|c|}{Timer0割込みベクタ} &      \\
\cline{1-3}
{\tt FFF5H} & \multicolumn{2}{|c|}{Timer1割込みベクタ} &      \\
\cline{1-3}
{\tt FFF6H} & \multicolumn{2}{|c|}{Timer2割込みベクタ} & RAM  \\
\cline{1-3}
{\tt FFF7H} & \multicolumn{2}{|c|}{ATA割込みベクタ}    &      \\
\cline{1-3}
{\tt FFF8H} & \multicolumn{2}{|c|}{SIO受信割込みベクタ}&      \\
\cline{1-3}
{\tt FFF9H} & \multicolumn{2}{|c|}{SIO送信割込みベクタ}&      \\
\cline{1-3}
{\tt FFFAH} & \multicolumn{2}{|c|}{PS2受信割込みベクタ}&      \\
\cline{1-3}
{\tt FFFBH} & \multicolumn{2}{|c|}{PS2送信割込みベクタ}&      \\
\cline{1-3}
{\tt FFFCH} & \multicolumn{2}{|c|}{INT12割込みベクタ}  &      \\
\cline{1-3}
{\tt FFFDH} & \multicolumn{2}{|c|}{INT13割込みベクタ}  &      \\
\cline{1-3}
{\tt FFFEH} & \multicolumn{2}{|c|}{INT14割込みベクタ}  &      \\
\cline{1-3}
{\tt FFFFH} & \multicolumn{2}{|c|}{INT15割込みベクタ}  &      \\
\hline
\end{tabular}
\end{center}

\subsection{ビデオメモリの構成}

ビデオメモリは1アドレスに8ビットのデータを格納できます。
この8ビットデータを ASCII コードとみなし対応する文字を画面に表示します。
画面は横80文字、縦25行の構成になっており、
アドレスと画面上の文字の位置は次の表のような対応になります。
左上隅が F000H 番地、その右隣が F001H 番地です。
1行は 80 文字なので右端が F04FH 番地になります。
2行目から先も同様な構成で、
最後の行(第25行)が F780H 番地から F7CF 番地になります。

\begin{center}
{\bf \tac のビデオメモリと画面の対応} \\
\begin{tabular}{|c|c|c|c|c|}
\hline
F000H & F001H & F002H & ... & F04FH \\
\hline
F050H & F051H & F052H & ... & F09FH \\
\hline
F0A0H & F0A1H & F0A2H & ... & F0EFH \\
\hline
      &       &       &     &       \\
\multicolumn{5}{|c|}{...}           \\
      &       &       &     &       \\
\hline
F780H & F781H & F782H & ... & F7CFH \\
\hline
\end{tabular}
\end{center}

\subsection{IPLプログラム}
\label{app:ipl}

F800H 番地からの 2016 ワードに IPL プログラムの機械語本体が、
FFE0H 番地からの 16 ワードに変数領域が配置されます。
IPL プログラムはマイクロドライブの第23セクタからの連続セクタに記録された
「\verb/.bin/形式」の機械語プログラムをメモリに読み込んで実行を開始します。

IPL プログラムのいくつかのルーチンはユーザプログラムに公開されており、
``{\tt Kernel/*/*/kernel.cmm}''から呼び出して使用することができます。
例えば、``{\tt Kernel/*/*/util.h}''にプロトタイプ宣言されている putch 関数は、
``{\tt Kernel/*/*/crt0.s}''に本体が記述してあるように見えますが、
実は、{\tt crt0.s }には IPL ルーチンへのジャンプ命令が書いてあるだけです。
{\tt kernel.cmm} が呼び出している putch 関数は IPL ルーチンの putch 関数です。
そのため、IPL 実行時から一貫した画面の表示や、
カーソル位置の管理が可能です。
以下の表に IPL ルーチンのエントリーポイント一覧と、
変数領域の内容を示します。

\begin{center}
{\bf \tac のIPLルーチンエントリーポイント} \\
\begin{tabular}{|c|l|l|}
\hline
番地  & \multicolumn{1}{|c|}{ルーチン} & \multicolumn{1}{|c|}{意味} \\
\hline
F800H & \verb/ipl();/                  & IPL に戻る                 \\
F802H & \verb/cls();/                  & 画面をクリアする           \\
F804H & \verb/putch(char ch);/         & 画面に1文字表示する        \\
F806H & \verb/putstr(char[] str);/     & 画面に文字列を表示する     \\
F808H & \verb/_puthex(int x);/         & 画面に16進数を表示する     \\
F80AH & \verb/readSct(int labH, int LabL, char[] buf);/ & 1セクタリード \\
\hline
\end{tabular}
\end{center}

\begin{center}
{\bf \tac のIPLルーチン変数領域} \\
\begin{tabular}{|c|l|l|}
\hline
番地  & \multicolumn{1}{|c|}{変数} & \multicolumn{1}{|c|}{意味} \\
\hline
FFE0H & hBuf & 1セクタリード用バッファアドレス(IPL実行時のみ使用) \\
FFE1H & vRam & VRAMアドレス(IPL実行時のみ使用)                    \\
FFE2H & cX   & 文字カーソル位置の X 座標(0〜79)                   \\
FFE3H & cY   & 文字カーソル位置の Y 座標(0〜24)                   \\
\hline
\end{tabular}
\end{center}

\subsection{割込みベクタと割込みルーチンの登録}

FFF0H 番地から FFFFH 番地は割込みベクタです。
INT0 〜 INT3、INT12 〜 INT15 は現在のところ使用されていません。
その他は、メモリマップに示された割込み要因から割込みが発生したときの
ジャンプアドレスを書き込んでおきます。

interrupt 関数と、addrof 演算子を用いて
割り込み処理を \cmml で記述することができます。
次のプログラムは、``{\tt Kernel/SioEcho/*/kernel.cmm}'' の一部です。

\begin{mylist}
\begin{verbatim}
/*
 * TaC カーネルの原型
 */
#include "util.h"

// SIO 送信割込み処理ルーチン
interrupt sioTrs() {                 // interrupt 関数!!
  if (ISEmpty(trsBuf)) {             // 送信すべきデータがないなら
    out(0x0001,0x0040);              // 送信割込み禁止
    trsIdle = true;
  } else {                           // そうでなければ
    char c = getBuf(trsBuf);         // 送信バッファから一文字取り出す
    out(0x0000,c);                   // データを送信
  }
}

// SIO 受信割込み処理ルーチン
interrupt sioRcv() {                 // interrupt 関数!!
  char c = in(0x0000);               // 受信データを入力する

  if (0x0020 <= c && c <= 0x007e) {  // 印刷可能な ASCII 文字なら
    putBuf(rcvBuf, c);               // 受信バッファへ転記
  }
  ...
}

void main() {
  int[] vect = __ItoA(0xfff0);       // 割込みベクタの配列
  vect[8] = addrof(sioRcv);          // #8(sio受信)ベクタの初期化
  vect[9] = addrof(sioTrs);          // #9(sio送信)ベクタの初期化

  initSIO();                         // SIO 関係の初期化
  ei();                              // 割込み許可
  ...
}
\end{verbatim}
\end{mylist}
