% 
% $Id$
%
\chapter{ファイルフォーマット}

\tac 開発環境で使用する2種類のバイナリ形式ファイルの内容について解説します。

\section{{\tt .o} 形式ファイル}
\label{app:oformat}

\verb/as--/ の出力ファイル形式です。
内容は再配置可能な機械語です。
\verb/ld--/ は、この形式の複数の入力ファイルを一つに結合します。
結合されたファイルも同じ{\tt .o} 形式ファイルです。

\subsection{ファイル形式}

以下に、ファイルの形式を図示します。
ヘッダは各セグメントの長さ等を格納した8ワード固定長の情報です。
テキストセグメントは機械語プログラムと文字列定数を格納します。
データセグメントは初期化データを格納します。
テキストリロケーションはテキストセグメントの再配置情報を、
データリロケーションはデータセグメントの再配置情報を格納します。
シンボルテーブルは
シンボル(\cmm プログラムの大域変数名や関数名、アセンブリ言語のラベル)の
値(アドレス)を格納します。
この情報を使用して、\verb/ld--/が{\tt .o} 形式ファイルの結合をします。
文字列テーブルはシンボルテーブルに格納されるシンボルの綴を格納します。

\begin{myminipage}
\begin{tabular}{|c|}
\hline
ヘッダ(8ワード) \\
\hline
\\
テキストセグメント \\
\\
\hline
\\
データセグメント \\
\\
\hline
テキストリロケーション \\
\hline
データリロケーション \\
\hline
シンボルテーブル \\
\hline
文字列テーブル \\
\hline
\end{tabular}
\\\vspace{0.2cm}
{\bf {\tt .o}ファイルフォーマット}
\end{myminipage}

\subsection{ヘッダ}

{\tt .o}形式ファイルのヘッダは次の構造体により定義されます。
ただし、ここで uint 型は符号無しの int 型とします。

\begin{mylist}
\begin{verbatim}
struct Exec {
  uint magic;     // マジックナンバー(0x0107)
  uint text;      // テキストセグメントサイズ(ワード単位)
  uint data;      // 初期化データセグメントサイズ(ワード単位)
  uint bss;       // 非初期化データセグメント(BSS)サイズ(ワード単位)
  uint syms;      // シンボルテーブルサイズ(ワード単位)
  uint entry;     // 常に 0
  uint trsize;    // テキストリロケーションサイズ(ワード単位)
  uint drsize;    // データリロケーションサイズ(ワード単位)
};
\end{verbatim}
\end{mylist}

\subsection{リロケーションレコード}

テキストリロケーション、データリロケーション領域には、
再配置情報を記録したリロケーションレコードの表が格納されます。
リロケーションレコードは次の構造体で定義されます。
レコードは 2 ワード長で、
第1ワードが再配置時に書き換えが必要なポインタのセグメント中アドレス、
第2ワードは上位2ビットがポインタの種類(type)データ、
下位14ビットがポインタ値を格納するシンボルの
シンボルテーブル上の添字(idx)データになります。

\begin{mylist}
\begin{verbatim}
struct Reloc {
  uint addr;       // 再配置すべきポインタのセグメント内アドレス
  uint type: 2,    // ポインタの型
       idx: 14;    // シンボルテーブルのポインタが登録されている位置
};

// type の意味
#define UNDEF 0    // 未定義
#define TEXT  1    // テキストセグメント
#define DATA  2    // データセグメント
#define BSS   3    // コモン
\end{verbatim}
\end{mylist}

\subsection{シンボルテーブル}

シンボルテーブルは、シンボルとアドレスを対応付けします。
アセンブラが処理したシンボル(ラベル)の中で EQU ラベルを除くものが全て
シンボルテーブルに出力されます。
また、アセンブラのソースプログラムのファイル名もシンボルテーブルに出力され、
\verb/objbin--/ プログラムが未定義シンボルを発見した場合、
ファイル名とともにエラー表示ができるようにしています。
ファイル名は '@' を先頭に付加してシンボルテーブルに登録されます。

シンボルは 1 文字目によって意味付けがされています。
意味は、「'\verb/@/' : ファイル名」、「'\verb/./' : ローカル名」です。

シンボルテーブルを構成するシンボルレコードの構成を次に示します。
シンボルレコードは 2 ワード長です。
第1ワードの上位 2 ビットがシンボルの種類(type)を表し、
下位14ビットが文字列テーブル上で
シンボルの綴が格納されている場所を表す添字データ(sIdx)を格納します。
第2ワード(val)はシンボルの値をセグメント内オフセットで表します。
ただし、未定義(UNDEF)シンボルの場合は 0 、
コモン(BSS)シンボルの場合は領域のサイズを格納します。

\verb/ld--/ は、複数の入力ファイル中に同名のコモンシンボルを発見した場合、
それらを一つの領域に重ね合わせます。
このとき、領域のサイズは重ね合わせたシンボルの中で最大のものと同じになります。
また、
一つのデータセグメントシンボルと一つ以上のコモンシンボルが見つかった場合は、
データセグメントシンボルに集約します。
未定義シンボル同士は一つの未定義シンボルに、
未定義シンボルと他の種類のシンボルは未定義ではない方のシンボルに集約します。
これ以外に同名のシンボルが見つかった場合は、
エラー(シンボルの2重定義)になります。

\begin{mylist}
\begin{verbatim}
struct Symbol {
  uint type: 2,    // シンボルの型
       sIdx:14;    // シンボル名称の文字列テーブル上の位置
  uint val;        // シンボルの値
};

// type の意味
#define UNDEF 0    // 未定義
#define TEXT  1    // テキストセグメント
#define DATA  2    // データセグメント
#define BSS   3    // コモン
\end{verbatim}
\end{mylist}

\subsection{文字列テーブル}

文字列テーブルはシンボルの綴を格納します。
文字列テーブルの内容は \verb/'\0'/ で終端された \cmml 型文字列の繰返しです。
\cmml 型文字列は、1文字を16ビットで表現します。
ヘッダに文字列表のサイズは格納されていないので、
ファイルサイズから文字列テーブルのサイズを知る必要があります。

シンボルテーブルに同じ綴のシンボルが複数ある場合
(「\verb/./」で始まるローカルシンボルは同じ綴の可能性がある)は、
メモリの節約のため、
複数のシンボルレコードで同じ文字列表エントリーを共用します。
\cmm コンパイラが自動的に生成するローカルラベルは、毎回、同一のパターンなので、
多くのシンボルレコードで文字列表エントリーの共用がされます。

\section{{\tt .bin} 形式ファイル}
\label{app:bformat}

ロードアドレスが確定した機械語プログラムを格納するためのファイル形式です。
この形式のプログラムはメモリにロードするだけで実行可能です。
\verb/objbin--/ プログラムによって{\tt .o}形式ファイルを
{\tt .bin}形式ファイルに変換します。
未定義シンボルを含む{\tt .o}形式ファイルは、
{\tt .bin}形式ファイルに変換することができません。

\subsection{ファイル形式}

8ビット版の TeC で使用してきた {\tt .bin} 形式ファイルを
単純に16ビットに拡張した形式のファイルです。
以下に、ファイルの形式を図示します。
第1ワードが機械語プログラムのロードアドレス、
第2ワードが機械語プログラムの長さ(ワード単位)を表現します。
第3ワードから先は機械語プログラム本体です。

\begin{myminipage}
\begin{tabular}{|c|}
\hline
ロードアドレス(1ワード) \\
\hline
プログラム長(1ワード) \\
\hline
\\
機械語プログラム \\
(プログラム長ワード)\\
\\
\hline
\end{tabular}
\\\vspace{0.2cm}{\bf {\tt .bin}形式ファイル} \\
\end{myminipage}

ブートプログラム(``{\tt boot.bin}'')、
OS カーネルプログラム(``{\tt kernel.bin}'')が、
この形式のプログラムファイルです。
