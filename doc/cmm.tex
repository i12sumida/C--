%
% C-- 言語 ver.3 の解説書
%
% 2016.03.12 V.3.0.0 : ver.3 用に改訂
% 
% $Id$
%
\documentclass[11pt,a4j,twoside,dvipdfmx]{jbook}
\usepackage{graphicx}
\usepackage{amsmath}
\usepackage{amssymb}
\usepackage{bm}
\usepackage{fancybox}
\usepackage{fancyhdr}
\usepackage{lastpage}
\usepackage[usenames]{color}
\usepackage{multicol}
\usepackage{bigfoot}  % \footnote{} 中で \verb/../ が利用可能になる。
%\usepackage{listings,jlisting}
\usepackage{url}

%----------------------------------------------------------------------
% ハイパーリンク設定(印刷用では色使いに注意)
\usepackage[colorlinks,linkcolor=blue,urlcolor=blue]{hyperref}  % PDF用
\usepackage{hyperref}                                           % 印刷用
\usepackage{pxjahyper}

%----------------------------------------------------------------------
\newcommand{\myfigure}[5]{
\begin{figure}[#1]
%\begin{figure}[tbp]
\begin{center}
\includegraphics[width=#2]{#3}
\vspace{-.3cm}
\caption{#4}
\label{fig:#5}
\end{center}
\end{figure}
}

%----------------------------------------------------------------------
\newcommand{\myfigureA}[5]{
\begin{figure*}[#1]
%\begin{figure*}[tbp]
\begin{center}
\includegraphics[width=#2]{#3}
\vspace{-.3cm}
\caption{#4}
\label{fig:#5}
\end{center}
\end{figure*}
}

%----------------------------------------------------------------------
\newcommand{\figref}[1]{図~\ref{fig:#1}}
\newcommand{\tabref}[1]{表~\ref{tab:#1}}
\newcommand{\ver}{Ver. 3.0.0}
\newcommand{\cmm}{{\tt C--}}
\newcommand{\cmml}{\cmm 言語}
\newcommand{\cmmc}{{\tt c--}}
\newcommand{\vcmmc}{{\tt vm-c--}}
\newcommand{\ccmmc}{{\tt c-c--}}
\newcommand{\cl}{{\tt C}言語}
\newcommand{\javal}{{\tt Java}言語}
\newcommand{\tac}{TaC}
\newcommand{\tacos}{TacOS}
\newcommand{\util}{{\tt Util--}}
\newcommand{\as}{{\tt as--}}
\newcommand{\ld}{{\tt ld--}}
\newcommand{\objbin}{{\tt objbin--}}
\newcommand{\objexe}{{\tt objexe--}}
\newcommand{\size}{{\tt size--}}
%\newcommand{\ul}{\underline{~~}}
\newcommand{\ul}{\_}
\newcommand{\lw}[1]{\smash{\lower2.0ex\hbox{#1}}}

%----------------------------------------------------------------------
\newenvironment{mylist}
{%\vspace{0.3cm}
\begin{center}\begin{tabular}{|c|}\hline\\\begin{minipage}{0.9\textwidth}}
{\end{minipage}\\\\\hline\end{tabular}\end{center}%\vspace{0.3cm}
}

%----------------------------------------------------------------------
\newenvironment{myminipage}
{\begin{minipage}{14.5cm}\vspace{0.2cm}\begin{center}}
{\end{center}\vspace{0.2cm}\end{minipage}}

%----------------------------------------------------------------------
\newenvironment{mytable}[3]
{\small\begin{table}[tbp]\begin{center}\caption{#2}\vspace{0.2cm}\label{tab:#3}}
{\end{center}\end{table}}

%----------------------------------------------------------------------
\begin{document}
%\setlength{\textwidth}{14cm}
%\setlength{\textheight}{22cm}
\setlength{\oddsidemargin}{10pt}
\setlength{\evensidemargin}{-10pt}
%\setlength{\topmargin}{10pt}
%\setlength{\marginparsep}{15pt}
%\setlength{\parskip}{0.3cm}
\setlength{\headsep}{1cm}
\frontmatter

% 表紙
\title{プログラミング言語 \cmm \\\ver}
\author{徳山工業高等専門学校\\情報電子工学科}
\date{}

\maketitle

% 著作権表示
\thispagestyle{empty}
~
\vfill
\begin{flushleft}
Copyright \copyright ~~ 2016 by \\
Dept. of Computer Science and Electronic Engineering, \\
Tokuyama College of Technology, JAPAN
\end{flushleft}

\vspace{0.8cm}

本ドキュメントは*全くの無保証*で提供されるものである.上記著作権者および
関連機関・個人は本ドキュメントに関して,その適用可能性も含めて,いかなる保証
も行わない.また,本ドキュメントの利用により直接的または間接的に生じたいかな
る損害に関しても,その責任を負わない.
\setcounter{page}{0}

% 目次
\tableofcontents

% 本文
\mainmatter
% 
% $Id$
%
\chapter{はじめに}

%\section{\cmm の特徴}
\cmml は \cl に似た小さなシステム記述用言語です。
徳山高専教育用 PC(\tac) のシステム記述用言語として開発されました。
\cmml は次に挙げる項目を満たすことを目標に設計されています。

\begin{description}
\item[「学習が容易な言語であること」]
\cmml は \cl をお手本にしていますが、
\cl の難しい部分を取り去り簡単に理解できるようになっています。
まず、無くても我慢できそうな文法は、思い切り良く省略しています。
例えば、\cl では多次元配列の形式にいくつかのレパートリーがありました。
しかし、Java 言語にはレパートリーはありません。
\cmml は Java 言語に倣い多次元配列のレパートリーを認めません。

また、\cl の混乱を招きそうな文法仕様を取り入れないように注意しています。
例えば、\cl の配列は関数に渡されるとポインタとして扱われます。
つまり、関数に渡すと型が変化してしまいます。
このような仕様は、初心者が言語を学習する場合に混乱を招きます。
\cmml は Java の参照の考えかたを取り入れ、
配列は一貫して参照型として取り扱われます。

その他にも、\cl の難しい文法を取り去る工夫をしてあります。
Java 言語と似た仕様にすることにより、
Java 言語でプログラミングの入門をした人が学習しやすくなっています。

\item[「実用的なシステム記述言語であること」]
\cmml は \tac のシステム記述言語として、
\tac の OS や \cmm コンパイラを記述することを目標にしています。
そのため、無闇に文法を簡単化すること無く、
実用的に使用するために必要な文法は残してあります。
例えば、制御文は{\tt if}、{\tt while}、{\tt for}、{\tt do-while}、
{\tt return}、{\tt break}、{\tt continue} 等が一通り準備されています。

また、なるべく効率の良いオブジェクトコードを出力する努力をしています。

\item[「\tac 上で実行可能なこと」]
最終的に \tac 上でセルフ開発環境を構築することを目標としています。
\cmm コンパイラは、セルフ開発環境の中核になるコンポーネントです。
そこで、\cmm コンパイラは \tac の限られた主記憶(64kW)で実行できるように
メモリを節約するような設計がされています。
コンパイラのプログラムが小さいこともそうですが、
単一の名前表で変数、関数、
構造体を管理する等してデータ構造も小さくするようにしています。

\item[「コンパイラを教材として使用できること」]
\cmm コンパイラは、高専や大学の学生が、
コンパイラの教材として使用することを想定して開発されました。
そのため、コンパイラがコンパクトに記述でき、
コンパイラのソースコードを学生が読めることも目標になっています。
2009年4月現在の \cmm コンパイラは \cl で 1,700 行程度で記述されています。

\end{description}
 % はじめに
% 
% 2章 C--言語開発環境のインストール
%
\chapter{\cmml 開発環境のインストール}

\cmml を体験するために、自分のパソコンで\cmm コンパイラを
使用できるようにしましょう。
{\bf 以下の順番で}作業し、開発環境をインストールしてください。


\section{ユーティリティのインストール}

{\tac}用のプログラムを作成するために、まず、
{\cmm}コンパイラ用ユーティリティをインストールする必要があります。
ソースコードは
\url{https://github.com/tctsigemura/Util--/}から入手します。

ダウンロードした配布物を展開し
「{\util}解説書」({\tt Util--/doc/umm.pdf})の手順に従いインストールします。
{\as}、{\ld}、{\objbin}、{\objexe}、{\size}の五つのプログラムが
{\tt /usr/local/bin}にインストールされます。

\section{コンパイラのインストール}
\label{chap2:compilerl}

{\cmm}コンパイラのソースコードは
\url{https://github.com/tctsigemura/C--/}から入手します。
ダウンロードした配布物を展開し以下の順にインストールします。

\subsection{コンパイラ本体のインストール}

%コンパイラ本体をインストールします。
{\tt C--/src}ディレクトリに移動し以下のように操作します。

\begin{mylist}
\begin{verbatim}
$ make
cc -std=c99 -Wall  -DDATE="\"`date`\"" -DVER=\"3.1.7\" ...
...
$ sudo make install
Password:
install -d -m 755 /usr/local/bin
install -m 755 c-- /usr/local/bin
install -m 755 vm-c-- /usr/local/bin
...
\end{verbatim}
\end{mylist}

以上で、{\cme}、{\cmc}、{\cmi}、{\cmv}、{\cmmc}、{\ccmmc}、{\icmmc}、{\vcmmc}の
八つのプログラムが作成され\verb;/usr/local/bin;にインストールされました。
{\cme}、{\cmc}、{\cmi}、{\cmv}は、
{\cmml}をコンパイルするために使用するシェルスクリプトです。
これらのシェルスクリプトは、
与えられたファイルの拡張子から処理するべき手順を判断し、
コンパイラやユーティリティを自動的に実行します。

\begin{itemize}
\item {\cme}は、
{\cmm}プログラムを{\tac}で実行できる「{\tt .exe}」ファイルに変換します。
\item {\cmc}は、
{\cmm}プログラムをUNIXやMacで実行できるファイルに変換します。
\item {\cmi}は、
{\cmm}プログラムを中間コードに変換します。
\item {\cmv}は、
{\cmm}プログラムをスタックマシンのニーモニックに変換します。
\item {\cmmc}は、{\tac}用の{\cmml}コンパイラ本体です。
{\cmm}プログラムを{\tac}のアセンブリ言語に変換します。
通常{\cme}から呼び出され、ユーザが直接使用することはありません。
\item {\ccmmc}は、{\cmm}プログラムを{\cl}に変換するトランスレータです。
変換された{\cl}プログラムはUNIXやMacで実行できます。
通常{\cmc}から呼び出され、ユーザが直接使用することはありません。
\item {\icmmc}は、コンパイラの勉強をしたい人のために、
中間言語(\pageref{app:vm}ページ)を出力して見せるコンパイラです。
通常{\cmi}から呼び出され、ユーザが直接使用することはありません。
\item {\vcmmc}は、コンパイラの勉強をしたい人のために、
中間言語(\pageref{app:vm}ページ)をよく反映した、
仮想スタックマシンのニーモニックを出力して見せるコンパイラです。
通常{\cmv}から呼び出され、ユーザが直接使用することはありません。
\end{itemize}

なお、これら八つのプログラムの使用方法は、
コマンドリファレンス(\pageref{app:command}ページ)で紹介します。

\subsection{ヘッダファイルのインストール}

{\cmml}プログラム用のヘッダファイルをインストールします。
\verb;C--/include;ディレクトリに移動し以下のように操作します。
\verb;/usr/local/cmmInclude;にインストールされます。

\begin{mylist}
\begin{verbatim}
$ sudo make install
Password:
install -d -m 755 /usr/local/cmmInclude
rm -f /usr/local/cmmInclude/*.hmm
install -m 644 *.hmm /usr/local/cmmInclude
\end{verbatim}
\end{mylist}

\subsection{ライブラリのインストール}

\verb;C--/lib;ディレクトリに移動しまずライブラリをコンパイルします。
以下のように操作します。

\begin{mylist}
\begin{verbatim}
$ make
as-- crt0.s
cm2e -c  ctype.cmm
cm2e -c  stdio.cmm
cm2e -c  printf.cmm
cm2e -c  stdlib.cmm
cm2e -c  string.cmm
as-- syslib.s
ld-- libtac.o crt0.o ctype.o stdio.o printf.o stdlib.o ...
\end{verbatim}
\end{mylist}

以上で、\verb;libtac.o;が作成されました。
これが、{\tacos}のアプリケーションプログラムにリンクされるライブラリです。
次に、インストールします。操作は次の通りです。

\begin{mylist}
\begin{verbatim}
$ sudo make install
Password:
install -d -m 755 /usr/local/cmmLib
install -m 644 libtac.o /usr/local/cmmLib
install -m 644 cfunc.hmm /usr/local/cmmLib
install -m 644 wrapper.c /usr/local/cmmLib
install -m 644 wrapper.h /usr/local/cmmLib
ln -fs wrapper.h /usr/local/cmmLib/crt0.h
\end{verbatim}
\end{mylist}

\verb;/usr/local/cmmLib;ディレクトリに、
いくつかのファイルがインストールされました。
\verb;libtac.o;は{\cmml}で記述した{\tacos}用のライブラリ関数です。
\verb;cfunc.hmm;は{\cmm}プログラムがパソコンで実行される際に
{\cmml}ライブラリ関数の代用として使用される{\cl}ライブラリ関数を
定義(\verb;#define;)します。
\verb;wrapper.c;は{\cmm}プログラムがパソコンで実行される際に使用される
{\cl}で記述された{\cmm}ライブラリ関数です。
\verb;cfunc.hmm;による置き換えができない関数が記述してあります。

\section{サンプルプログラムのコンパイル}

\verb;C--/samples/hello;ディレクトリに移動します。
{\tt hello.cmm}プログラムが準備してあります。
これをコンパイルして実行してみましょう。
{\tt hello.cmm}プログラムは以下のような{\cmml}プログラムです。

\begin{mylist}
\begin{verbatim}
//
// hello.cmm : C--のサンプルプログラム
//
#include <stdio.hmm>

public int main() {
  printf("hello,world\n");
  return 0;
}
\end{verbatim}
\end{mylist}

\subsection{コンパイルしてPCで実行}

\verb;C--/samples/hello;ディレクトリで以下のように操作すると、
{\tt hello.cmm}プログラムをコンパイルしてPCで実行することができます。
なお、{\cmc}は、「\ref{chap2:compilerl} コンパイラのインストール」で
インストールしたシェルスクリプトです。
コンパイラとユーティリティを自動的に起動して、
{\cmml}プログラムをPCで実行可能なプログラムに変換します。

\begin{mylist}
\begin{verbatim}
$ cm2c -o hello hello.cmm 
$ ./hello
hello,world
\end{verbatim}
\end{mylist}

\subsection{コンパイルしてTaCで実行}

\verb;C--/samples/hello;ディレクトリで以下のように操作すると、
{\tt hello.cmm}プログラムをコンパイルして{\tac}で実行可能な
{\tt hello.exe}を作ることができます。
なお、{\cme}は、「\ref{chap2:compilerl} コンパイラのインストール」で
インストールしたシェルスクリプトです。
コンパイラとユーティリティを自動的に起動して、
{\cmml}プログラムを{\tac}で実行可能なプログラムに変換します。
{\tt hello.exe}をメモリカードにコピーすると{\tac}で実行できます。

\begin{mylist}
\begin{verbatim}
$ cm2e -o hello hello.cmm 
$ ls -l
...
-rw-r--r--  1 sigemura  staff   138 May  5 01:03 hello.cmm
-rw-r--r--  1 sigemura  staff  8164 May 28 23:28 hello.exe
-rw-r--r--  1 sigemura  staff  5862 May 28 23:28 hello.map
\end{verbatim}
\end{mylist}

\subsection{いろいろな中間ファイル}

上の例で示した他に、{\tt hello.cmm}から数種類のファイルを作ることができます。

\begin{itemize}

\item {\tt hello.c}は、
{\cmc}を{\tt -S}オプション付きで実行して
{\tt hello.cmm}を変換した{\cl}ファイルです。
「\verb/$ cm2c -S hello.cmm/」のように実行します。
上記の{\tt hello}は、これを{\cl}コンパイラでコンパイルしたものです。

\begin{mylist}
\begin{verbatim}
#include <stdio.h>
#define _cmm_0S "hello,world\n"
int main(){
printf(_cmm_0S);
return 0;
}
\end{verbatim}
\end{mylist}

\item {\tt hello.s}は、
{\cme}を{\tt -S}オプション付きで実行して
{\tt hello.cmm}を変換した{\tac}のアセンブリ言語プログラムです。
「\verb/$ cm2e -S hello.cmm/」のように実行します。
上記の{\tt hello.exe}は、
このファイルから{\cmm}コンパイラ用のユーティリティ({\tt Util--})を
用いて作成されました。

\begin{mylist}
\begin{verbatim}
_stdin  WS      1
_stdout WS      1
_stderr WS      1
.L1     STRING  "hello,world\n"
_main   PUSH    FP
        LD      FP,SP
        CALL    __stkChk
        LD      G0,#.L1
        PUSH    G0
        CALL    _printf
        ADD     SP,#2
        LD      G0,#0
        POP     FP
        RET
\end{verbatim}
\end{mylist}

\item {\tt hello.i}は、{\tt hello.cmm}を{\cmi}により
中間言語に変換したプログラムです。\\
「\verb/$ cm2i hello.cmm/」のように実行します。
中間言語の詳細は、
\pageref{app:vm}ページで紹介します。

\begin{mylist}
\begin{verbatim}
vmNam(25)
vmWs(1)
vmNam(26)
vmWs(1)
vmNam(27)
vmWs(1)
vmLab(1)
vmStr("hello,world\n")
vmNam(65)
vmEntry(0)
vmLdLab(1)
vmArg()
vmCallF(1, 62)
vmPop()
vmLdCns(0)
vmMReg()
vmRet()
\end{verbatim}
\end{mylist}

\item {\tt hello.v}は、{\tt hello.cmm}を{\cmv}により
仮想スタックマシンのニーモニックに変換したプログラムです。
中間言語とニーモニックは一対一に対応しています。\\
「\verb/$ cm2v hello.cmm/」のように実行します。
仮想スタックマシンのニーモニックの詳細は、
\pageref{app:vm}ページで紹介します。

\begin{mylist}
\begin{verbatim}
_stdin
        WS      1
_stdout
        WS      1
_stderr
        WS      1
.L1
        STRING  "hello,world\n"
_main
        ENTRY   0
        LDC     .L1
        ARG
        CALLF   1,_printf
        POP
        LDC     0
        MREG
        RET
\end{verbatim}
\end{mylist}
\end{itemize}

\subsection{Makefileファイル}

{\tt Makefile}は、
コンパイル手順を記述したファイルです。
数種類の例を示しますので参考にしてください。

\begin{mylist}
\begin{verbatim}
#
# Makefile.tac : C--言語からTacの実行形式に変換する手順
#

hello.exe: hello.cmm
        cm2e -o hello hello.cmm

clean:
        rm -f hello.exe hello.map
\end{verbatim}
\end{mylist}

\begin{mylist}
\begin{verbatim}
#
# Makefile.unix : C--言語からMacOSやUNIXの実行形式に変換する手順
#

hello: hello.cmm
        cm2c -o hello hello.cmm

clean:
        rm -f hello
\end{verbatim}
\end{mylist}

\begin{mylist}
\begin{verbatim}
#
# C-- 言語から a.out、.exe、 .v ファイルを作る手順
#

all: hello hello.exe hello.v

# UNIX, Mac の a.out へ変換
hello: hello.cmm
        cm2c -o hello hello.cmm

# TacOS の実行形式を作る
hello.exe: hello.cmm
        cm2e -o hello.exe hello.cmm

# C--コンパイラの中間言語に変換する
hello.v: hello.cmm
        cm2v hello.cmm

clean:
        rm -f hello hello.c hello.s hello.v *.lst *.sym *.map \
	*.o *.exe *~

\end{verbatim}
\end{mylist}
 % インストール
% 
%  3章 C--言語の仕様
%
\chapter{\cmml の仕様}

システム記述言語として実績のある \cl を参考に\cmml は設計されました。
しかし、\cl は設計が古く分かりにくい仕様が多い言語でもあります。
そこで、設計が割と新しい \javal を参考に \cl の問題点を避けるようにしました。
最終的に \cmml は、
{\bf 「\javal の特徴を取り入れた簡易\cl」} になりました。

\section{コメント}

\cmml では、2種類のコメントを使用できます。
一つは \verb!/* 〜 */! 形式のコメント、
もう一つは \verb!// 〜 行末! 形式のコメントです。
以下に例を示します。
コメントは、空白を挿入できるところならどこにでも書くことができます。

\begin{mylist}
\begin{verbatim}
/*
 *  コメントの例を示すプログラム
 */
#include <stdio.hmm>             // printf を使用するために必要

public int main( /* 引数なし */ ) {
  printf("%d\n",10);
  return 0;                      // main は int 型なので
}
\end{verbatim}
\end{mylist}

\section{プリプロセッサ}

\cmml の仕様ではありませんが、
通常 \cmm コンパイラは \cl 用のプリプロセッサと組み合わせて使用します。
前の章に Makefile の例がありましたが、
プリプロセッサとコンパイラをパイプで接続して使用します。
プリプロセッサのお陰で、
\verb/#define/、\verb/#include/等のディレクティブを使用することができます。

\cmm コンパイラはプリプロセッサが処理した結果を受け取ります。
受け取った入力の、ヘッダファイル内部に相当する部分にエラーを見つけた場合でも、
正しくエラー場所をレポートできます。
また、\cl を出力するトランスレータとして動作する場合は、
出力した \cl プログラム中に適切な\verb/#include/ディレクティブを出力したり、
ヘッダファイル内部を省略して出力する等の処理をしています
\footnote{前章の hello.c の例を見てください。}
。

\section{データ型}

\cmml のデータ型は、\javal と同様に基本型と参照型に大別できます。
基本型には、 int 型、 char 型、 boolean 型の3種類があります。
参照型には、配列型と構造体型があります。
参照型は \javal の参照型と良く似た型です。

\subsection{基本型}

int 型は整数値の表現に使用します。
char 型は文字の表現に使用します。
boolean 型は true と false のどちらかの値を取る論理型です。
\javal の boolean 型と同じものです。
違う型の間での代入はできません。
また、自動的な型変換も行いません
\footnote{この仕様は厳しすぎました。将来、一部の自動的な型変換は許可します。}
。

\subsubsection{int 型}
int型は16bit 符号付き2の補数表現2進数です。
(トランスレータ版では\cl の int 型に置き換えますので、
Cコンパイラに依存します。)
-32768 から 32767 までの範囲の数値を表現することができます。
int 型変数は次のように宣言します。

\begin{mylist}
\begin{verbatim}
// int 型変数を宣言した例
int a;                       // int 型のグローバル変数
int b = 10;                  // 初期化もできる
public int main() {
  int c;                     // ローカル変数
  int d = 20;                // 初期化もできる
  return 0;
}
\end{verbatim}
\end{mylist}

int 型の定数は、
上のプログラム例にあるように 「\verb/10/」、「\verb/20/」と書きます。
また、16進数、8進数で書くこともできます。上の例と同じ値を、
16進数では「\verb/0xa/」、「\verb/0x14/」のように、
8進数では「\verb/012/」、「\verb/024/」のように書きます。
int 型は、四則演算、ビット毎の論理演算、シフト演算、比較演算等の
計算に使用できます。

\subsubsection{char 型}
char 型は文字を格納するデータ型です。
内部表現は 8bit の ASCII コードですが整数型の一種ではなく、
int 型との自動的な型変換や四則演算等ができません。
char 型を用いてできる計算は同値比較だけです
\footnote{この仕様も厳しすぎました。将来、大小比較も許可する予定です。}
。
「\verb/chr/演算子」、「\verb/ord/演算子」を用いた明示的な型変換により、
char 型の文字と int 型の ASCII コードの間で相互変換ができます。

\begin{mylist}
\begin{verbatim}
// char 型変数を宣言した例
char a;                      // char 型のグローバル変数
char b = 'A';                // 初期化もできる
public int main() {
  char c;                    // ローカル変数
  char d = 'D';              // 初期化もできる
  int  i = ord(b);           // char 型から ASCII コードに変換
  c = chr(i);                // ASCII コードから char 型に変換
  return 0;
}
\end{verbatim}
\end{mylist}

char 型の定数は、
上のプログラム例にあるように 「\verb/'A'/」、「\verb/'D'/」と書きます。
制御文字を表現するために\tabref{chap3:escape}の
エスケープ文字が用意されています
\footnote{正確には、
コンパイラではなくアセンブラが、
エスケープ文字の処理をしています。}。


\begin{mytable}{tbhp}{エスケープ文字}{chap3:escape}
\begin{tabular}{ll}
\hline
\multicolumn{1}{c}{エスケープ文字} & \multicolumn{1}{c}{意味} \\
\hline
\verb/\n/ & 改行 \\
\verb/\r/ & 復帰 \\
\verb/\t/ & 水平タブ \\
\verb/\x/16進数 & 文字コードを16進数で直接指定 \\
\verb/\/8進数 & 文字コードを8進数で直接指定 \\
\verb/\/文字 & 印刷可能な文字を指定(\verb/\'/ や \verb/\\/等) \\
\hline
\end{tabular}
\end{mytable}

\subsubsection{boolean型}

\cl では int 型で論理値を表現しました。
そのため、条件式を書かなければならないところに間違って代入式を書いたミスを
発見できず苦労することがよくありました。
\cmml では論理型 boolean を導入したので、
このようなミスをコンパイラが発見できます。

\begin{mylist}
\begin{verbatim}
// C言語でよくある条件式と代入式の書き間違え
if (a=1) {
  ...
}
\end{verbatim}
\end{mylist}

boolean 型は論理演算と同値比較演算のオペランドになることができます。
boolean 型の定数値は、\verb/true/、\verb/false/ と書き表します
\footnote{{\tt true}の内部表現は{\tt 1}、{\tt false}の内部表現は{\tt 0}です。}
。
boolean 型も int 型と互換性がありません。
「\verb/bool/演算子」、「\verb/ord/演算子」に
よる明示的な型変換を用いると int 型との変換が可能です。

次に boolean 型の変数を宣言して使用する例を示します。

\begin{mylist}
\begin{verbatim}
// boolean 型の使用例
boolean b = true;     // true は定数

public int main() {
  boolean c = x==10;  // 比較演算の結果は boolean 型
  b = b && c;         // 論理演算の結果も boolean 型
  if (b) {            // 論理値なので条件として使用できる
    b = false;        // false も定数
    ...
  }
  int i = ord(b);     // boolean 型から内部表現への変換
  b = bool(i);        // 内部表現から boolean 型への変換
  ...
}
\end{verbatim}
\end{mylist}

\subsection{参照型}
\label{chap3:ref}

参照型は \cl のポインターに似た型です。
\javal の参照型と非常に良く似ています。
参照型には配列型と構造体型があります。
参照型の値はインスタンスのアドレスです。
参照型の特別な値として何も指していない状態を表す \verb/null/ があります。
\verb/null/ の値はアドレスの \verb/0/ です。
ここでは、配列、多次元配列、文字列、 \verb/void/ 配列、
構造体について解説します。

\subsubsection{配列}
\label{chap3:array}
\cmml でも配列を使用することができます。
配列は「\verb/型名[]/」と宣言します。
例えば、int型やchar型の配列(参照変数)は次のように宣言します。

\begin{mylist}
\begin{verbatim}
// 配列の参照変数を宣言した例
int[] a;
char[] b;
\end{verbatim}
\end{mylist}

\myfigure{tbhp}{5cm}{array.pdf}{配列の構造}{chap3:array}

\figref{chap3:array}のように、
\cmml の配列は参照変数と配列インスタンスの組合せによって実現します。
上の宣言では、参照変数が生成されるだけです。
配列として使用するためには次の例のように
「\verb/array/」や「\verb/malloc/」を用いて
配列インスタンスを割り付ける必要があります。
「\verb/array/」は、配列インスタンスを静的に割り付けます。
「\verb/array/」は、関数の外で宣言される配列だけで使用できます。
「\verb/malloc/」は、配列インスタンスを実行時に動的にヒープ領域に割り付けます。
「\verb/malloc/」で割り付けた領域は、
使用後「\verb/free/」によって解放する必要があります。
%「\verb/free/」し忘れが予想されるので、
%配列は静的に割り付けることをおすすめします。

\begin{mylist}
\begin{verbatim}
// 配列インスタンスを割り付ける例
int[] a, b = array(10);      // 要素数10のint配列領域を割り付ける
public int main() {
  a = malloc(sizeof(int)*5); // 要素数5のint配列領域を割り付ける
  ...
  free(a);                   // malloc した領域は忘れず解放する
  return 0;
}
\end{verbatim}
\end{mylist}

配列は次のようなプログラムでアクセスできます。
添字は 0 から始まります。

\begin{mylist}
\begin{verbatim}
// 配列をアクセスする例
b[9] = 1;
a[0] = b[9] + 1;
\end{verbatim}
\end{mylist}

\subsubsection{多次元配列}

多次元配列は配列の配列として表現されます。
\figref{chap3:array2}に示すように、
1次元配列の参照を要素とした配列を使うと2次元配列になります。
下に、2次元配列を使用するプログラム例を示します。
プログラム中の \verb/a2/ は、\verb/malloc/ によって、
\figref{chap3:array2}のようなデータ構造に作り上げられます。
「\verb/malloc/」を使用する場合は、プログラムが複雑になってしまいますが、
長方形ではない配列も実現できます。
プログラム中 \verb/b2/ は、「\verb/array/」を使用して
多次元配列に必要な配列インスタンスを割り付けた例です。
このように「\verb/array/」を使用すると、
多次元配列に必要な複雑なデータ構造を簡単に割り付けることができます。

%プログラムを簡単に記述するために、
%多次元配列は「\verb/array/」を使用して静的に割り付けることを
%強くおすすめします。

\myfigure{tbhp}{6cm}{array2.pdf}{2次元配列の構造}{chap3:array2}

\begin{mylist}
\begin{verbatim}
// 2次元配列の例
int[][] a2, b2 = array(4,3);       // 2次元配列の領域を割り付ける
public int main() {
  a2 = malloc(sizeof(void[])*4);   // 参照の配列を割り付ける
  for (int i=0; i<4; i=i+1)
    a2[i] = malloc(sizeof(int)*3); // int 型の1次元配列を割り付ける
  ...
  for (int i=0; i<4; i=i+1)
    free(a2[i]);                   // int 型の1次元配列を解放
  free(a2);                        // 参照の配列を解放
  return 0;
}
\end{verbatim}
\end{mylist}

次のプログラムは多次元配列をアクセスする例です。
このプログラムは、\figref{chap3:array2}の右下の要素
(最後の要素)に 1 を代入したあと、
それを使用して \verb/b2/ の要素の値を決めます。

\begin{mylist}
\begin{verbatim}
  a2[3][2] = 1;
  b2[3][2] = a2[3][2] + 1;
\end{verbatim}
\end{mylist}

\subsubsection{文字列}

\cl 同様に、文字列は char 型の配列として表現されます。
文字列は、文字コード \verb/0x00/ の文字で終端されます。
次のプログラム例のように文字列定数を用いることができ、
また、文字列の終端を \verb/'\0'/ との比較で判断できます。

\begin{mylist}
\begin{verbatim}
// 文字列の使用例
char[] str = "0123456789";
void putstr() {
  for (int i=0; str[i]!='\0'; i=i+1)  // 文字列の終端まで
    putchar(str[i]);                  // 一文字づつ出力
}
\end{verbatim}
\end{mylist}

\subsubsection{ void 配列}

\verb/void/型の配列は、どんな参照型とも互換性がある特別な型になります。
\cmml は型チェックが厳しいので、異なる型の間での代入はできません。
しかし、次のプログラム例のように \verb/void/ 型配列を用いることにより、
異なる参照型の間で代入ができます。
また、 \verb/malloc/ 関数や \verb/free/ 関数は、
\verb/void/型配列を使って実現されています。

\begin{mylist}
\begin{verbatim}
// 型変換の例
struct XYZ { int x,y,z; };
XYZ p = { 1, 2, 3 };
void putXYZ() {
  void[] tmp = p;            // tmp にはどんな参照も代入可能
  int[]  xyz = tmp;          // tmp からはどんな参照へも代入可能
  for (int i=0; i<3; i=i+1)
    printf("%d\n",xyz[i]);   // 構造体を配列として無理矢理アクセス
}
\end{verbatim}
\end{mylist}

\subsubsection{構造体}

\label{chap3:struct}
\cmml の構造体は \javal のクラスからメソッドを取り除いたものに良く似ています。
構造体は次のプログラム例のように宣言します。
この例は重連結リストを構成するための \verb/Node/ 型を宣言しています。
\cl の構造体宣言に良く似ていますが、
構造体名が型名になる点と、
構造体が参照型である点が異なります。
%\cl の構造体よりむしろメソッドを含まない \javal の class に似ていると言えます。

\begin{mylist}
\begin{verbatim}
struct Node {
  Nodo next;    // 自身と同じ型を参照
  Nodo prev;    // 自身と同じ型を参照
  int  val;     // ノードのデータ
};
\end{verbatim}
\end{mylist}

\verb/Node/ 型の変数を宣言して使用する例を下に示します。
構造体名が型名になるので \verb/struct/ を書かずに変数宣言します。
構造体メンバは、「\verb/参照.メンバ名/」形式で参照します。
\verb/new/ の代わりに \verb/malloc/ を使用する他は
\javal と良く似た書き方になります。

\begin{mylist}
\begin{verbatim}
public int main() {
  Node start;                         // 重連結リストのルート
  start = malloc(sizeof(Node));       // 番兵をリストに投げ込む
  start.next = start;                 // 番兵を初期化する
  start.prev = start;
  start.val  = 0;
  ...
}
\end{verbatim}
\end{mylist}

\subsubsection{だまし型}

内容不明のまま参照型を宣言できます。
他の言語で記述された関数を呼び出す場合など、
単に参照型として扱うだけで十分な場合があります。
\verb/typedef/の後に名前を書きます。
以下に使用例を示します。

\begin{mylist}
\begin{verbatim}
typedef FILE;   // FILE 型を定義する

void f() {
  FILE fp = fopen(...); // C言語の関数を呼び出す
  fprintf(fp,...);      // C言語の関数を呼び出す
  ...
}
\end{verbatim}
\end{mylist}

\section{関数}
\label{chap3:func}
\cmml でも、\cl 同様に関数を宣言して使用することができます。
\cl と比較して \cmml は厳格さを求めています。
つまり、先に宣言された関数しか呼び出すことができませんし、
引数の個数や型が完全に一致しないとコンパイル時エラーになります。
ライブラリ関数を使用する場合は、
呼び出す前に必ずプロトタイプ宣言をするか、
適切なヘッダファイルをインクルードする必要があります。
また、関数の型を省略することができません。
値を返さない関数は \verb/void/、
整数を返す関数は \verb/int/、
論理値を返す関数は \verb/boolean/ 等と明示する必要があります。

\cl や \javal と同様に関数の仮引数は、関数の自動変数と同じように使用できます。
可変個引数の関数を宣言することもできます。
可変個引数関数の仮引数は「\verb/.../」と書きます。
可変個引数関数の内部では、
\pageref{chap4:args}ページで説明する
\ul args 関数を使用して引数をアクセスします
\footnote{\cl へのトランスレータ版では、
可変個引数関数を定義することができません。
\cl の可変個引数関数(\verb/print/等)を呼び出すことだけできます。}。
次にプログラム例を示します。

\begin{mylist}
\begin{verbatim}
#include <stdio.hmm>       // printf のプロトタイプ宣言が含まれる
int f() {                  // 引数の無い関数
  return 1;                // void 型以外の関数は必ずreturnが必要
}

void g(int x) {
  x = x * x;               // 仮引数は変数のように使用できる
  printf("%04x\n", x);     // プロトタイプ宣言が必要
}

public int main() {
  int x = f();             // 関数の呼び出し
  g(x);                    // 引数の型と個数が一致する必要がある
  return 0;                // void 型以外の関数は必ずreturnが必要
}
\end{verbatim}
\end{mylist}

\section{interrupt関数}
%特殊な種類の関数として interrupt 関数があります。
interrupt 関数は、OSカーネル等の割込みハンドラを\cmml
で記述するために用意しました。
コンパイラに\verb/-K/オプションを与えないと使用できません。

%普通の関数と異なり、
interrupt 関数はCPUのコンテキスト(フラグ、レジスタ等)を全く破壊しません。
関数の入口でコンテキストをスタックに保存し、出口で復旧します。
割込みにより起動される関数なので、
プログラムから呼び出すことはできません。
仮引数を宣言することもできません。
次の例のように関数型の代わりに \verb/interrupt/ と書きます。
例中の \verb/main/ 関数のように、
addrof 演算子(\pageref{chap3:addrof}ページ参照)や
\ul iToA 関数(\pageref{chap4:itoa}ページ参照)を使用して、
割込みベクタに interrupt 関数を登録します。

\begin{mylist}
\begin{verbatim}
interrupt timerHdr() {         // タイマー割込みハンドラのつもり
  ...
}
public int main() {
  int[] vect = _iToA(0xffe0);  // vect は割込みベクタの配列
  vect[4] = addrof(timerHdr);  // vect[4] はタイマー割込みのベクタ
  ...
}
\end{verbatim}
\end{mylist}

\section{変数}

\cmml の変数は静的な変数と自動変数の2種類です。
関数の外部で宣言した変数は全て静的な大域変数になります。
逆に、関数の内部で宣言した変数は
全てブロック内でローカルな自動変数になります。
関数の内部では、どこでも変数宣言が可能です。
ローカル変数の有効範囲はブロックの終わりまでです。
同じ名前の変数があった場合はローカル変数が優先されます。
ローカル変数同士の名前の重複は認めません(\javal と同じ規則、\cl と異なる)。
次にプログラム例を示します。

\begin{mylist}
\begin{verbatim}
#include <stdio.hmm>
int   n = 10;                 // 静的な変数
int[] a = array(10);          // 静的な配列
public int main() {
  int i;                      // 関数内ローカル変数
  for (i=0;i<n;i=i+1) {       // 大域変数 n のこと
    int j = i * i;            // ブロック内ローカル変数
    printf("%d\n",j);
    int n = j * j;            // どこでも変数宣言可能
    printf("%d %d\n",j,n);    // ローカルな n のこと
  }                        
  printf("%d\n",n);           // 大域変数 n のこと
  return 0;
}
\end{verbatim}
\end{mylist}

\section{変数の初期化}

基本型の変数は、いつでも宣言と同時に初期化することができます。
参照型の変数は静的に割り付けられる場合だけ初期化できます。
構造体内部に入れ子になった参照型は\verb/null/で初期化することしかできません。
ただし名前表のようなデータ構造を作るために、
文字列での初期化だけは可能にしてあります。

\begin{mylist}
\begin{verbatim}
int     n = 10;                             // 基本型変数の初期化
int[]   a = { 1, 2, 3 };                    // 基本型配列の初期化
int[][] b = {{1,2,0},{1,2,3,0}};            // いびつな配列の初期化
struct List { int  val; List next; };       // 構造体の宣言
List    r = { 1, null };                    // 構造体変数の初期化
struct NameEntry { char[] name, int val; };
NameEntry[] weekTable = {                   // 名前表を作成する例
  {"Sun", 1}, {"Mon", 2}, {"Tue", 3}
};
public int main() {
  int i = 10;                               // 自動変数の初期化
  return 0;
}
\end{verbatim}
\end{mylist}

\section{public 修飾子}

関数、大域変数を他のコンパイル単位から参照できるようにします。
\verb/public/修飾子の付いていない関数や大域変数は
他のコンパイル単位からは見えないので、
重複を心配しないで自由に名前を付けることができます。
main関数はスタートアップルーチンから呼び出されるので、
必ず\verb/public/修飾をしなければなりません。

\begin{mylist}
\begin{verbatim}
int        n = 10;            // 同じ .cmm ファイル内だけで参照可
public int m = 20;            // 他の .cmm ファイルからも参照可

void f() { ... }              // 同じ .cmm ファイル内だけで参照可
public void g() { ... }       // 他の .cmm ファイルからも参照可

public void printf(char[] s, ...);  // ライブラリ関数は public

public int main() {          // main は必ず public
  f();
  g();
  printf("\n");
  return 0;
}
\end{verbatim}
\end{mylist}

\section{演算子}

\cmml には \cl をお手本に一通りの演算子が準備されています。
しかし、コンパイラを小さくする目的で、
レパートリーの多い代入演算子、
前置後置等の組合せが複雑なインクリメント演算子とデクリメント演算子を
省略しました。

\subsection{代入演算}

\cl や \javal には、たくさんの種類の代入演算子があり便利に使用できました。
\cmml では、コンパイラをコンパクトに実装するために代入演算子を「 \verb/=/」の
1種類だけにしました。
\cl や \javal 同様、代入演算の結果は代入した値になります。

\cmml では、代入演算子の左辺と右辺が厳密に同じ型でなければなりません
これは、コンパイル時になるべく多くのバグを発見するための仕様です。
参照型の場合も型が厳密に一致している必要があります。
ただし、\verb/void[]/ だけ例外的にどの参照型とも代入可能です。
自動的な型変換はありません。

\begin{mylist}
\begin{verbatim}
int     a;
boolean b = true;
char    c;
struct X { int r; };
struct Y { int r; };
a = 10;            // 同じ型なので代入可能
c = a;             // (エラー)型が異なるので代入できない
a = b;             // (エラー)型が異なるので代入できない
int i = a = 9;     // 代入演算(a=9)の結果は代入した値(9)
                   // 代入演算の結果(9)を i に代入する
X x = { 1 };
Y y;
y = x;             // (エラー)型が異なるので代入できない
void[] p = x;      // void[] にはどんな参照型も代入可能
y = p;             // void[] はどんな参照型にも代入可能
\end{verbatim}
\end{mylist}

\subsection{数値演算}

int 型データの計算に、2項演算子の
「\verb/+/」(和)、
「\verb/-/」(差)、
「\verb/*/」(積)、
「\verb+/+」(商)、
「\verb/%/」(余)
が使用できます。
その他に、単項演算子
「\verb/+/」、
「\verb/-/」が使用できます。
演算子の優先順位は数学と同じです。
計算(数値演算)をして、計算結果を変数に代入(代入演算)する例を次に示します。

\begin{mylist}
\begin{verbatim}
x = -10 + 3 * 2;
\end{verbatim}
\end{mylist}

\subsection{比較演算}

{\bf (1)整数型の大小比較と同値の判定}、
{\bf (2)参照型、文字型、論理型の同値の判定}ができます。
大小比較の演算子は、
「\verb/>/」(より大きい)、
「\verb/>=/」(以上)、
「\verb/</」(未満)、
「\verb/<=/」(以下) の4種類です。
同値を判定する演算子は、
「\verb/==/」(等しい)、
「\verb/!=/」(異なる)
の2種類です。
比較演算の結果は論理型です。
比較演算の結果を論理型変数に代入することができます。
論理型は if 文や while 文などの条件に使用できます。
次に、比較演算の例を示します。

\begin{mylist}
\begin{verbatim}
int x = 11;
boolean b;
b = x > 10;              // 整数の大小比較
if (b==false) { ... }    // 論理型の同値判定
\end{verbatim}
\end{mylist}

\subsection{論理演算}

論理型のデータを対象にした演算です。
演算結果も論理型になります。
単項演算子「\verb/!/」(否定)、
2項演算子
「\verb/&&/」(論理積)、
「\verb/||/」(論理和)が使用できます。
次に、論理演算の例を示します。
論理型変数 \verb/b/ に比較結果を求めた後で、
\verb/b/ の否定を if 文の条件に使用しています。

\begin{mylist}
\begin{verbatim}
int x = 11;
boolean b;
b = 10 <= x && x <= 20;  // (10<=x) と (x<=20) の論理積を b に代入
if (!b) { ... }          // 論理値の否定
\end{verbatim}
\end{mylist}

\subsection{ビット毎の論理演算}

整数値を対象にした演算です。
演算結果も整数値になります。
単項演算子「\verb/~/」(全ビットを反転)、
2項演算子
「\verb/&/」(ビット毎の論理積)、
「\verb/|/」(ビット毎の論理和)、
「\verb/^/」(ビット毎の排他的論理和)が使用できます。
次に、ビット毎の論理演算の例を示します。
マスクを使用して、変数 \verb/x/ の下位8ビットを取り出して表示します。
\verb/printf/の括弧内で下位8ビットを取り出すために
ビット毎の論理演算をしています。

\begin{mylist}
\begin{verbatim}
int x   = 0xabcd;
int msk = 0x00ff;
printf("%x", x & msk);
\end{verbatim}
\end{mylist}

\subsection{シフト演算}

整数値を対象にした演算です。
演算結果も整数値になります。
2項演算子
「\verb/>>/」(右算術シフト)、
「\verb/<</」(左算術シフト)が使用できます。
算術シフトしかありません。
次に、シフト演算の例を示します。
シフト演算とマスクを使用して、
変数 \verb/x/ の値の上位8ビットを取り出して表示します。
算術シフトですから、マスクを忘れないように注意する必要があります。
\verb/printf/の括弧内で上位8ビットを取り出す計算をしています。

\begin{mylist}
\begin{verbatim}
int x   = 0xabcd;
printf("%x", x >> 8 & 0x00ff);
\end{verbatim}
\end{mylist}

\subsection{参照演算}

配列と構造体をアクセスするための「\verb/[ 添字式 ]/」や「\verb/./」は、
参照を対象にする演算子と考えることができます。
「\verb/[ 添字式 ]/」演算子は、配列参照と添字式から配列要素の値を求めます。
「\verb/./」演算子は、構造体参照とメンバ名からメンバの値を求めます。
配列の配列である多次元配列のアクセスは、
「\verb/[ 添字式 ]/」演算子により取り出した配列要素値が配列参照なので、
更に「\verb/[ 添字式 ]/」演算子により次の配列要素を取り出すと考えます。
実際、\cmm コンパイラの内部でもそのように考えて扱っています。
次に多次元配列や構造体を使用したプログラムの例を示します。
「\ref{chap3:ref} 参照型」に示したプログラム例も参考にしてください。

\begin{mylist}
\begin{verbatim}
// 多次元配列は配列参照の配列と考える
int[][] a = {{1,2},{3,4}};  // 2次元配列を作る
void f() {
  int[] b = a[0];           // 2次元配列の要素は1次元配列
  int   c = b[1];           // 1次元配列の要素は int 型
  int   d = a[0][1];        // c と d は同じ結果になる
}

// 構造体リスト例
struct List {               // リスト構造のノード型
  List next;                // 次のノードの参照
  int  val;                 // ノードの値
};
List a;                     // リストのルートを作る

void g() {
  a = malloc(sizeof(List));       // リストの先頭ノードを作る
  a.val  = 1;
  a.next = malloc(sizeof(List));  // リストの2番目ノードを作る
  a.next.val  = 2;
  a.next.next = null;             // 2番目ノードは参照の参照
}
\end{verbatim}
\end{mylist}

\subsection{sizeof 演算}

変数のサイズを知るための演算子です。
\verb/malloc/ で領域を割り付けるとき使用します。
「\verb/sizeof(型)/」のように使用します。
型が基本型の場合は「変数の領域サイズ」、
構造体の場合は「インスタンスの領域サイズ」をバイト単位で返します。
型が配列型の場合は、何型の配列かとは関係なく「参照の領域サイズ
(アドレスのバイト数)」を返します。
通常、参照の領域サイズは「\verb/sizeof(void[])/」と書きます。
配列インスタンスのサイズが必要なときは、
\verb/sizeof/の値に配列の要素数を掛け算して求めます。
以下に使用例を示します。

\begin{mylist}
\begin{verbatim}
// sizeof 演算子の使用例
int a = sizeof(int);                // int のサイズ(TaC版で 2)
int b = sizeof(char);               // charのサイズ(TaC版で 1)
int c = sizeof(boolean);            // booleanのサイズ(TaC版で 1)
int d = sizeof(void[]);             // 参照のサイズ(TaC版で 2)
struct X { int x; int y; };
int e = sizeof(X);                  // 構造体Xのサイズ(TaC版で 4)
int[] f = malloc(sizeof(int)*3);    // 大きさ3のint配列を割当
X[] g = malloc(sizeof(void[])*3);   // 大きさ3の参照配列を準備
g[0] = malloc(sizeof(X));           // 構造体インスタンスを割当
g[1] = malloc(sizeof(X));
g[2] = malloc(sizeof(X));
\end{verbatim}
\end{mylist}

\subsection{addrof 演算}
\label{chap3:addrof}

関数や大域変数のアドレスを知るための演算子です。
「\verb/addrof(大域名)/」のように使用し整数型の値を返します。
\verb/interrupt/関数を割込みベクタに登録したりする目的で使用します。
配列や構造体の要素や、関数のローカル変数のアドレスを求めることはできません。

\subsection{ord 演算}
\label{chap3:ord}

char型、boolean型の値をint型に変換します。
char型の場合は文字の ASCII コード、
boolean型の場合は\verb/true=1/、\verb/false=0/となります。

\subsection{chr 演算}
\label{chap3:chr}

int型の ASCII コードから、char型の文字に変換します。

\subsection{bool 演算}
\label{chap3:bool}

int型の\verb/1/、\verb/0/から、boolean型の論理値に変換します。

以下に、ord、chr、bool 演算子の使用例を示します。

\begin{mylist}
\begin{verbatim}
// ord(), chr(), bool()演算子の使用例
int i = 0x41;        // 'A'のASCIIコード
char c = chr(i);     // cに、文字'A'が代入される
c = chr(ord(c)+1);   // cに、文字'B'が代入される
i = ord(true);       // i は 1 になる
boolean b = bool(1); // b は true になる
\end{verbatim}
\end{mylist}

\subsection{カンマ演算}
複数の式を接続して文法上一つの式にします。
例えば、
式が一つしか書けない for 文の再初期化部分に二つの式を書くために使用できます。
カンマ演算子は、最も優先順位の低い演算子です。

\begin{mylist}
\begin{verbatim}
// カンマ式を使用した例
for (i=0;i<0;i=i+1,j=j-1) { ... } // 再初期化に二つの式を書いた
\end{verbatim}
\end{mylist}

\subsection{演算子のまとめ}

関数の呼び出しも厳密には「\verb/(引数リスト)/」演算子と考えることができますが、
関数の呼び出しかたは\pageref{chap3:func}ページで説明したので省略します。
その他に、演算の順序を明確にするための「\verb/(式)/」も含めて、
演算子の優先順位を\tabref{chap3:operator}にまとめます。
表の上の方に書いてある演算子が優先順位の高い演算子です。
同じ高さにある演算子同士は同じ優先順位になります。
計算は優先順位の高いものから順に行われます.
例えば,「\verb/*/」は「\verb/+/」よりも優先順位が高いので先に計算されます.
優先順位が同じ場合は結合規則の欄で示した順に計算されます。

\begin{mytable}{tbhp}{演算子の優先順位}{chap3:operator}
\begin{tabular}{l|l}
\hline
\multicolumn{1}{c}{演算子} & \multicolumn{1}{|c}{結合規則} \\
\hline
\verb/sizeof()/、\verb/addrof()/、\verb/ord()/、\verb/chr()/、\verb/bool()/、
関数呼出、\verb/()/、\verb/[]/、\verb/./ & 左から右 \\
\verb/+/(単項演算子)、\verb/-/(単項演算子)、\verb/!/、\verb/~/ & 右から左 \\
\verb/*/、\verb!/!、\verb/%/             & 左から右 \\
\verb/+/、\verb/-/                       & 左から右 \\
\verb/<</、\verb/>>/                     & 左から右 \\
\verb/>/、\verb/>=/、\verb/</、\verb/<=/ & 左から右 \\
\verb/==/、\verb/!=/                     & 左から右 \\
\verb/&/                                 & 左から右 \\
\verb/^/                                 & 左から右 \\
\verb/|/                                 & 左から右 \\
\verb/&&/                                & 左から右 \\
\verb/||/                                & 左から右 \\
\verb/=/                                 & 右から左 \\
\verb/,/                                 & 左から右 \\
\end{tabular}
\end{mytable}

\section{文}

関数の内部に記述されるもので、変数宣言以外の記述を文と呼びます。
文は機械語に変換されて実行されます。
文には、空文、式文、ブロック、制御文(if 文や for 文)等があります。
\cmml の制御文には switch 文がありませんが
\cl にある他の制御文は一通り揃っています。

\subsection{空文}

単独の「\verb/;/」を空文と呼び、
文法上、一つの文として扱います。
本文のない for 文等で形式的な本文として使用します。
次に、空文を用いる例を示します。

\begin{mylist}
\begin{verbatim}
for (i=2; i<n; i=i*i)  // 必要なことはこの1行で全部記述できた
  ;                    // 空文
\end{verbatim}
\end{mylist}

\subsection{式文}

式の後ろに「\verb/;/」を付けたものを式文と呼び、
文法上、一つの文として扱います。
\cmml の文法に代入文はありませんが、
代入式に「\verb/;/」を付けた「式文」が同じ役割に使用されます。

\begin{quote}
\begin{verbatim}
式 ;
\end{verbatim}
\end{quote}

\subsection{ブロック}

「\verb/{/」と「 \verb/}/」で括って複数の文をグループ化し、
文法上、一つの文にします。
if 文や while 文の「本文」は、文法的には一つの文でなければなりません。
複数の文を「本文」として実行させたい場合はブロックにします。
また、ブロックはローカル変数の有効範囲を決定します。
ブロック内部で宣言された変数の有効範囲はブロックの最後までです。

\begin{quote}
\begin{verbatim}
{ 文 または 変数宣言 ... }
\end{verbatim}
\end{quote}

\subsection{if 文}

条件によって実行の流れを変更するための文です。
「条件式」は論理型の値を返す式でなければなりません。
「条件式」の値が true の場合「本文1」が実行され、
false の場合「本文2」が実行されます。
なお、else 節(「else 本文2」の部分)は省略することができます。

\begin{quote}
\begin{verbatim}
if ( 条件式 ) 本文1 【 else 本文2 】 
\end{verbatim}
\end{quote}

\subsection{while 文}

条件が成立している間、while 文の「本文」を実行します。
「条件式」は論理型の値を返す式でなければなりません。
まず、「条件式」を計算し、値が true なら「本文」が実行されます。
これは、「条件式」の値が false になるまで繰り返されます。

\begin{quote}
\begin{verbatim}
while ( 条件式 ) 本文
\end{verbatim}
\end{quote}

\subsection{do-while 文}

条件が成立している間、do-while 文の「本文」を実行します。
「条件式」は論理型の値を返す式でなければなりません。
まず「本文」を実行し、次に「条件式」を計算します。
「条件式」の値が true なら、再度、「本文」の実行に戻ります。
これは、「条件式」の値が false になるまで繰り返されます。

\begin{quote}
\begin{verbatim}
do 本文 while ( 条件式 ) ;
\end{verbatim}
\end{quote}

\subsection{for 文}

便利に拡張された while 文です。
「条件式」は論理型でなければなりません。
まず、「初期化式」か「ローカル変数宣言」を実行します。
次に「条件式」を計算し、値が true なら「本文」を実行します。
最後に「再初期化式」を実行し、その後「条件式」の計算に戻ります。
これは、「条件式」の値が false になるまで繰り返されます。

「ローカル変数宣言」で宣言された変数は、「条件式」、「再初期化式」、
「本文」で使用することができますが、それ以降では使用できません。
「初期化式」、「条件式」、「再初期化式」のどれも省略可能です。
「条件式」を省略した場合は無限ループの記述になります。

\begin{quote}
\begin{verbatim}
for(【初期化式|ローカル変数宣言】;【条件式】;【再初期化式】) 本文
\end{verbatim}
\end{quote}

\begin{mylist}
\begin{verbatim}
// for 文の使用例
int i;
for (i=0; i<10; i=i+1) { ... }
n = i;                              // i をここで使用できる

for (int j=0; j<10; j=j+1) { ... }
n = j;                              // (エラー)j が未定義になる

for (;;) { ... }                    // 無限に本文を繰り返す
\end{verbatim}
\end{mylist}

\subsection{return 文}

関数から戻るときに使用します。
関数の途中で使用すると、関数の途中から呼び出し側に戻ることができます。
void 型以外の関数では、関数の最後の文が return 文でなければなりません。
「式」は void 型の関数では書いてはなりません。
逆に、void 型以外の関数では書かなければなりません。
「式」の型と関数の型は一致していなければなりません。
なお、interrupt 関数には void 型の関数と同じルールが適用されます。

\begin{quote}
\begin{verbatim}
return 【 式 】 ;
\end{verbatim}
\end{quote}

\begin{mylist}
\begin{verbatim}
// void 型以外の関数
int f() {
  ...
  if (err) return 1;     // f の途中から戻る
  ...
  return 0;              // この return は省略できない
}
// void 型の関数
void g() {
  ...
  if (err) return;       // g の途中から戻る
  ...
  return;                // この return は省略しても良い
}
\end{verbatim}
\end{mylist}

\subsection{break 文}

for 文や while 文、do-while 文の繰り返しから脱出します。
多重ループから一度に脱出することはできません。

\begin{quote}
\begin{verbatim}
break;
\end{verbatim}
\end{quote}

\subsection{continue 文}

for 文や while 文、do-while 文の本文の実行をスキップします。
for 文では再初期化式に、while 文と do-while 文では条件式にジャンプします。

\begin{quote}
\begin{verbatim}
continue;
\end{verbatim}
\end{quote}
 % C-- 言語の仕様
% 
%  4章 ライブラリ関数
%
\chapter{ライブラリ関数}

\cmml で使用できる関数です。
必要最低限の関数が、\tac 版、\cl トランスレータ版で使用できます。

\section{標準入出力ライブラリ}

\verb/#include <stdio.hmm>/を書いた後で使用します。
トランスレータ版では\cl の高水準I/O関数の呼出しに変換されます。
\tac 版でも入出力の自動的なバッファリングを行います。
\tac 版ではバッファサイズは 128 バイトです。
以下の関数が使用できます。

\subsection{printf 関数}
標準出力ストリームに\verb/format/文字列を用いた変換付きで出力します。
出力した文字数を関数の値として返します。
トランスレータ版では\cl の\verb/printf/関数の呼出しに変換されます。
\tac 版では簡易\verb/printf/関数を使用できます。

\tac 版の簡易\verb/printf/関数では、
\verb/format/文字列に以下のような変換を記述できます。

\begin{quote}
\begin{verbatim}
%[-][数値]変換文字
\end{verbatim}
\end{quote}

\verb/-/を書くと左詰めで表示します。
数値は表示に使用するカラム数を表します。
数値を\verb/0/で開始した場合は、
数値の右づめ表示で空白の代わりに\verb/0/が用いられます。
使用できる変換文字は次の表の通りです。

\begin{quote}
\begin{tabular}{c|l}
\multicolumn{1}{c|}{変換文字} & \multicolumn{1}{c}{意味} \\\hline
\verb/o/ & 整数値を8進数で表示する \\
\verb/d/ & 整数値を10進数で表示する \\
\verb/x/ & 整数値を16進数で表示する \\
\verb/c/ & ASCIIコードに対応する文字を表示する \\
\verb/s/ & 文字列を表示する \\
\verb/%/ & \verb/%/を表示する \\
\end{tabular}
\end{quote}

\subsection{puts関数}

標準出力ストリームへ1行出力します。
エラーが発生した場合は\verb/null/を、正常時には\verb/s/を返します。

\begin{quote}
\begin{verbatim}
public char[] puts(char[] s);
\end{verbatim}
\end{quote}

\subsection{putchar関数}

標準出力ストリームへ1文字出力します。
エラーが発生した場合は\verb/true/を、正常時には\verb/false/を返します。

\begin{quote}
\begin{verbatim}
public boolean putchar(char c);
\end{verbatim}
\end{quote}

\subsection{getchar関数}

標準入力ストリームから1文字入力します。
\cl の\verb/getchar/関数と異なり\verb/char/型なので EOF チェックができません。
現在のところ、\tacos では標準入力を EOF にする方法は準備されていません。

\begin{quote}
\begin{verbatim}
public char getchar();
\end{verbatim}
\end{quote}

\subsection{fopen関数}

ファイルを開きます。
\verb/path/引数はファイルへのパス、
\verb/mode/引数はオープンのモードです。
パスは``\verb;/;''区切りで表現します。
\tacos にはカレントディレクトリはありません。
\verb/fopen/は正常に\verb/FILE/構造体、
エラー時に\verb/null/を返します。

\begin{quote}
\begin{verbatim}
public FILE fopen(char[] path, char[] mode);
\end{verbatim}
\end{quote}

\verb/mode/の意味は次の通りです。

\begin{quote}
\begin{tabular}{c|l}
\multicolumn{1}{c|}{mode} & \multicolumn{1}{c}{意味} \\\hline
\verb/r/ & 読み込みモードで開く \\
\verb/w/ & 書き込みモードで開く(ファイルが無ければ作る) \\
\verb/a/ & 追記モードで開く(ファイルが無ければ作る)
\end{tabular}
\end{quote}

\subsection{fclose関数}

ストリームをクローズします。
\tacos では、
標準入出力ストリーム(\verb/stdin/、\verb/stdout/、\verb/stderr/)を
クローズすることはできません。
\verb/fclose/は正常に\verb/false/、
エラー時に\verb/true/を返します。

\begin{quote}
\begin{verbatim}
public boolean fclose(FILE stream);
\end{verbatim}
\end{quote}

\subsection{fprintf関数}

出力ストリームを明示できる\verb/printf/関数です。
\verb/stream/引数に出力先を指定します。
出力ストリームは、\verb/fopen/で開いたファイルか\verb/stdout/です。

\begin{quote}
\begin{verbatim}
public int fprintf(FILE stream, char[] format, ...);
\end{verbatim}
\end{quote}

\subsection{fputs関数}

出力ストリームを明示できる\verb/puts/関数です。
\verb/stream/引数に出力先を指定します。
出力ストリームは、\verb/fopen/で開いたファイルか\verb/stdout/です。

\begin{quote}
\begin{verbatim}
public char[] fputs(char[] s, FILE stream);
\end{verbatim}
\end{quote}

\subsection{fputc関数}

出力ストリームを明示できる\verb/putchar/関数です。
\verb/stream/引数に出力先を指定します。
出力ストリームは、\verb/fopen/で開いたファイルか\verb/stdout/です。

\begin{quote}
\begin{verbatim}
public boolean fputc(char c);
\end{verbatim}
\end{quote}

\subsection{fgets関数}

任意の入力ストリームから1行入力します。
入力は\verb/buf/に文字列として格納します。
\verb/n/引数には\verb/buf/のサイズを渡します。
通常、\verb/buf/には\verb/\n/も格納されます。
\verb/fgets/は、EOFで\verb/null/を、
正常時には\verb/buf/を返します。

\begin{quote}
\begin{verbatim}
public char[] fgets(char[] buf, int n, FILE stream);
\end{verbatim}
\end{quote}

\cmm では、\cl の\verb/gets/が使用できません。
\verb/gets/はバッファオーバーフローの危険があるので\cmm には持込みませんでした。
\cmm で、\verb/gets/を使用したい時は\verb/fgets/を使用して次のように書きます。

\begin{quote}
\begin{verbatim}
while (fgets(buf, N, stdin)!=null) { ...
\end{verbatim}
\end{quote}


\subsection{fgetc関数}

任意の入力ストリームから1文字入力します。
\cl の\verb/fgetc/関数と異なり\verb/char/型なので EOF チェックができません。
\tac 版では安全のため、
\verb/fgetc/関数がEOFに出会うと強制的にプログラムを終了する仕様になっています。
\verb/fgetc/関数を実行する前に、必ず、
\verb/feof/関数を用いて EOF チェックをする必要があります。

\begin{quote}
\begin{verbatim}
public char fgetc(FILE stream);
\end{verbatim}
\end{quote}

\subsection{feof関数}

入力ストリームが EOF になっていると \verb/true/ を返します。
\verb/fgetc/ を実行する前に EOF チェックのために使用します。

\begin{quote}
\begin{verbatim}
public boolean feof(FILE stream);
\end{verbatim}
\end{quote}

\subsection{ferror関数}

ストリームがエラーを起こしていると \verb/true/ を返します。

\begin{quote}
\begin{verbatim}
public boolean ferror(FILE stream);
\end{verbatim}
\end{quote}

\subsection{fflush関数}

出力ストリームのバッファをフラッシュします。
入力ストリームをフラッシュすることはできません。
正常時\verb/false/、エラー時\verb/true/ を返します。
\verb/stderr/はバッファリングされていないので、
フラッシュしても何も起きません。

\begin{quote}
\begin{verbatim}
public boolean fflush(FILE stream);
\end{verbatim}
\end{quote}

\subsection{perror関数}

現在の\verb/errno/グローバル変数の値に応じたエラーメッセージを表示します。
\verb/msg/はエラーメッセージの先頭に付け加えます。
\verb/errno/はシステムコールやライブラリ関数がセットします。
\tabref{chap4:err}に\tac 版のエラーとメッセージの一覧を示します。

\begin{quote}
\begin{verbatim}
public void perror(char[] msg);
\end{verbatim}
\end{quote}

\subsection{プログラム例}

\cmml で記述した、標準入出力関数の使用例を以下に示します。

\subsubsection{TacOS 専用版のプログラム例}

\tacos では、\verb/errno/変数にセットされるエラー番号が負の値になっています。
また、アプリケーションが負の終了コードで終わった場合、
シェルがエラーメッセージを表示する仕様になっています。

更にライブラリは、ユーザプログラムのバグが原因と考えられるエラーや、
メモリ不足のような対処が難しいエラーが発生したとき、
負の終了コードでプログラムを終了します。
そこで、次のようなエラー処理を簡略化したプログラムを書くことができます。

%このプログラムでは、
%メモリ不足で\verb/FILE/構造体の割り付けができない場合は、
%\verb/fopen/内部でエラー終了しシェルがエラーメッセージを表示します。

\begin{mylist}
\begin{verbatim}
// ファイルの内容を表示するプログラム
// (TacOS 専用バージョン)
#include <stdio.hmm>
#include <errno.hmm>
public int main(int argc, char[][] argv) {
  FILE fp = fopen("a.txt", "r");
  if (fp==null) exit(errno);     // エラー表示をシェルに任せる
  while (!feof(fp))
    putchar(fgetc(fp));
  fclose(fp);
  return 0;
}
\end{verbatim}
\end{mylist}

\subsubsection{TacOS トランスレータ共通版のプログラム例}

\cl プログラム風に記述することもできます。
この例では、\verb/fopen/がエラーになったファイルの名前をエラーメッセージに
含めることができます。

\begin{mylist}
\begin{verbatim}
// ファイルの内容を表示するプログラム
// (トランスレータ、TacOS 共通バージョン)
#include <stdio.hmm>
public int main(int argc, char[][] argv) {
  char fname = "a.txt";
  FILE fp = fopen(fname, "r");
  if (fp==null) {
    perror(fname);     // エラー表示を自分で行う
    return 1;
  }
  while (!feof(fp))
    putchar(fgetc(fp));
  fclose(fp);
  return 0;
}
\end{verbatim}
\end{mylist}

\begin{mytable}{tbhp}{エラー一覧}{chap4:err}
\begin{tabular}{l|l|l}
\multicolumn{1}{c|}{記号名}
 & \multicolumn{1}{c|}{メッセージ}
 & \multicolumn{1}{c}{意味} \\\hline
ENAME     & Invalid file name           & ファイル名が不正 \\
ENOENT    & No such file or directrory  & ファイルが存在しない \\
EEXIST    & File exists                 & 同名ファイルが存在する \\
EOPEND    & File is opened              & 既にオープンされている \\
ENFILE    & File table overflow         & システムのオープン超過 \\
EBADF     & Bad file number             & ファイル記述子が不正 \\
ENOSPC    & No space left on device     & デバイスに空き領域が不足 \\
EPATH     & Bad path                    & パスが不正 \\
EMODE     & Bad mode                    & モードが一致しない \\
EFATTR    & Bad attribute               & ファイルの属性が不正 \\
ENOTEMP   & Directory is not empty      & ディレクトリが空でない \\
EINVAL    & Invalid argument            & 引数が不正 \\
EMPROC    & Process table overflow      & プロセスが多すぎる \\
ENOEXEC   & Bad EXE file                & EXE ファイルが不正 \\
EMAGIC    & Bad MAGIC number            & 不正なマジック番号 \\
EMFILE    & Too many open files         & プロセス毎のオープン超過 \\
ECHILD    & No children                 & 子プロセスが存在しない \\
ENOZOMBI  & No zombie children          & ゾンビの子が存在しない \\
ENOMEM    & Not enough memory           & 十分な空き領域が無い \\
ESYSNUM   & Invalid system call number  & システムコール番号が不正 \\
EZERODIV  & Zero division               & ゼロ割り算 \\
EPRIVVIO  & Privilege violation         & 特権違反 \\
EILLINST  & Illegal instruction         & 不正命令 \\
EUSTK     & Stack overflow              & スタックオーバーフロー \\
EUMODE    & stdio: Bad open mode        & モードと使用方法が矛盾 \\
EUBADF    & stdio: Bad file pointer     & 不正な fp が使用された \\
EUEOF     & fgetc: EOF was ignored      & fgetc前にEOFチェック必要 \\
EUNFILE   & fopen: Too many open files  & プロセス毎のオープン超過 \\
EUSTDIO   & fclose: Standard i/o should & 標準ioはクローズできない \\
          &  not be closed              &                          \\
EUFMT     & fprintf: Invalid conversion & 書式文字列に不正な変換 \\
EUNOMEM   & malloc: Insufficient memory & ヒープ領域が不足 \\
EUBADA    & free: Bad address           & mallocした領域ではない \\
\end{tabular}
\end{mytable}

\section{標準関数}

\verb/#include <stdlib.hmm>/を書いた後で使用します。

\subsection{malloc関数}

ヒープ領域に引数の整数で指定バイト数のメモリ領域を確保し、
領域を指す参照を返します。
\verb/malloc/関数は\verb/void[]/型なので、
領域を指す参照は全ての参照変数に代入できます。

\tac 版では、ヒープ領域に十分な空きが見つからないとき
終了コード\verb/EUNOMEM/でプログラムを終了します。
トランスレータ版では、エラーメッセージを表示したあと
終了コード\verb/1/でプログラムを終了します。

\begin{quote}
\begin{verbatim}
public void[] malloc(int size);
\end{verbatim}
\end{quote}

\subsection{free関数}

\verb/malloc/関数で割当てた領域を解放します。
\tac 版では、領域が\verb/malloc/関数で割当てたものではない可能性がある場合
(マジックナンバーが破壊されている、管理されている空き領域と重なる等)、
終了コード\verb/EUBADA/でプログラムを終了します。

\begin{quote}
\begin{verbatim}
public void free(void[] mem);
\end{verbatim}
\end{quote}

\subsection{atoi関数}

\verb/atoi/関数は引数に渡した10進数文字列を解析して、
それが表現する値を返します。

\begin{quote}
\begin{verbatim}
public int atoi(char[] s);
\end{verbatim}
\end{quote}

\subsection{srand関数}

\verb/srand/関数は擬似乱数発生器を\verb/seed/で初期化します。

\begin{quote}
\begin{verbatim}
public void srand(int seed);
\end{verbatim}
\end{quote}

\subsection{rand関数}

\verb/rand/関数は次の擬似乱数を発生します。

\begin{quote}
\begin{verbatim}
public int rand();
\end{verbatim}
\end{quote}

\subsection{exit関数}

\verb/exit/関数はオープン済みのストリームをフラッシュしてから
プログラムを終了します。

\verb/status/は、親プロセスに返す終了コードです。
\verb/0/が正常終了の意味、\verb/1/以上はユーザが決めた終了コード、
負の値は\tabref{chap4:err}に示す記号名で定義されています。
負の値を返すと親プロセスがシェルの場合、
シェル側でエラーメッセージを表示してくれます。

\begin{quote}
\begin{verbatim}
public void exit(int status);
\end{verbatim}
\end{quote}

\section{文字列操作関数}

\verb/#include <string.hmm>/を書いた後で使用します。

\subsection{strCpy関数}

\verb/s/文字列を文字配列\verb/d/にコピーし、
\verb/d/を関数値として返します。
トランスレータ版では\cl の\verb/strcpy/関数に置き換えられます。

\begin{quote}
\begin{verbatim}
public char[] strCpy(char[] d, char[] s);
\end{verbatim}
\end{quote}

\subsection{strNcpy関数}

文字列\verb/s/の最大\verb/n/文字を文字配列\verb/d/にコピーし、
\verb/d/を関数値として返します。
文字配列の使用されない部分には\verb/'\0'/が書き込まれます。
文字列の長さが\verb/n-1/より長い場合は、
\verb/'\0'/が書き込まれないので注意して下さい。
トランスレータ版では\cl の\verb/strncpy/関数に置き換えられます。

\begin{quote}
\begin{verbatim}
public char[] strNcpy(char[] d, char[] s, int n);
\end{verbatim}
\end{quote}

\subsection{strCat関数}

文字列\verb/s/を文字配列\verb/d/に格納されている文字列の後ろに追加し、
\verb/d/を関数値として返します。
トランスレータ版では\cl の\verb/strcat/関数に置き換えられます。

\begin{quote}
\begin{verbatim}
public char[] strCat(char[] d, char[] s);
\end{verbatim}
\end{quote}

\subsection{strNcat関数}

文字列\verb/s/の先頭\verb/n/文字未満を、
文字配列\verb/d/に格納されている文字列の後ろに追加し、
\verb/d/を関数値として返します。
\verb/d/に格納された文字列の最後には\verb/'\0'/が書き込まれます。
トランスレータ版では\cl の\verb/stnrcat/関数に置き換えられます。

\begin{quote}
\begin{verbatim}
public char[] strNcat(char[] d, char[] s, int n);
\end{verbatim}
\end{quote}

\subsection{strCmp関数}

文字列\verb/s1/と文字列\verb/s2/を比較します。
\verb/strCmp/関数は、アスキーコード順で\verb/s1/が大きいとき正の値、
\verb/s1/が小さいとき負の値、同じ時\verb/0/を返します。
トランスレータ版では\cl の\verb/strcmp/関数に置き換えられます。

\begin{quote}
\begin{verbatim}
public int strCmp(char[] d, char[] s);
\end{verbatim}
\end{quote}

\subsection{strNcmp関数}

文字列\verb/s1/と文字列\verb/s2/の先頭\verb/n/文字を比較します。
\verb/strNcmp/関数は、
\verb/strcmp/関数同様にアスキーコード順で大小を判断します。
トランスレータ版では\cl の\verb/strncmp/関数に置き換えられます。

\begin{quote}
\begin{verbatim}
public int strNcmp(char[] d, char[] s, int n);
\end{verbatim}
\end{quote}

\subsection{strLen関数}

文字列\verb/s/の長さを返します。
長さに\verb/'\0'/は含まれません。
トランスレータ版では\cl の\verb/strlen/関数に置き換えられます。

\begin{quote}
\begin{verbatim}
public int strLen(char[] s);
\end{verbatim}
\end{quote}

\subsection{strChr関数}

文字列\verb/s/の中で最初に文字\verb/c/が現れる位置を、
\verb/s/文字配列の添字で返します。
文字\verb/c/が含まれていない場合は\verb/-1/を返します。
トランスレータ版では\verb;lib/wrapper.c;の関数に置き換えられます。

\begin{quote}
\begin{verbatim}
public int strChr(char[] s, char c);
\end{verbatim}
\end{quote}

\subsection{strRchr関数}

文字列\verb/s/の中で最後に文字\verb/c/が現れる位置を、
\verb/s/文字配列の添字で返します。
文字\verb/c/が含まれていない場合は\verb/-1/を返します。
トランスレータ版では\verb;lib/wrapper.c;の関数に置き換えられます。

\begin{quote}
\begin{verbatim}
public int strRchr(char[] s, char c);
\end{verbatim}
\end{quote}

\subsection{strStr関数}

文字列\verb/s1/の中に文字列\verb/s2/が現れる位置を、
\verb/s1/文字配列の添字で返します。
文字列\verb/s2/が含まれていない場合は\verb/-1/を返します。
トランスレータ版では\verb;lib/wrapper.c;の関数に置き換えられます。

\begin{quote}
\begin{verbatim}
public int strStr(char[] s1, char[] s2);
\end{verbatim}
\end{quote}

\subsection{subStr関数}

文字列\verb/s/の先頭\verb/pos/文字を省いた部分文字列を返します。
返した文字列は\verb/s/とメモリ領域を共用していますので注意が必要です。
トランスレータ版では\verb;lib/wrapper.c;の関数に置き換えられます。

\begin{quote}
\begin{verbatim}
public char[] subStr(char[] s, int pos);
\end{verbatim}
\end{quote}

\section{参照型(アドレス)操作関数}

\cmml にはキャスティング演算や、ポインタ演算がありません。
そこで、以下の関数で代用します。

\subsection{\ul \ul ItoA 関数}
\label{chap4:itoa}

整数から参照へ型を変換する関数です。
16ビット整数を引数に \verb/void[]/ 参照(16ビットのアドレス)を返します。
\ul \ul ItoA 関数のプロトタイプ宣言を次に示します。
関数の値は \verb/void[]/ 型の参照なので、
どのような参照型変数にも代入できます。

\begin{quote}
\begin{verbatim}
public void[] __ItoA(int a);
\end{verbatim}
\end{quote}

\ul \ul ItoA 関数の応用例を次に示します。
\pageref{app:memmap}ページに示すように
F000H 番地は \tac のビデオメモリの開始番地です。
このプログラムは、
ビデオメモリを配列のようにアクセスし画面を文字 'A' で埋めつくします。

\begin{mylist}
\begin{verbatim}
// ビデオ RAM を配列としてアクセスする
char[] vRam = __ItoA(0xf000);
for (int i=0; i<80*24; i=i+1)
  vRam[i] = 'A';
\end{verbatim}
\end{mylist}

\subsection{\ul \ul AtoI 関数}

参照から整数へ型を変換する関数です。
参照(16ビットアドレス)を引数に16ビットの整数を返します。
\ul \ul AtoI 関数のプロトタイプ宣言を次に示します。
引数の型は \verb/void[]/ なので、
参照型ならどんな型でも渡すことができます。

\begin{quote}
\begin{verbatim}
public int __AtoI(void[] a);
\end{verbatim}
\end{quote}

\ul \ul AtoI 関数の応用例を次に示します。
このプログラムは、\tac の IPL ルーチン(実物)の一部です。
まず、指定セクタから読んだブートプログラムをメモリに格納します。
次に、このプログラムの先頭アドレスにジャンプします。

\begin{mylist}
\begin{verbatim}
// IPL の一部(ブートプログラムを読み込む)
  char[] hBuf = malloc(256);          // マイクロドライブ用バッファ
  int    sct = 23;                    // 23セクタから格納されている
  readSct(0,sct,hBuf);                // 1セクタ読み出す
  char[] mem = __ItoA(bswap(hBuf[0]));// 先頭ワードがロードアドレス
  int    len = bswap(hBuf[1]);        // 第2ワードがプログラム長
  int    idx = 2;                     // bswapは上/下位バイトを交換
  int    offs = 0;
  while(true) {
     while (idx<256) {                // 1セクタ256ワードに付き
       mem[offs] = bswap(hBuf[idx]);  // 機械語をメモリにコピー
       offs = offs + 1;
       idx  = idx  + 1;
     }
     if (offs>=len) break;            // プログラム長を超えたら終了
     sct = sct + 1;                   // 次のセクタを読む
     readSct(0,sct,hBuf);
     idx = 0;
  }
  putstr("\n\nStart boot loader@0x");
  puthex(__AtoI(mem));
  putstr(" ...\n\n");
  _jump(__AtoI(mem));                 // ロードしたプログラムに飛ぶ
\end{verbatim}
\end{mylist}

\subsection{\ul \ul AddrAdd 関数}

\cl のポインタ演算の代用にする関数です。
参照(アドレス)と整数を引数に渡し、
参照から整数ワード先の参照(アドレス)を返します。
この関数は \verb/malloc/、\verb/free/ 関数の実現に使用されています。
\ul \ul AddrAdd 関数のプロトタイプ宣言を次に示します。

\begin{quote}
\begin{verbatim}
public void[] __AddrAdd(void[] a, int n);
\end{verbatim}
\end{quote}

\subsection{\ul \ul acmp 関数}

参照(アドレス)の大小比較を行う関数です。
\cmml では参照の大小比較はできません。
\javal でも参照の大小比較はできないので、通常はこの仕様で十分と考えられます。
しかし、\verb/malloc/、\verb/free/ 関数等の実現には
アドレスの大小比較が必要です。
そこで、アドレスの大小比較をする \ul \ul acmp 関数を用意しました。
\ul \ul acmp 関数のプロトタイプ宣言は下のようになります。
\ul \ul acmp 関数は、\verb/a/ の方が大きい場合は 1 を、
\verb/b/ の方が大きい場合は -1 を、
\verb/a/ と \verb/b/ が等しい場合は 0 を返します。

\begin{quote}
\begin{verbatim}
public int __acmp(void[] a, void[] b);
\end{verbatim}
\end{quote}

\section{符号無し整数の操作関数}

\cmml では、int 型も char 型も符号付き整数です。
内部表現に2の補数表現を用いているので、
ほとんどの操作は、
符号付きと考えても符号無しと考えても同じ機械語命令列で処理できます。
そのため、符号付き数を符号無し数の代用に使用することができます。
しかし、大小比較は符号付きと符号無しで異なる機械語を必要とします。
そこで、この部分だけ関数を使用した特別な比較方法を使います。

\subsection{\ul \ul ucmp 関数}

符号無し数の比較を行う関数です。
\ul \ul ucmp 関数のプロトタイプ宣言を次に示します。
\ul \ul ucmp 関数は、\verb/a/ の方が大きい場合は 1 を、
\verb/b/ の方が大きい場合は -1 を、
\verb/a/ と \verb/b/ が等しい場合は 0 を返します。

\begin{quote}
\begin{verbatim}
public int __ucmp(int a, int b);
\end{verbatim}
\end{quote}

次に \ul \ul ucmp 関数の応用例を示します。
二つの符号無し数の大小関係によって3通りに場合分けしています。

\begin{mylist}
\begin{verbatim}
  int a = 0xffff;
  int b = 0x0000;
  if (__ucmp(a,b)>0) {
    ...                          // a > b の場合
  } else if (__ucmp(a,b)==0) {
    ...                          // a = b の場合
  } else if (__ucmp(a,b)<0) {
    ...                          // a < b の場合
  }
\end{verbatim}
\end{mylist}

\section{可変個引数関数のサポート関数}

\cmml は引数の数が一定ではない可変個引数関数の宣言を認めますが、
当該関数を記述するために必要な可変個引数へのアクセス方法を提供しません。
そこで、サポート関数の力を借りて引数へアクセスする手段を提供します。

\subsection{\ul \ul args 関数}
\label{chap4:args}

printf 関数のような可変個引数の関数を実現するために、
可変個引数関数の内部で引数を配列としてアクセスできるようにする関数です。
\ul \ul args 関数は \ul \ul args を呼び出した
\cmm 関数の第2引数を添字 0 とする配列を返します。
\ul \ul args 関数のプロトタイプ宣言を次に示します。

\begin{quote}
\begin{verbatim}
public int[] __args();
\end{verbatim}
\end{quote}

次に可変個引数関数の記述例として、
printf 関数を記述したプログラムの一部を示します。

\begin{mylist}
\begin{verbatim}
int printf(char[] fmt, ...) {      // ... は可変個引数の関数を表す
  int[] args = __args();           // args配列は第2引数以降を格納
  int n = 0;                       // 引数用のカウンタ
  int arg = args[n];               // 第2引数から順に取り出す
  for (i=0; fmt[i]!='\0'; i=i+1) { // 第1引数fmtは普通に使用できる
    char c = fmt[i];               // 書式文字列から1文字取り出す
    if (c=='%') {                  // 書式文字列に % 発見
      c = fmt[i=i+1];              // % に続く文字を取り出す
      ...
      } else if (c=='c') {         // %c なら
        putch(arg);                // 1文字出力
        arg = args[n=n+1];         // arg を更新
      } else if (c=='s') {         // %s なら
        putstr(__ItoA(arg));       // 引数は文字配列参照のはず
        arg = args[n=n+1];         // arg を更新
      } else ...
    ...
\end{verbatim}
\end{mylist}


\section{32ビット演算関数}

\cmml には32ビット整数型がありません。
32ビットの四則演算をするための関数を用意しました。
これらの関数では32ビットデータを要素数2の int 型配列で表現します。

\subsection{\ul add32 関数}

32ビットの加算ルーチンです。
dst、src 二つの配列を渡し結果を dst 配列に格納します。
また、関数値として dst 配列への参照を返します。
つぎに、\ul add32 関数のプロトタイプ宣言を示します。

\begin{quote}
\begin{verbatim}
public int[] _add32(int[] dst, int[] src);
\end{verbatim}
\end{quote}

\ul add32 関数の使用例を次に示します。
%最後の行は、\ul add32 関数が dst 配列への参照を返すことを利用して、
%直前の2行と同じ処理をコンパクトに記述した例になっています。

\begin{mylist}
\begin{verbatim}
int[] one = { 0, 1 };               // 32ビットの 1
int[] two = { 0, 2 };               // 32ビットの 2
int[] dat = array(2);

dat[0] = dataH;                     // 32ビットデータ
dat[1] = dataL;

_add32(dat, one);                   // データに +1 する
_add32(dat, two);                   // データに +2 する

_add32(_add32(dat, one), two);      // 前の2行と同じ演算を短く記述

printf("%04x%04x\n",dat[0],dat[1]); // 計算結果表示
\end{verbatim}
\end{mylist}

\subsection{\ul sub32 関数}

32ビットの減算ルーチンです。
dst、src 二つの配列を渡し dst から src を引いた結果を dst 配列に格納します。
また、関数値として dst 配列への参照を返します。
つぎに、\ul sub32 関数のプロトタイプ宣言を示します。

\begin{quote}
\begin{verbatim}
public int[] _sub32(int[] dst, int[] src);
\end{verbatim}
\end{quote}

\subsection{\ul mul32 関数}

32ビットの乗算ルーチンです。
16ビットデータと16ビットデータの積を32ビットデータとして返します。
一方の16ビットデータを dst 配列の2番目の要素(\verb/dst[1]/)に格納して、
もう一方の16ビットデータを src として渡します。
dst 配列の1番目の要素(\verb/dst[0]/)は計算に使用しません。
計算結果は dst 配列に32ビットデータとして格納します。
また、関数値として dst 配列への参照を返します。
つぎに、\ul mul32 関数のプロトタイプ宣言を示します。

\begin{quote}
\begin{verbatim}
public int[] _mul32(int[] dst, int src);
\end{verbatim}
\end{quote}

\subsection{\ul div32 関数}

32ビットの割算ルーチンです。
32ビットデータを16ビットデータで割って、
商と余りをそれぞれ16ビットで求めます。
32ビットデータを dst 配列で、
16ビットデータを src として渡します。
計算結果は、商が \verb/dst[0]/ に、余りが \verb/dst[1]/ に、
それぞれ16ビットデータとして格納されます。
また、関数値として dst 配列への参照を返します。
つぎに、\ul div32 関数のプロトタイプ宣言を示します。

\begin{quote}
\begin{verbatim}
public int[] _div32(int[] dst, int src);
\end{verbatim}
\end{quote}

\subsection{Ld32 マクロ}

32ビットデータを初期化するためのマクロです。
\verb/util.h/ ファイル中で次のように宣言されています。

\begin{quote}
\begin{verbatim}
#define Ld32(dst,h,l) ((dst)[0]=(h),(dst)[1]=(l))
\end{verbatim}
\end{quote}

使用例を示します。

\begin{mylist}
\begin{verbatim}
int[] dat = array(2);

Ld32(dat, dataH, dataL);            // 32ビットデータ

_add32(dat, one);                   // データに +1 する
\end{verbatim}
\end{mylist}

\section{バイトアクセス関数}

Boot プログラム中でマイクロドライブの構造を解析する際に
必要になった関数です。
PC 等が書き込んだマイクロドライブ上のデータを TaC が読み込むと、
\cmml の配列に次表のような順序で格納されます。
このデータをバイト単位でアクセスできると便利です。

\begin{center}
{\bf マイクロドライブのデータを int 配列に格納した状態} \\
\begin{tabular}{| r | c | c |}
\hline
\multicolumn{1}{|c|}{添字} & \multicolumn{2}{|c|}{内 容}          \\
\hline
0 & 第2バイト & 第1バイト \\
\hline
1 & 第4バイト & 第3バイト \\
\hline
2 & 第6バイト & 第5バイト \\
\hline
      &       &       \\
\multicolumn{3}{|c|}{...}           \\
      &       &       \\
\hline
\end{tabular}
\end{center}

\subsection{byte 関数}

byte 関数は、
int 型配列 b に上記表の順にバイトデータが格納されているとみなし、
第 x バイトを取り出します。
第1バイトを取り出したいときは、x を 0 にします。
次に byte 関数のプロトタイプ宣言を示します。

\begin{quote}
\begin{verbatim}
public int byte(int[] b, int x); 
\end{verbatim}
\end{quote}

\subsection{Word マクロ}

バイト配列からバイト単位で位置を指定して、
16ビットデータを取り出すマクロです。
\verb/util.h/ ファイル中で次のように宣言されています。

\begin{quote}
\begin{verbatim}
#define Word(buf, idx) ((byte((buf),(idx)+1)<<8) | byte((buf),(idx)))
\end{verbatim}
\end{quote}
 % ライブラリ関数
% 
%  5章 システムコール
%
\chapter{システムコール}

\tacos のシステムコールを呼び出す関数です。
{\bf トランスレータ版では使用できません。}
\verb/#include <syslib.hmm>/と書いた後で使用します。

\section{プロセス関連}

\tacos では、
\verb/exec/で新しいプロセスを作ると同時に新しいプログラムを実行します。
UNIXの\verb/fork-exec/方式とは異なります。

%\tac にはベースレジスタや多重仮想記憶のような機構がないので、
%\verb/fork/システムコールが実現できませんでした。

\subsection{exec}

\verb/name/でファイル名を指定して新しいプロセスでプログラムの実行を開始します。
\verb/argv/は、開始するプログラムの\verb/main/関数の
第2引数(\verb/char[][]argv/)に渡される文字列配列です。
\verb/exec/は正常なら\verb/0/、エラー発生なら負のエラー番号を返します。

\begin{quote}
\begin{verbatim}
#include <syslib.hmm>
public int exec(char[] name, void[] argv);
\end{verbatim}
\end{quote}

\verb/argv[0]/にプログラム名、
\verb/argv[1]/に第1引数のように格納します。
最後に\verb/null/を格納します。
次に使用例を示します。

\begin{mylist}
\begin{verbatim}
char[][] args = {"prog", "param1", "param2", null};
void f() {
  exec("/bin/prog.exe", args);
}
\end{verbatim}
\end{mylist}

子プロセス側は次のようなプログラムになります。

\begin{mylist}
\begin{verbatim}
int main(int argc, char[][]argv) {
  int c = argc;       // 前のプログラムで起動されたとき 3
  char[] s = argv[1]; // 前のプログラムで起動されたとき "param1"
  return 0;
}
\end{verbatim}
\end{mylist}

\subsection{\ul exit}

\verb/_exit/はプログラム(プロセス)を終了します。
\verb/_exit/は入出力のバッファをフラッシュしません。
\verb/_exit/は緊急終了用に使用し、
普通は\verb/exit/を使用します。

\verb/status/は、親プロセスに返す終了コードです。
\verb/0/が正常終了の意味、\verb/1/以上はユーザが決めた終了コード、
負の値は\tabref{chap4:err}に示す記号名で定義されています。
負の値を返すと親プロセスがシェルの場合、
シェル側でエラーメッセージを表示してくれます。

\begin{quote}
\begin{verbatim}
#include <syslib.hmm>
public void _exit(int status);
\end{verbatim}
\end{quote}

\subsection{wait}

\verb/wait/は子プロセスの終了を待ちます。
\verb/stat/には大きさ1の\verb/int/配列を渡します。
子プロセスが終了した際、\verb/stat[0]/に終了コードが書き込まれます。
\verb/wait/は正常なら\verb/0/、エラー発生なら負のエラー番号を返します。

\begin{quote}
\begin{verbatim}
#include <syslib.hmm>
public int wait(int[] stat);
\end{verbatim}
\end{quote}

\subsection{sleep}

\verb/sleep/はプロセスを指定された時間だけ停止します。
\verb/ms/はミリ秒単位で停止時間を指定します。
\verb/ms/に負の値を指定すると\verb/EINVAL/エラーになります。
それ以外では、\verb/sleep/は\verb/0/を返します。

\begin{quote}
\begin{verbatim}
#include <syslib.hmm>
public int sleep(int ms);
\end{verbatim}
\end{quote}

\section{ファイル操作}

\tacos は、microSDカードのFAT16ファイルシステムを扱うことができます。
VFATには対応していません。

\subsection{creat}

\verb/creat/は新規ファイルを作成します。
\verb/path/は新しいファイルのパスです。
\verb/creat/は正常なら\verb/0/、エラー発生なら負のエラー番号を返します。

\begin{quote}
\begin{verbatim}
#include <syslib.hmm>
public int creat(char[] path);
\end{verbatim}
\end{quote}

\subsection{remove}

\verb/remove/はファイルを削除します。
\verb/path/は削除するファイルのパスです。
\verb/remove/は正常なら\verb/0/、エラー発生なら負のエラー番号を返します。

\begin{quote}
\begin{verbatim}
#include <syslib.hmm>
public int remove(char[] path);
\end{verbatim}
\end{quote}

\subsection{mkDir}

\verb/mkDir/は新規のディレクトリを作成します。
\verb/path/は新しいディレクトリのパスです。
\verb/mkDir/は正常なら\verb/0/、エラー発生なら負のエラー番号を返します。

\begin{quote}
\begin{verbatim}
#include <syslib.hmm>
public int mkDir(char[] path);
\end{verbatim}
\end{quote}

\subsection{rmDir}

\verb/rmDir/はディレクトリを削除します。
\verb/path/は削除するディレクトリのパスです。
\verb/rmDir/は正常なら\verb/0/、エラー発生なら負のエラー番号を返します。
削除するディレクトリが空でない場合はエラーになります。

\begin{quote}
\begin{verbatim}
#include <syslib.hmm>
public int rmDir(char[] path);
\end{verbatim}
\end{quote}

\section{ファイルの読み書き}

ファイルの読み書きには、
通常は\pageref{chap4:stdio}ページの標準入出力ライブラリ関数を用います。
以下のシステムコールは、主にライブラリ関数の内部で使用されます。

\subsection{open}

\verb/open/はファイルを開きます
\verb/path/は開くファイルのパスです。
\verb/mode/には\verb/READ/、\verb/WRITE/、\verb/APPEND/のいずれかを指定します。
\verb/open/は正常なら非負のファイル記述子、
エラー発生なら負のエラー番号を返します。
ファイルが存在しない場合は、どのモードでもエラーになります。
新規ファイルに書き込みたい場合は、
事前に\verb/creat/システムコールを用いてファイルを作成しておく必要があります。

\verb/open/はディレクトリを\verb/READ/モードで開くことができます。
ディレクトリは\pageref{chap4:readDir}ページの\verb/readDir/関数で読みます。

\begin{quote}
\begin{verbatim}
#include <syslib.hmm>
public int open(char[] path, int mode);
\end{verbatim}
\end{quote}

\subsection{close}

\verb/close/は\verb/open/で開いたファイルを閉じます。
\verb/fd/は閉じるファイルのファイル記述子です。
\verb/close/は正常なら\verb/0/、エラー発生なら負のエラー番号を返します。

\begin{quote}
\begin{verbatim}
#include <syslib.hmm>
public int close(int fd);
\end{verbatim}
\end{quote}

\subsection{read}

\verb/read/は\verb/open/を用い\verb/READ/モードで開いたファイルから
データを読みます。
\verb/fd/はファイル記述子です。
\verb/buf/はデータを読み込むバッファ、
\verb/len/はバッファサイズ(バイト単位)です。
\verb/read/は正常なら読み込んだバイト数、
エラー発生なら負のエラー番号を返します。
EOFでは\verb/0/を返します。

\begin{quote}
\begin{verbatim}
#include <syslib.hmm>
public int read(int fd, void[] buf, int len);
\end{verbatim}
\end{quote}

\subsection{write}

\verb/write/は\verb/open/を用い\verb/WRITE/、
\verb/APPEND/モードで開いたファイルからデータを読みます。
\verb/fd/はファイル記述子です。
\verb/buf/は書き込むデータが置いてあるバッファ、
\verb/len/は書き込むデータのサイズ(バイト単位)です。
\verb/write/は正常なら書き込んだバイト数、
エラー発生なら負のエラー番号を返します。

\begin{quote}
\begin{verbatim}
#include <syslib.hmm>
public int write(int fd, void[] buf, int len);
\end{verbatim}
\end{quote}

\subsection{seek}

\verb/seek/は\verb/open/を用い開いたファイルの読み書き位置を変更します。
\verb/fd/はファイル記述子です。
seek位置は、上位16bit(\verb/ptrh/)と
下位16bit(\verb/ptrl/)を組み合わせて指定します。
\verb/seek/は正常なら\verb/0/、
エラー発生なら負のエラー番号を返します。

\begin{quote}
\begin{verbatim}
#include <syslib.hmm>
public int seek(int fd, int ptrh, int ptrl);
\end{verbatim}
\end{quote}

\section{コンソール関連}

コンソール入出力には、
通常は\pageref{chap4:stdio}ページの標準入出力ライブラリ関数を用います。
以下のシステムコールは、主にライブラリ関数の内部で使用されます。
\verb/conRead/はライブラリ関数が\verb/stdin/からの読み込みをする場合に、
\verb/conWrite/はライブラリ関数が\verb/stdout/、\verb/stderr/への
書き込みをする場合にライブラリ関数内部で使用されます。

\subsection{conRead}

\verb/conRead/はキーボードから1行入力します。
読み込んだ内容は\verb/buf/で指定されるバッファに文字列として格納されます。
\verb/len/は\verb/buf/のバイト数です。
文字列の最後には\verb/'\0'/が格納されます。
\verb/'\n'/は含まれません。
\verb/buf/の大きさは入力可能な文字数より\verb/1/大きくする必要があります。
\verb/conRead/はキーボードから入力した文字数を返します。

\begin{quote}
\begin{verbatim}
#include <syslib.hmm>
public int conRead(char[] buf, int len);
\end{verbatim}
\end{quote}

\subsection{conWrite}

\verb/conWrite/は画面に文字列を出力します。
\verb/buf/には出力する文字列を格納して渡します。

\begin{quote}
\begin{verbatim}
#include <syslib.hmm>
public int conWrite(char[] buf);
\end{verbatim}
\end{quote}

\verb/conWrite/は以下の制御文字を解釈します。

\begin{quote}
\begin{tabular}{c|l}
\multicolumn{1}{c|}{制御文字} & \multicolumn{1}{c}{働き} \\\hline
\verb/'\r'/ & カーソルを現在行の左端に移動する \\
\verb/'\n'/ & カーソルを次の行の左端に移動する \\
\verb/'\t'/ & カーソルを次のTABストップに移動する \\
\verb/'\x08'/ & カーソルを右に1文字分移動する \\
\verb/'\x0c'/ & 画面をクリアしカーソルを画面右上端に移動する \\
\end{tabular}
\end{quote}
 % システムコール

\appendix
% 
% $Id$
%
\chapter{\cmml 文法まとめ}

以下に、\cmml の文法を BNF 風にまとめたものを掲載します。
仕様の最後に掲載されている ``\verb/#ディレクティブ/'' は、
プリプロセッサがソースプログラムのファイル名を
\cmm コンパイラに伝えるために出力するものです。
\cmm コンパイラは、
``\verb/#ディレクティブ/'' によって、
ソースプログラムやインクルードファイルの名前を知ることができます。

\begin{verbatim}
次ページで用いる文法表記方法
 全角記号がメタ文字、意味は次の通り
 (1) A ※   はAのゼロ回以上の繰り返し
 (2) 《...》はグループ
 (3) 【...】は省略可能
 (4) A | B は A または B
\end{verbatim}

{\small\tt
\begin{tabular}{l l}
プログラム:  & 《 構造体宣言 | だまし型宣言 | 大域変数定義 | 関数定義 |\\
              &  関数宣言 》※ \\
構造体宣言:  & \verb+struct+ 名前 \verb+{+
                《 型 名前 《 \verb+,+ 名前 》※ \verb+;+ 》※ \verb+} ;+ \\
だまし型宣言:& \verb+typedef+ 名前 \verb+;+ \\
大域変数定義:& 【修飾子】 型 大域変数並び \verb+;+ \\
大域変数並び:& 名前 【 \verb+=+ 初期化 】
                《 \verb+,+ 名前 【 \verb+=+ 初期化 】 》※ \\
修飾子:      & \verb+public+ \\
型:          & \verb+int+  | \verb+char+ | \verb+boolean+ |
                \verb+void+ | \verb+interrupt+ | 名前 | 型\verb+[]+ \\
初期化:      & 定数 | \verb+{+ 初期化 《 \verb+,+ 初期化 》※ \verb+}+ | \\
              & \verb+array(+ 数値 《 \verb+,+ 数値 》※ \verb+)+ \\
定数:        & 数値 | 文字列 | \verb+null+ | \verb+true+ |
                \verb+false+ \\
関数定義:    & 【修飾子】 型 名前\verb+(+ 引数リスト \verb+)+ ブロック \\
関数宣言:    & 【修飾子】 型 名前\verb+(+ 引数リスト \verb+) ;+ \\
引数リスト:  & 【 型 名前 《 \verb+,+ 型 名前 》※~
                【 \verb+, ...+ 】 】 | \verb+...+ \\
ブロック:    & \verb+{+ 《 局所変数定義 | 文 》※ \verb+}+ \\
局所変数定義:& 型 局所変数並び \verb+;+ \\
局所変数並び:& 名前 【 \verb+=+ 代入式 】
                《 \verb+,+ 名前 【 \verb+=+ 代入式 】 》※ \\
&\\
文:          & IF文 | WHILE文 | DO-WHILE文 | FOR文 | RETURN文 | \\
              & BREAK文 | CONTINUE文 | ブロック | 式 \verb+;+ | \verb+;+\\
IF文:        & \verb+if (+ 式 \verb+)+ 文  【 \verb+else+ 文  】 \\
WHILE文:     & \verb+while (+ 式 \verb+)+ 文 \\
DO-WHILE文:   & \verb+do+ 文 \verb+while (+ 式 \verb+) ;+ \\
FOR文:       & \verb+for(+ 【 式 | 局所変数定義 】 \verb+;+
               【 式 】 \verb+;+ 【 式 】 \verb+)+ 文 \\
RETURN文:    & \verb+return+ 【 式 】\verb+;+ \\
BREAK文:     & \verb+break;+ \\
CONTINUE文:  & \verb+continue;+ \\
&\\
式:          & 代入式 | 式 \verb+,+ 代入式 \\
代入式:      & 論理OR式 | 論理OR式 \verb+=+ 代入式 \\
論理OR式:    & 論理AND式 | 論理OR式  \verb+||+ 論理AND式 \\
論理AND式:   & OR式 | 論理AND式 \verb+&&+ OR式 \\
OR式:        & XOR式 | OR式 \verb+|+ XOR式 \\
XOR式:       & AND式 | XOR式 \verb+^+ AND式 \\
AND式:       & 等式 | AND式 \verb+&+ 等式 \\
等式:        & 比較式 | 等式 \verb+==+ 比較式 | 等式 \verb+!=+ 比較式 \\
比較式:      & シフト式 | 比較式 \verb+<+ シフト式  |
                比較式 \verb+<=+ シフト式 | \\
              & 比較式 \verb+>+ シフト式 | 比較式 \verb+>=+ シフト式 \\
シフト式:    & 和式 | シフト式 \verb+<<+ 和式 | シフト式 \verb+>>+ 和式 \\
和式:        & 積式 | 和式 \verb/+/ 積式 | 和式 \verb+-+ 積式 \\
積式:        & 単項式 | 積式 \verb+*+ 単項式 |
                積式 \verb+/+ 単項式 | 積式 \verb+%+ 単項式 \\
単項式:      & 因子 | \verb/+/ 単項式 | \verb+-+ 単項式 |
                \verb+!+ 単項式 | \verb+~+ 単項式 \\
因子:        & 名前 | 定数 | 関数呼出 | \verb+(+ 式 \verb+)+ |
                因子\verb+[+ 代入式 \verb+]+ | \\
              & 因子\verb+.+名前 | \verb+sizeof(+ 型 \verb+)+ |
                \verb+addrof(+ 名前 \verb+)+ | \verb+ord(+ 式 \verb+)+ | \\
              &  \verb+chr(+ 式 \verb+)+ | \verb+bool(+ 式 \verb+)+ \\
関数呼出:    & 名前 \verb+(+ 【 代入式 《 \verb+,+ 代入式 》※ 】 \verb+)+\\
&\\
その他 & \\
コメント:    & \verb+/* ... */+ または \verb+// ...+ \\
ディレクティブ: & \verb+#+ 行番号 \verb+"+ファイル名\verb+"+\\
\end{tabular}
}
  % 付録 文法まとめ
% 
% 付録:コマンドリファレンス
%
\chapter{コマンドリファレンス}
\label{app:command}

%UNIX や MacOS 上で動作する{\cmm}コンパイラの使用方法を解説します。

\section{{\cme}コマンド}

{\cmm}プログラムを{\tac}で実行できる{\tt .exe}ファイルに変換します。
{\cme}コマンドは、内部で「\ref{command:cmmc} {\cmmc}コマンド」や
「{\tt Util--}ユーティリティ」プログラムを自動的に呼び出すシェルスクリプトです。

\begin{flushleft}
{\bf 形式 : }~~~\verb/cm2e [-h] [-o exec] [-s ###] [-E] [-S] [-K] [-c] [-Dxx=yy] <file>.../
\end{flushleft}

{\tt <file>...}の各ファイルについて、
プリプロセッサ({\tt cpp})、
コンパイラ({\tt c--})、
アセンブラ({\tt as--})を順に呼び出し、
リロケータブルオブジェクト(「{\tt Util--}解説書」参照)に変換します。
次に、
リンカー({\tt ld--})を用いリロケータブルオブジェクトを結合します。
最後に、実行可能形式作成プログラム({\tt objexe--})を呼び出し
{\tt .exe}ファイルを作成します。

{\cme}は、
指定されたファイルの拡張子からファイルの種類を判断し、
必要な処理を自動的に判断します。
拡張子「{\tt .cmm}」は{\cmml}のソースプログラム、
「{\tt .s}」は{\tac}のアセンブリ言語プログラム、
「{\tt .o}」は{\tac}のリロケータブルオブジェクトと判断します。

\begin{quote}
\hspace{-1em}以下のオプションが使用できます。

\begin{description}
\item[{\tt -h}] : 使用方法メッセージを表示します。
\item[{\tt -o}] : 作成する{\tt .exe}ファイルの名前を指定します。
{\tt -o}オプションの後ろに空白で区切ってファイル名を入力します。
\item[{\tt -s}] : プログラムのスタック領域サイズをバイト単位で明示します。
{\tt -s}オプションの後ろに空白で区切ってサイズを10進数で入力します。
スタックサイズを明示しない場合は、規定値の600が使用されます。
\item[{\tt -E}] : プリプロセッサで処理した{.cmm}ファイルの内容を
標準出力ストリームに書き出します。
\item[{\tt -S}] : アセンブラソースプログラム{\tt .s}の作成まで行い、
それより後の処理を行いません。
\item[{\tt -K}] : {\tacos}カーネル用モードでコンパイルを行います。
コンパイル結果に、
ユーザプログラム用のスタックオーバーフローチェック機能を埋め込みません。
\item[{\tt -c}] : リロケータブルオブジェクトファイル{\tt .o}の作成まで行い、
それより後の処理を行いません。
\item[{\tt -D}] : このオプションは、そのままプリプロセッサに渡されます。
次に使用例を示します。\\
\verb/$ cm2e -DDEBUG=1 -o hello hello.cmm/
\end{description}
\end{quote}

\section{{\cmc}コマンド}

{\cmm}プログラムを{\cl}プログラムに変換した後、
{\cl}プログラムをコンパイルしてUNIXやMacの実行形式を作成します。
{\cmc}コマンドは、内部で「\ref{command:ccmmc} {\ccmmc}コマンド」や
{\tt C}コンパイラドライバを自動的に呼び出すシェルスクリプトです。

\begin{flushleft}
{\bf 形式 : }~~~\verb/cm2c [-h] [-o exec] [-S] [-E] [-c] [-Dxx=yy] <file>.../
\end{flushleft}

{\tt <file>...}の各ファイルについて、
プリプロセッサ({\tt cpp})
トランスレータ({\tt c-c--})、
{\tt C}コンパイラドライバ({\tt cc})を順に呼び出し、
UNIXやMacのリロケータブルオブジェクトに変換します。
次に、{\tt C}コンパイラドライバ({\tt cc})を呼び出し、
リロケータブルオブジェクトを結合し実行可能ファイルを作成します。

{\cmc}は、
指定されたファイルの拡張子をからファイルの種類を判断し、
必要な処理を自動的に判断します。
拡張子「{\tt .cmm}」は{\cmml}のソースプログラム、
「{\tt .c}」は{\cl}ソースプログラム、
「{\tt .o}」はUNIXやMacのリロケータブルオブジェクトと判断します。

\begin{quote}
\hspace{-1em}以下のオプションが使用できます。

\begin{description}
\item[{\tt -h}] : 使用方法メッセージを表示します。
\item[{\tt -o}] : 作成する実行可能ファイルの名前を指定します。
{\tt -o}オプションの後ろに空白で区切ってファイル名を入力します。
\item[{\tt -E}] : プリプロセッサで処理した{.cmm}ファイルの内容を
標準出力ストリームに書き出します。
\item[{\tt -S}] : {\cl}ソースプログラム{\tt .c}の作成まで行い、
それより後の処理を行いません。
\item[{\tt -c}] : リロケータブルオブジェクトファイル{\tt .o}の作成まで行い、
それより後の処理を行いません。
\item[{\tt -D}] : このオプションは、そのままプリプロセッサに渡されます。
\end{description}
\end{quote}

\section{{\cmv}コマンド}

{\cmm}プログラムをコンパイルして仮想スタックマシンのニーモニックに変換します。
{\cmv}コマンドは、
内部で「\ref{command:vcmmc} {\vcmmc}コマンド」を呼び出すシェルスクリプトです。

\begin{flushleft}
{\bf 形式 : }~~~\verb/cm2v [-h] [-E] [-Dxx=yy] <file>.../
\end{flushleft}

{\tt <file>...}の各ファイルについて、
プリプロセッサ({\tt cpp})、
コンパイラ({\tt vm-c--})
を順に呼び出し仮想スタックマシンのニーモニック({\tt .v})を出力します。
{\cmv}コマンドに指定できるファイルは、
拡張子「{\tt .cmm}」の{\cmml}ソースプログラムだけです。

\begin{quote}
\hspace{-1em}以下のオプションが使用できます。

\begin{description}
\item[{\tt -h}] : 使用方法メッセージを表示します。
\item[{\tt -E}] : プリプロセッサで処理した{.cmm}ファイルの内容を
標準出力ストリームに書き出します。
それより後の処理を行いません。
\item[{\tt -D}] : このオプションは、そのままプリプロセッサに渡されます。
\end{description}
\end{quote}

\section{{\cmmc}コマンド}
\label{command:cmmc}

{\cmml}の{\tac}用コンパイラです。
{\bf 通常は{\cme}から起動されユーザが直接使用することはありません。}
\cmml で記述されたプログラムを入力し、
\tac アセンブリ言語で記述したプログラムに変換します。
\cmmc コマンドの書式は次の通りです。

\begin{flushleft}
{\bf 形式 : }~~~\verb/c-- [-h] [-v] [-O0] [-O] [-O1] [-K] [<source file>]/\\
({\bf 注意}:オプションは書式の順番で指定する必要があります。)
\end{flushleft}

引数に \cmml のソースプログラムファイルを指定した場合は、
指定されたファイルからソースプログラムを読み込みます。
ファイルが省略された場合は標準入力ストリームからソースプログラムを読み込みます。
どちらの場合もコンパイル結果は標準出力ストリームに出力します。
ソースプログラムファイルの拡張子は「\verb/.cmm/」にします。

\verb/-h/、\verb/-v/オプションは使用法メッセージを表示します。
\verb/-O0/オプションを指定すると、
ソースコード中の定数式をコンパイル時に計算したり、
実行されることがないプログラムの部分を削除したりする等の最適化をしません。
\verb/-O/、\verb/-O1/は最適化を促すオプションですが、
デフォルトで\verb/ON/になっているので指定する必要はありません。
\verb/-K/オプションを使うと、
関数入口へのスタックオーバーフロー検出コードの埋め込みが抑制されます。
\tacos のカーネルをコンパイルするときに使用するオプションです。

%\cmmc コマンドは、\cl のプリプロセッサと組み合わせて次のように使用します。
%コンパイル結果はリダイレクトを用いて拡張子「\verb/.s/」のファイルに出力します。
%
%\begin{mylist}
%\begin{verbatim}
%$ cc -E -std=c99 -nostdinc -I/usr/local/cmmInclude \
%-I/usr/local/cmmLib - < hello.cmm | c-- > hello.s
%\end{verbatim}
%\end{mylist}
%
%この実行例は、OSX 10.11 の場合です。
%OSX 10.11 の\verb/cpp/コマンドは、
%\verb;//;コメントをうまく処理できないようです。
%\verb/cc -E/コマンドを代用にするとうまく処理できます。
%他の実行環境では、
%その環境にあう実行方法を試行錯誤する必要があるかもしれません。

\section{{\ccmmc}コマンド}
\label{command:ccmmc}

{\cmm}プログラムを{\cl}に変換して出力するトランスレータです。
{\bf 通常は{\cmc}から起動されユーザが直接使用することはありません。}

\begin{flushleft}
{\bf 形式 : }~~~\verb/c-c-- [-h] [-v] [-O0] [-O] [-O1] [-K] [<source file>]/\\
({\bf 注意}:オプションは書式の順番で指定する必要があります。)
\end{flushleft}

%出力された{\cl}は、
%\verb;/user/local/cmmLib/; にインストールされた、
%\verb;cfunc.hmm; や \verb;wrapper.c; を用いて、
%UNIX や MacOS で実行できるようになります。

引数の意味は{\cmmc}コマンドと同様です。
%次の実行例は変換結果を画面に表示しています。

%\begin{mylist}
%\begin{verbatim}
%$ cc -E -std=c99 -nostdinc -I/usr/local/cmmInclude \
% -I/usr/local/cmmLib - < hello.cmm | c-c--
%#include <stdio.h>
%#define _cmm_str_L0 "hello,world\n"
%int main(){
%printf(_cmm_str_L0);
%}
%\end{verbatim}
%\end{mylist}

%次の実行例は変換結果を\verb/hello.c/ファイルに格納したあと、
%{\cl}コンパイラでコンパイルして実行した例です。

%\begin{mylist}
%\begin{verbatim}
%$ cc -E -std=c99 -nostdinc -I/usr/local/cmmInclude \
% -I/usr/local/cmmLib - < hello.cmm | c-c-- > hello.c
%$ cc -o hello -Wno-parentheses-equality hello.c \
% /usr/local/cmmLib/wrapper.c
%$ ./hello
%hello,world
%\end{verbatim}
%\end{mylist}

\section{{\vcmmc}コマンド}
\label{command:vcmmc}

仮想スタックマシンのニーモニックを出力する{\cmm}コンパイラです。
{\bf 通常は{\cmv}から起動されユーザが直接使用することはありません。}

仮想スタックマシンのニーモニックは、
コンパイラ内部で用いている中間言語(\pageref{app:vm}ページ参照)と、
ほぼ一対一に対応します。
中間言語や仮想スタックマシンを学習したいときに利用します。
{\vcmmc}コマンドの書式は次の通りです。

\begin{flushleft}
{\bf 形式 : }~~~\verb/vm-c-- [-h] [-v] [-O0] [-O] [-O1] [-K] [<source file>]/
\end{flushleft}

引数の意味は{\cmmc}コマンドと同様です。
%次の実行例は変換結果を画面に表示しています。
%
%\begin{mylist}
%\begin{verbatim}
%$ cc -E -std=c99 -nostdinc -I/usr/local/cmmInclude \
% - < hello.cmm | vm-c--
%_stdin
%        WS      1
%_stdout
%        WS      1
%_stderr
%        WS      1
%.L1
%        STRING  "hello,world\n"
%_main
%        ENTRY   0
%        LDC     .L1
%        ARG
%        CALLF   1,_printf
%        POP
%        LDC     0
%        MREG
%        RET
%\end{verbatim}
%\end{mylist}

 % 付録 コマンドリファレンス
% 
%  付録:中間言語
%
\chapter{中間言語}
\label{app:vm}

\cmm コンパイラに入力された\cmm プログラムは、
一旦、以下で説明する中間言語に変換されます。
その後、中間言語から仮想のスタックマシンや
\tac のニーモニックに変換されます。
なお、{\cl}トランスレータは構文木から直接{\cl}を生成するので、
中間言語を用いません。

\section{仮想スタックマシン}

以下では中間言語を変換する先として仮想のスタックマシンを想定します。
仮想スタックマシンの命令は、中間言語から、ほぼ、一対一に変換できます。
仮想スタックマシンは、次のようなものです。

\begin{itemize}
\item 
仮想スタックマシンが扱うデータは、基本的にワードデータ(16bit)です。
{\cmml}の\verb/int/型、参照(アドレス)はワードデータにピッタリ格納されます。
\verb/char/型はワードデータの下位8bit、
\verb/boolean/型はワードデータの下位の1bitに格納されます。

\item
\verb/char/型と\verb/boolean/型の{\bf 配列要素だけ}、
メモリ節約のためバイトデータ(8bit)です。
\verb/LDB/命令、\verb/STB/命令がバイト配列のデータをアクセスします。

\item
仮想スタックマシンのプログラムは次の書式のニーモニックで記述します。

\begin{quote}
\begin{verbatim}
[ラベル]    [命令   [オペランド[,オペランド]...]]
\end{verbatim}
\end{quote}

\end{itemize}

\section{書式}

中間言語は次のような命令行で表現されます。

\begin{quote}
\begin{verbatim}
命令([オペランド[,オペランド]...])

例: vmLdCns(3)    // 定数3をスタックに積む
\end{verbatim}
\end{quote}

\section{命令}

中間言語の命令には「ラベル生成命令」、「マシン命令」、
「マクロ命令」、「擬似命令」があります。

\subsection{ラベル生成命令}

プログラムのジャンプ先やデータのためにラベルを生成します。

\subsubsection{vmName}

名前を表現するラベルを宣言します。
名前表の\verb/idx/番目に登録されている名前を
ラベルとして定義するニーモニックを出力します。
ラベルの先頭には\verb/'.'/または\verb/'_'/が付加されます。
\cmml で\verb/public/修飾された名前に\verb/'_'/が付加されます。

\begin{quote}
\begin{verbatim}
中 間 言 語 : vmName(idx)
ニーモニック: ラベル

変換例:vmName(3)  => .a     // 名前表の3番目に a があった場合
\end{verbatim}
\end{quote}

\subsubsection{vmTmpLab}

コンパイラが自動的に生成した番号で管理されるラベルを出力します。
このラベルは\cmm プログラムソースには存在しない名前です。
整数\verb/n/で区別できるラベル\verb/ln/をニーモニックに出力します。

\begin{quote}
\begin{verbatim}
中 間 言 語 : vmTmpLab(n)
ニーモニック: ln

変換例:vmTmpLab(3)  => .L3
\end{verbatim}
\end{quote}

\subsection{マシン命令}

スタックマシンの命令や\tac の機械語命令に変換されるべき、
中間言語命令です。

\subsubsection{vmEntry}

関数の入口処理をする命令です。
\verb/idx/は名前表で関数名が登録されている位置です。
\verb/n/は関数の中で「{\bf 同時に使用される}ローカル変数の数」です。
\verb/vmEntry/は、\verb/n/個のローカル変数領域を確保します。
関数内でスコープが切り替わり
同じ領域が複数のローカル変数で共用できる場合があるので、
「{\bf 同時に使用される}ローカル変数の数」が指定されます。

\begin{quote}
\begin{verbatim}
中 間 言 語 : vmEntry(n,idx)
ニーモニック: ラベル  ENTRY   n

変換例:vmEntry(1,3)  => .a  ENTRY  1
\end{verbatim}
\end{quote}

\subsubsection{vmEntryK}

カーネル関数の入口処理をする命令です。
引数の意味は\verb/vmEntry/と同じです。
\cmm コンパイラに\verb/-K/オプションを付けて実行した場合は、
\verb/vmEntry/のかわりに\verb/vmEntryK/が使用されます。

\begin{quote}
\begin{verbatim}
中 間 言 語 : vmEntryK(n,idx)
ニーモニック: ラベル  ENTRYK  n

変換例:vmEntryK(1,3) => .a  ENTRYK 1
\end{verbatim}
\end{quote}

\subsubsection{vmRet}

関数の出口処理をする命令です。
ローカル変数を捨てて関数から戻ります。
通常の関数、カーネル関数で共通に使用します。

\begin{quote}
\begin{verbatim}
中 間 言 語 : vmRet()
ニーモニック: RET

変換例:vmRet() => RET
\end{verbatim}
\end{quote}


\subsubsection{vmEntryI}

\verb/interrupt/型関数の入口処理をする命令です。
引数の意味は\verb/vmEntry/と同じです。

\begin{quote}
\begin{verbatim}
中 間 言 語 : vmEntryI(n,idx)
ニーモニック: ラベル  ENTRYI  n

変換例:vmEntryI(1,3) => .a  ENTRYI 1
\end{verbatim}
\end{quote}

\subsubsection{RETI}

\verb/interrupt/関数の出口処理をする命令です。
ローカル変数を捨てて\verb/interrupt/型関数から戻ります。

\begin{quote}
\begin{verbatim}
中 間 言 語 : vmRet()
ニーモニック: RETI
\end{verbatim}
\end{quote}

\subsubsection{vmMReg}

スタックから関数の返り値を取り出し、
返り値用のハードウェアレジスタに移動します。

\begin{quote}
\begin{verbatim}
中 間 言 語 : vmMReg()
ニーモニック: MREG
\end{verbatim}
\end{quote}

\subsubsection{vmArg}

関数を呼出す前に、関数に渡す引数を準備します。
スタックから値を取り出し引数領域にコピーします。
複数の引数がある場合は、最後の引数から順に処理します。

\begin{quote}
\begin{verbatim}
中 間 言 語 : vmArg()
ニーモニック: ARG
\end{verbatim}
\end{quote}

以下に二つの引数を持つ関数\verb/f/を呼び出す例を示します。

\begin{mylist}
\begin{verbatim}
// C-- ソース
void f(int a, int b) { ... }
void g() { f(1, 2); }

// 関数 g の中間コード
vmEntry(0,5)  // 名前表の5番目に g があるとする
vmLdCns(2)
vmArg()
vmLdCns(1)
vmArg()
vmCallP(2,4)  // 名前表の4番目に f があるとする
vmRet()

// 関数 g のニーモニック
.g
        ENTRY   0
        LDC     2
        ARG
        LDC     1
        ARG
        CALLP   2,.f
        RET
\end{verbatim}
\end{mylist}

\subsubsection{vmCallP}

値を返さない関数を呼び出します。
\verb/n/は関数引数の個数、
\verb/idx/は名前表で関数名が登録されている位置です。

\begin{quote}
\begin{verbatim}
中 間 言 語 : vmCallP(n,idx)
ニーモニック: CALLP n,ラベル

変換例:vmCallP(2, 4) => CALLP 2,.f
\end{verbatim}
\end{quote}

\subsubsection{vmCallF}

値を返す関数を呼び出します。
\verb/n/と\verb/idx/の意味は\verb/vmCallP/と同じです。
\verb/vmCallF/は関数の実行が終了した時、
返り値をハードウェアレジスタから取り出しスタックに積みます。

\begin{quote}
\begin{verbatim}
中 間 言 語 : vmCallF(n,idx)
ニーモニック: CALLF n,ラベル

変換例:vmCallF(2, 4) => CALLF 2,.f
\end{verbatim}
\end{quote}

\subsubsection{vmJmp}

無条件ジャンプ命令です。
整数\verb/n/はジャンプ先ラベルの番号を表します。
ラベル\verb/ln/は\verb/vmTmpLab/で出力される\verb/n/番目のラベルです。

\begin{quote}
\begin{verbatim}
中 間 言 語 : vmJmp(n)
ニーモニック: JMP   ln

変換例:vmJmp(3) => JMP .L3
\end{verbatim}
\end{quote}

\subsubsection{vmJT}

スタックから論理値を取り出し\verb/true/ならジャンプします。
\verb/n/、\verb/ln/の意味は\verb/vmJmp/と同じです。

\begin{quote}
\begin{verbatim}
中 間 言 語 : vmJT(n)
ニーモニック: JT   ln

変換例:vmJT(3) => JT  .L3
\end{verbatim}
\end{quote}

\subsubsection{vmJF}

スタックから論理値を取り出し\verb/false/ならジャンプします。
\verb/n/、\verb/ln/の意味は\verb/vmJmp/と同じです。

\begin{quote}
\begin{verbatim}
中 間 言 語 : vmJF(n)
ニーモニック: JF   ln

変換例:vmJF(3) => JF  .L3
\end{verbatim}
\end{quote}

\subsubsection{vmLdCns}

定数\verb/c/をスタックに積みます。

\begin{quote}
\begin{verbatim}
中 間 言 語 : vmLdCns(c)
ニーモニック: LDC   c

変換例:vmLdCns(3) => LDC  3
\end{verbatim}
\end{quote}

\subsubsection{vmLdGlb}

グローバル変数の値をスタックに積みます。
\verb/idx/は名前表で変数名が登録されている位置です。

\begin{quote}
\begin{verbatim}
中 間 言 語 : vmLdGlb(idx)
ニーモニック: LDG   ラベル

変換例:vmLdGlb(3) => LDG .a
\end{verbatim}
\end{quote}

\subsubsection{vmLdLoc}

\verb/n/番目のローカル変数の値をスタックに積みます。
ローカル変数の番号は\verb/1/から始まります。

\begin{quote}
\begin{verbatim}
中 間 言 語 : vmLdLoc(n)
ニーモニック: LDL  n   

変換例:vmLdLoc(3) => LDL  3
\end{verbatim}
\end{quote}

\subsubsection{vmLdArg}

現在の関数の\verb/n/番目の引数の値をスタックに積みます。
引数の番号は第1引数から順に、\verb/1/以上の番号が割り振られます。

\begin{quote}
\begin{verbatim}
中 間 言 語 : vmLdArg(n)
ニーモニック: LDA  n   

変換例:vmLdArg(3) => LDA  3
\end{verbatim}
\end{quote}

\subsubsection{vmLdStr}

文字列リテラルの参照をスタックに積みます。
整数\verb/n/はラベルの番号を表します。
ラベル\verb/ln/は\verb/vmTmpLab/で出力される\verb/n/番目のラベルです。

\begin{quote}
\begin{verbatim}
中 間 言 語 : vmLdStr(n)
ニーモニック: LDC  ln   

変換例:vmLdStr(3) => LDC  .L3
\end{verbatim}
\end{quote}

\subsubsection{vmLdLab}

ラベルの値(アドレス)をスタックに積みます。
\verb/idx/は名前表でラベルが登録されている位置です。

\begin{quote}
\begin{verbatim}
中 間 言 語 : vmLdLab(idx)
ニーモニック: LDC  ラベル

変換例:vmLdLab(3) => LDC  .a
\end{verbatim}
\end{quote}

\subsubsection{vmLdWrd}

ワード配列の要素を読み出すための命令です。
まずスタックから、添字、ワード配列のアドレスの順に取り出します。
次にワード配列の要素の内容をスタックに積みます。

\begin{quote}
\begin{verbatim}
中 間 言 語 : vmLdWrd()
ニーモニック: LDW
\end{verbatim}
\end{quote}

\subsubsection{vmLdByt}

バイト配列の要素を読み出すための命令です。
まずスタックから、添字、バイト配列のアドレスの順に取り出します。
次にバイト配列の要素の内容をワードに変換してスタックに積みます。
変換はバイトデータの上位に\verb/0/のビットを付け加えることで行います。

\begin{quote}
\begin{verbatim}
中 間 言 語 : vmLdByt()
ニーモニック: LDB
\end{verbatim}
\end{quote}

\subsubsection{vmStGlb}

スタックトップの値をグローバル変数にストアします。
\verb/idx/は名前表で変数名が登録されている位置です。
スタックをポップしません。

\begin{quote}
\begin{verbatim}
中 間 言 語 : vmStGlb(idx)
ニーモニック: STG   ラベル

変換例:vmStGlb(3) => STG .a
\end{verbatim}
\end{quote}

\subsubsection{vmStLoc}

スタックトップの値を\verb/n/番目のローカル変数にストアします。
スタックをポップしません。

\begin{quote}
\begin{verbatim}
中 間 言 語 : vmStLoc(n)
ニーモニック: STL   n

変換例:vmStLoc(3) => STL 3
\end{verbatim}
\end{quote}

\subsubsection{vmStArg}

スタックトップの値を\verb/n/番目の引数にストアします。
スタックをポップしません。

\begin{quote}
\begin{verbatim}
中 間 言 語 : vmStArg(n)
ニーモニック: STA  n   

変換例:vmStArg(3) => STA  3
\end{verbatim}
\end{quote}

\subsubsection{vmStWrd}

ワード配列の要素を書き換える命令です。
まずスタックから、添字、ワード配列のアドレスの順に取り出します。
次にスタックトップの値をワード配列の要素に書き込みます。
後半ではスタックをポップしません。

\begin{quote}
\begin{verbatim}
中 間 言 語 : vmStWrd()
ニーモニック: STW
\end{verbatim}
\end{quote}

\subsubsection{vmStByt}

バイト配列の要素を書き換える命令です。
まずスタックから、添字、バイト配列のアドレスの順に取り出します。
次にスタックトップの値をバイトデータに変換して配列に書き込みます。
後半ではスタックをポップしません。
変換はワードデータの下位8bitを取り出すことで行います。

\begin{quote}
\begin{verbatim}
中 間 言 語 : vmStByt()
ニーモニック: STB
\end{verbatim}
\end{quote}

\subsubsection{vmNeg}

まず、スタックから整数を取り出し2の補数を計算します。
次に、計算結果をスタックに積みます。

\begin{quote}
\begin{verbatim}
中 間 言 語 : vmNeg()
ニーモニック: NEG
\end{verbatim}
\end{quote}

\subsubsection{vmNot}

まず、スタックから論理値を取り出し否定を計算します。
次に、計算結果をスタックに積みます。

\begin{quote}
\begin{verbatim}
中 間 言 語 : vmNot()
ニーモニック: NOT
\end{verbatim}
\end{quote}

\subsubsection{vmBNot}

まず、スタックから整数を取り出し1の補数を計算します。
次に、計算結果をスタックに積みます。

\begin{quote}
\begin{verbatim}
中 間 言 語 : vmBNot()
ニーモニック: BNOT
\end{verbatim}
\end{quote}

\subsubsection{vmChar}

まず、スタックから整数を取り出し下位8bitだけ残しマスクします。
次に、計算結果をスタックに積みます。

\begin{quote}
\begin{verbatim}
中 間 言 語 : vmChar()
ニーモニック: CHAR
\end{verbatim}
\end{quote}

\subsubsection{vmBool}

まず、スタックから整数を取り出し最下位ビットだけ残しマスクします。
次に、計算結果をスタックに積みます。

\begin{quote}
\begin{verbatim}
中 間 言 語 : vmBool()
ニーモニック: BOOL
\end{verbatim}
\end{quote}

\subsubsection{vmAdd}

まず、スタックから整数を二つ取り出し和を計算します。
次に、計算結果をスタックに積みます。

\begin{quote}
\begin{verbatim}
中 間 言 語 : vmAdd()
ニーモニック: ADD
\end{verbatim}
\end{quote}

\subsubsection{vmSub}

まず、スタックから整数を一つ取り出し$x$とします。
次に、スタックから整数をもう一つ取り出し$y$とします。
最後に、スタックに$x-y$を積みます。

\begin{quote}
\begin{verbatim}
中 間 言 語 : vmSub()
ニーモニック: SUB
\end{verbatim}
\end{quote}

\subsubsection{vmShl}

まず、スタックからシフトするビット数、シフトされるデータの順に取り出します。
次に、左シフトを計算します。
最後に、計算結果をスタックに積みます。

\begin{quote}
\begin{verbatim}
中 間 言 語 : vmShl()
ニーモニック: SHL
\end{verbatim}
\end{quote}

\subsubsection{vmShr}

まず、スタックからシフトするビット数、シフトされるデータの順に取り出します。
次に、{\bf 算術}右シフトを計算します。
最後に、計算結果をスタックに積みます。

\begin{quote}
\begin{verbatim}
中 間 言 語 : vmShr()
ニーモニック: SHR
\end{verbatim}
\end{quote}

\subsubsection{vmBAnd}

まず、スタックから整数を二つ取り出しビット毎の論理積を計算します。
次に、計算結果をスタックに積みます。

\begin{quote}
\begin{verbatim}
中 間 言 語 : vmBAnd()
ニーモニック: BAND
\end{verbatim}
\end{quote}

\subsubsection{vmBXor}

まず、スタックから整数を二つ取り出しビット毎の排他的論理和を計算します。
次に、計算結果をスタックに積みます。

\begin{quote}
\begin{verbatim}
中 間 言 語 : vmBXor()
ニーモニック: BXOR
\end{verbatim}
\end{quote}

\subsubsection{vmBOr}

まず、スタックから整数を二つ取り出しビット毎の論理和を計算します。
次に、計算結果をスタックに積みます。

\begin{quote}
\begin{verbatim}
中 間 言 語 : vmBOr()
ニーモニック: BOR
\end{verbatim}
\end{quote}

\subsubsection{vmMul}

まず、スタックから整数を二つ取り出し積を計算します。
次に、計算結果をスタックに積みます。

\begin{quote}
\begin{verbatim}
中 間 言 語 : vmMul()
ニーモニック: MUL
\end{verbatim}
\end{quote}

\subsubsection{vmDiv}

まず、スタックから整数を一つ取り出し$x$とします。
次に、スタックから整数をもう一つ取り出し$y$とします。
最後に、スタックに$x \div y$を積みます。

\begin{quote}
\begin{verbatim}
中 間 言 語 : vmDiv()
ニーモニック: DIV
\end{verbatim}
\end{quote}

\subsubsection{vmMod}

まず、スタックから整数を一つ取り出し$x$とします。
次に、スタックから整数をもう一つ取り出し$y$とします。
最後に、スタックに$x$を$y$で割った余りを積みます。

\begin{quote}
\begin{verbatim}
中 間 言 語 : vmMod()
ニーモニック: MOD
\end{verbatim}
\end{quote}

\subsubsection{vmGt}

まず、スタックから整数を一つ取り出し$x$とします。
次に、スタックから整数をもう一つ取り出し$y$とします。
最後に、比較($x > y$)の結果(論理値)をスタックに積みます。

\begin{quote}
\begin{verbatim}
中 間 言 語 : vmGt()
ニーモニック: GT
\end{verbatim}
\end{quote}

\subsubsection{vmGe}

まず、スタックから整数を一つ取り出し$x$とします。
次に、スタックから整数をもう一つ取り出し$y$とします。
最後に、比較($x \ge y$)の結果(論理値)をスタックに積みます。

\begin{quote}
\begin{verbatim}
中 間 言 語 : vmGe()
ニーモニック: GE
\end{verbatim}
\end{quote}

\subsubsection{vmLt}

まず、スタックから整数を一つ取り出し$x$とします。
次に、スタックから整数をもう一つ取り出し$y$とします。
最後に、比較($x < y$)の結果(論理値)をスタックに積みます。

\begin{quote}
\begin{verbatim}
中 間 言 語 : vmLt()
ニーモニック: LT
\end{verbatim}
\end{quote}

\subsubsection{vmLe}

まず、スタックから整数を一つ取り出し$x$とします。
次に、スタックから整数をもう一つ取り出し$y$とします。
最後に、比較($x \le y$)の結果(論理値)をスタックに積みます。

\begin{quote}
\begin{verbatim}
中 間 言 語 : vmLe()
ニーモニック: LE
\end{verbatim}
\end{quote}

\subsubsection{vmEq}

まず、スタックから整数を一つ取り出し$x$とします。
次に、スタックから整数をもう一つ取り出し$y$とします。
最後に、比較($x = y$)の結果(論理値)をスタックに積みます。

\begin{quote}
\begin{verbatim}
中 間 言 語 : vmEq()
ニーモニック: EQ
\end{verbatim}
\end{quote}

\subsubsection{vmNe}

まず、スタックから整数を一つ取り出し$x$とします。
次に、スタックから整数をもう一つ取り出し$y$とします。
最後に、比較($x \neq y$)の結果(論理値)をスタックに積みます。

\begin{quote}
\begin{verbatim}
中 間 言 語 : vmNe()
ニーモニック: NE
\end{verbatim}
\end{quote}

\subsubsection{vmPop}

スタックから値を一つ取り出し捨てます。

\begin{quote}
\begin{verbatim}
中 間 言 語 : vmPop()
ニーモニック: POP
\end{verbatim}
\end{quote}

\subsection{マクロ命令}

コード生成にヒントを与えるために、
ニーモニックに対応するレベルまで展開しないで、
マクロ命令として中間コードを出力する場合があります。

\subsubsection{vmBoolOR}

論理OR式の最後で計算結果の論理値をスタックに積むマクロ命令です。
整数\verb/n1/、\verb/n2/、\verb/n3/はラベルの番号を表します。
論理式の途中から\verb/true/、\verb/false/が定まった時点で
マクロを展開したニーモニック中の\verb/n1/、
\verb/n2/ラベルにジャンプしてきます。
\verb/n2/が\verb/-1/の場合は短く展開されます。

\begin{quote}
\begin{verbatim}
中 間 言 語 : vmBoolOR(n1, n2, n3)
ニーモニック: 以下のように展開されます

vmBoolOR(1, 2, 3)  | vmBoolOR(1, -1, 3)
-------------------+--------------------
      JMP   .L3    |      JMP   .L3
.L1                | .L1
      LDC   1      |      LDC   1
      JMP   .L3    | .L3
.L2                |
      LDC   0      |
.L3                |
\end{verbatim}
\end{quote}

\subsubsection{vmBoolAND}

論理AND式の最後で計算結果の論理値をスタックに積むマクロ命令です。

\begin{quote}
\begin{verbatim}
中 間 言 語 : vmBoolAND(n1, n2, n3)
ニーモニック: 以下のように展開されます

vmBoolAND(1, 2, 3) | vmBoolAND(1, -1, 3)
-------------------+--------------------
      JMP   .L3    |      JMP   .L3
.L1                | .L1
      LDC   0      |      LDC   0
      JMP   .L3    | .L3
.L2                |
      LDC   1      |
.L3                |
\end{verbatim}
\end{quote}

\subsection{擬似命令}

データ生成用の疑似命令です。

\subsubsection{vmDwName}

名前へのポインタを生成します。
\verb/idx/は名前表で名前が登録されている位置です。

\begin{quote}
\begin{verbatim}
中 間 言 語 : vmDwName(idx)
ニーモニック: DW  ラベル

変換例:vmDwName(3) => DW .a
\end{verbatim}
\end{quote}

\subsubsection{vmDwLab}

\verb/vmTmpLab/で出力されるラベルへのポインタを生成します。
整数\verb/n/はラベルの番号を表します。
ラベル\verb/ln/は\verb/vmTmpLab/で出力される\verb/n/番目のラベルです。

\begin{quote}
\begin{verbatim}
中 間 言 語 : vmDwLab(n)
ニーモニック: DW  ln   

変換例:vmDwLab(3) => DW  .L3
\end{verbatim}
\end{quote}

\subsubsection{vmDwCns}

ワードデータを生成します。
整数\verb/c/は生成するデータの値です。

\begin{quote}
\begin{verbatim}
中 間 言 語 : vmDwCns(c)
ニーモニック: DW  c   

変換例:vmDwCns(3) => DW  3
\end{verbatim}
\end{quote}

\subsubsection{vmDbCns}

バイトデータを生成します。
整数\verb/c/は生成するデータの値です。
\verb/vmDbCns/はバイト配列の初期化で使用されます。

\begin{quote}
\begin{verbatim}
中 間 言 語 : vmDbCns(c)
ニーモニック: DB  c   

変換例:vmDbCns(3) => DB  3
\end{verbatim}
\end{quote}

\subsubsection{vmWs}

ワードデータ領域(配列)を生成します。
整数\verb/n/は生成するワードの数です。

\begin{quote}
\begin{verbatim}
中 間 言 語 : vmWs(n)
ニーモニック: WS  n

変換例:vmWs(3) => WS  3
\end{verbatim}
\end{quote}

\subsubsection{vmBs}

バイトデータ領域(配列)を生成します。
整数\verb/n/は生成するバイトの数です。

\begin{quote}
\begin{verbatim}
中 間 言 語 : vmBs(n)
ニーモニック: BS  n

変換例:vmWs(3) => BS  3
\end{verbatim}
\end{quote}

\subsubsection{vmStr}

文字列を生成します。
整数\verb/str/は文字列です。

\begin{quote}
\begin{verbatim}
中 間 言 語 : vmStr(s)
ニーモニック: STRING s

変換例:vmStr("hello\n") => STRING "hello\n"
\end{verbatim}
\end{quote}
      % 付録 中間言語

% 発行元など
\backmatter
\pagestyle{empty}
\onecolumn
~
\subsubsection{変更履歴}
\begin{flushleft}
2016年03月15日 \cmm V.3.0.0 用作成  \\
\end{flushleft}

\vfil

\subsubsection{対応ソフトウェアのバージョン}
\begin{tabular}{|l|l|}
\hline
{\tt C--}      & Ver.3.0.0 \\
\hline
{\tt AS--}     & Ver.2.1.1 \\
\hline
{\tt LD--}     & Ver.2.0.0 \\
\hline
{\tt OBJBIN--} & Ver.2.0.0 \\
\hline
{\tt SIZE--}   & Ver.2.0.0 \\
\hline
{\tt OBJEXE--} & Ver.1.0.0 \\
\hline
\end{tabular}

\vfill\vfill
\begin{center}
\fbox{\parbox{10cm}{ \vspace{0.3cm}
\large{\bf プログラミング言語 \cmm} \\
\\
 発行年月 2016年3月 \ver \\
 発  行 独立行政法人国立高等専門学校機構 \\
      徳山工業高等専門学校 \\
      情報電子工学科 重村哲至 \\
      〒745-8585 山口県周南市学園台 \\
      sigemura@tokuyama.ac.jp \\
}}
\end{center}
\vfill
\end{document}
