% 
% 2章 C--言語開発環境のインストール
%
\chapter{\cmml 開発環境のインストール}

\cmml を体験するために、自分のパソコンで\cmm コンパイラを
使用できるようにしましょう。


\section{ユーティリティのインストール}

\tac 用のプログラムを作成するために、まず、
\cmm コンパイラ用ユーティリティをインストールする必要があります。
ソースコードは
\url{https://github.com/tctsigemura/Util--/}から入手します。
インストールの準備ができたらダウンロードした配布物を解凍し
\verb/Util--/ディレクトリで以下のように操作します。

\begin{mylist}
\begin{verbatim}
$ make
...
cc -std=c99  -DDATE="\"`date`\"" -DVER=\"2.1.1\" \
	-o as-- syntax.c lexical.c util.c
cc -std=c99 -DDATE="\"`date`\"" -DVER=\"2.0.0\" -DARC=\"TaC\"
  -o ld-- ld.c
cc -std=c99 -o objbin-- objbin.c
cc -std=c99 -o size-- size.c
cc -std=c99 -o objexe-- objexe.c
$ sudo make install
...
install -d -m 755 /usr/local/bin
install -m 755 as-- /usr/local/bin
...
$
\end{verbatim}
\end{mylist}

以上で、\as 、\ld 、\objbin 、\objexe 、\size の五つのプログラムが
\verb;/usr/local/bin;にインストールされました。

\begin{itemize}
\item {\as}は\tac 用のアセンブラです。
アセンブラは{\cmmc}が出力したアセンブリ言語プログラムを、
リロケータブルオブオブジェクトに変換します。
リロケータブルオブジェクトは、
機械語プログラム、名前表、再配置情報表からなるファイルです。
詳しくは、「\util 解説書」で説明します。

\item {\ld}は\tac 用のリンカです。
リンカは複数のリロケータブルオブジェクトを入力して、
一つのリロケータブルオブジェクトに結合します。

\item {\objbin}は\tac 用のローダです。
ローダはリロケータブルオブジェクトを入力して、
ロードアドレスが決定された機械語を出力します。
出力ファイルの形式については「\util 解説書」で説明します。
ローダは、\tacos のカーネルを作成する時に使用されます。

\item {\objexe}は\tacos のアプリケーションプログラムの
実行形式ファイルを作ります。
これはリロケータブルオブジェクトを入力して、
シンボルテーブルなどリンカしか使用しない情報を取り除いたファイルを出力します。
出力ファイルの形式については「\util 解説書」で説明します。

\item {\size}は、
リロケータブルオブジェクトのテキストセグメント、
初期化データセグメント、
非初期化データセグメントの大きさを表示します。
完成したプログラムのメモリ使用量を見積もるために使用します。
\end{itemize}

\section{コンパイラのインストール}

\cmm コンパイラのソースコードは
\url{https://github.com/tctsigemura/C--/}から入手します。
ダウンロードした配布物を解凍し以下の順にインストールします。

\subsection{コンパイラ本体のインストール}

コンパイラ本体をインストールします。
\verb;C--/src;ディレクトリに移動し以下のように操作します。

\begin{mylist}
\begin{verbatim}
$ make
cc -std=c99 -Wall  -DDATE="\"`date`\"" -DVER=\"3.0.0\" -DARC=\"TaC\" \
...
$ sudo make install
Password:
install -d -m 755 /usr/local/bin
install -m 755 c-- /usr/local/bin
install -m 755 vm-c-- /usr/local/bin
install -m 755 c-c-- /usr/local/bin
$
\end{verbatim}
\end{mylist}

以上で、\cmmc 、{\tt vm-c--}、{\tt c-c--}の三つのプログラムが
作成され\verb;/usr/local/bin;にインストールされました。

\begin{itemize}
\item {\cmmc}は、\tac 用の本物のコンパイラです。
\tac のアセンブリ言語を出力します。
\item {\tt vm-c--}は、コンパイラの勉強をしたい人のために、
中間コードをそのまま出力して見せるコンパイラです。
\item {\tt c-c--}は、\cmml を普通のパソコンやマイコンで試してみたい人のために、
\cmml を \cl に変換するトランスレータです。
変換された \cl プログラムは普通のパソコンで実行できるはずです。
\end{itemize}

\subsection{ヘッダファイルのインストール}

\cmml プログラム用のヘッダファイルをインストールします。
\verb;C--/include;ディレクトリに移動し以下のように操作します。
\verb;/usr/local/cmmInclude;にインストールされます。


\begin{mylist}
\begin{verbatim}
$ sudo make install
Password:
install -d -m 755 /usr/local/cmmInclude
rm -f /usr/local/cmmInclude/*.hmm
install -m 644 *.hmm /usr/local/cmmInclude
$
\end{verbatim}
\end{mylist}

\subsection{ライブラリのインストール}

\verb;C--/lib;ディレクトリに移動しまずライブラリをコンパイルします。
以下のように操作します。

\begin{mylist}
\begin{verbatim}
$ make
as-- crt0.s
cc -E -std=c99 -nostdinc  -I/usr/local/cmmInclude - < ctype.cmm |
  c--  > ctype.s
as-- ctype.s
...
ld-- libtac.o crt0.o ctype.o stdio.o stdlib.o string.o syslib.o ...
rm stdio.s string.s stdlib.s ctype.s
$
\end{verbatim}
\end{mylist}

以上で、\verb;libtac.o;が作成されました。
これが、\tacos のアプリケーションプログラムにリンクされるライブラリです。
次に、インストールします。操作は次の通りです。

\begin{mylist}
\begin{verbatim}
$ sudo make install
Password:
install -d -m 755 /usr/local/cmmLib
install -m 644 libtac.o /usr/local/cmmLib
install -m 644 cfunc.hmm /usr/local/cmmLib
install -m 644 wrapper.c /usr/local/cmmLib
$
\end{verbatim}
\end{mylist}

\verb;/usr/local/cmmLib;ディレクトリに、
いくつかのファイルがインストールされました。
\verb;libtac.o;は\cmml で記述した\tacos 用のライブラリ関数です。
\verb;cfunc.hmm;は\cmm プログラムがパソコンで実行される際に
\cmml ライブラリ関数の代用として使用される \cl ライブラリ関数を定義します。
\verb;wrapper.c;は\cmm プログラムがパソコンで実行される際に使用される
\cl で記述された \cmm ライブラリ関数です。
\verb;cfunc.hmm;による置き換えができない関数が記述してあります。

\section{サンプルプログラムのコンパイル}

\verb;C--/samples/hello;ディレクトリに移動します。
{\tt hello.cmm}プログラムが準備してあります。
これをコンパイルしてみましょう。
なお、\verb;C--/samples/c--;ディレクトリには、
\cmml で記述した\cmm コンパイラが置いてあります。
後で、こちらのコンパイルにも挑戦してください。

\subsection{コンパイル}

このディレクトリで make コマンドを実行します。
下に示す実行例のように{\tt hello.cmm}が自動的にコンパイルされ、
{\tt hello.exe}、{\tt hello}、{\tt hello.c}、{\tt hello.s}、{\tt hello.v}の
五つのファイルができます。

実行例2行は{\tt hello.cmm}から{\tt hello.c}を作る手順を示しています。
実行例3行は{\tt hello.c}から{\tt hello}を作る手順を示しています。
実行例4行は{\tt hello.cmm}から{\tt hello.v}を作る手順を示しています。
実行例5行は{\tt hello.cmm}から{\tt hello.s}を作る手順を示しています。
実行例6〜8行は{\tt hello.s}から{\tt hello.exe}を作る手順を示しています。
実行例を参考に、自分の目的に必要な操作法方法を確認してください。

\begin{mylist}
\begin{verbatim}
 1: $ make
 2: cc -E -DC -std=c99 -nostdinc -I/usr/local/cmmInclude
    -I/usr/local/cmmLib - <  hello.cmm | c-c-- > hello.c
 3: cc -o hello -Wno-parentheses-equality hello.c
    /usr/local/cmmLib/wrapper.c
 4: cc -E  -std=c99 -nostdinc -I/usr/local/cmmInclude
    -I/usr/local/cmmLib - <  hello.cmm | vm-c-- > hello.v
 5: cc -E  -std=c99 -nostdinc -I/usr/local/cmmInclude
    -I/usr/local/cmmLib - <  hello.cmm | c-- > hello.s
 6: as-- hello.s
 7: ld-- mod.o /usr/local/cmmLib/libtac.o hello.o > hello.sym
 8: objexe-- hello.exe mod.o 600 | sort --key=1 > hello.map
 9: $
\end{verbatim}
\end{mylist}

\subsection{関連ファイル}

上の実行例で、使用したり作成されたりしたファイルの主なものを紹介します。

\begin{itemize}

\item {\tt hello.cmm}は、
\cmml のもっとも簡単なサンプルプログラムです。
画面に、\verb/hello,world/と表示します。

\begin{mylist}
\begin{verbatim}
//
// hello.cmm : C--のサンプルプログラム
//
#include <stdio.hmm>

public int main() {
  printf("hello,world\n");
  return 0;
}
\end{verbatim}
\end{mylist}

\item {\tt hello.exe}は、
\verb/hello.cmm/をコンパイルした\tacos 用の実行形式ファイルです。
\tac の実機で実行できます。

\item {\tt hello}は、
\verb/hello.cmm/をコンパイルしたパソコン用の実行形式のファイルです。
次のようにパソコンで実行できます。

\begin{mylist}
\begin{verbatim}
$ ./hello 
hello,world
$
\end{verbatim}
\end{mylist}

\item {\tt hello.c}は、
\verb/hello.cmm/を{\tt c-c--} で\cl に変換したプログラムです。
上記の{\tt hello}は{\tt hello.c}を\cl コンパイラでコンパイルしたものです。

\begin{mylist}
\begin{verbatim}
#include <stdio.h>
#define _cmm_str_L0 "hello,world\n"
int main(){
printf(_cmm_str_L0);
}
\end{verbatim}
\end{mylist}

\item {\tt hello.s}は、
\verb/hello.cmm/を\cmmc で\tac のアセンブリ言語に変換したプログラムです。
上記の{\tt hello.exe}ファイルは、
このファイルから\cmm コンパイラ用ユーティリティ(\util)を用いて作成されました。

\begin{mylist}
\begin{verbatim}
_stdin  WS      1
_stdout WS      1
_stderr WS      1
.L1     STRING  "hello,world\n"
_main   PUSH    FP
        LD      FP,SP
        CALL    __stkChk
        LD      G0,#.L1
        PUSH    G0
        CALL    _printf
        ADD     SP,#2
        LD      G0,#0
        POP     FP
        RET
\end{verbatim}
\end{mylist}

\item {\tt hello.v}は、
\verb/hello.cmm/を{\tt vm-c--}により\cmm
コンパイラの中間言語に変換したプログラムです。
中間言語の詳細は、????で紹介します。

\begin{mylist}
\begin{verbatim}
_stdin
        WS      1
_stdout
        WS      1
_stderr
        WS      1
.L1
        STRING  "hello,world\n"
_main
        ENTRY   0
        LDC     .L1
        ARG
        CALLF   1,_printf
        POP
        LDC     0
        MREG
        RET
\end{verbatim}
\end{mylist}

\item {\tt Makefile}は、
コンパイル手順を記述したファイルです。
これまでの実行例で、{\tt make}コマンドで自動的にコンパイルができたのは、
このファイルのおかげです。
\cmm プログラムは、
コンパイル前にプリプロセッサで処理する必要があります。
コンパイルする度に
プリプロセッサやコンパイラを順に手動で起動するのは面倒なので、
{\tt Makefile}を準備した方が良いと思います。

\verb;C--/samples/hello;ディレクトリには、
先程の実行例で使用された{\tt Makefile}の他に、
実際に使用する場合を想定したMacOS、UNIX 用の{\tt Makefile.unix}と、
\tacos 用の{\tt Makefile.}が準備してあります。
以下に後の二つのファイルの内容を掲載します。
これらは、OSX 10.10 で動作するように調整してあります。
これを参考に、自分の目的と環境に合った{\tt Makefile}を作ってください。

\begin{mylist}
\begin{verbatim}
#
# Makefile.unix : C--言語からMacOSやUNIXの実行形式に変換する手順
#
INCDIR=/usr/local/cmmInclude
LIBDIR=/usr/local/cmmLib
CFLAGS+=-Wno-parentheses-equality
CPP=cc -E -DC -std=c99 -nostdinc -I${INCDIR} -I${LIBDIR} - < 
.SUFFIXES: .o .cmm .c
.cmm.c:
        ${CPP} $*.cmm | c-c-- > $*.c
OBJS=hello.o
hello: ${OBJS}
        cc -o hello ${CFLAGS} ${OBJS} ${LIBDIR}/wrapper.c
clean:
        rm -f hello ${OBJS}
\end{verbatim}
\end{mylist}

\begin{mylist}
\begin{verbatim}
#
# Makefile.tac : C-- 言語からTacOSの実行形式に変換する手順
#
INCDIR=/usr/local/cmmInclude
LIBDIR=/usr/local/cmmLib
CFLAGS+=-Wno-parentheses-equality
CPP=cc -E -std=c99 -nostdinc -I${INCDIR} - < 
.SUFFIXES: .o .cmm .s
.cmm.s:
        ${CPP} $*.cmm | c-- > $*.s
.s.o:
        as-- $*.s
OBJS=hello.o
hello.exe: ${OBJS}
        ld-- mod.o ${LIBDIR}/libtac.o ${OBJS} > hello.sym
        objexe-- hello.exe mod.o 600 | sort --key=1 > hello.map
clean:
        rm -f hello.exe mod.o ${OBJS} *.sym *.lst *.map
\end{verbatim}
\end{mylist}

\end{itemize}
