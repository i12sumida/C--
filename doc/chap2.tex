% 
% 2章 C--言語開発環境のインストール
%
\chapter{\cmml 開発環境のインストール}

\cmml を体験するために、自分のパソコンで\cmm コンパイラを
使用できるようにしましょう。
{\bf 以下の順番で}作業し、開発環境をインストールしてください。


\section{ユーティリティのインストール}

{\tac}用のプログラムを作成するために、まず、
{\cmm}コンパイラ用ユーティリティをインストールする必要があります。
ソースコードは
\url{https://github.com/tctsigemura/Util--/}から入手します。

ダウンロードした配布物を展開し
「{\util}解説書」({\tt Util--/doc/umm.pdf})の手順に従いインストールします。
{\as}、{\ld}、{\objbin}、{\objexe}、{\size}の五つのプログラムが
{\tt /usr/local/bin}にインストールされます。

\section{コンパイラのインストール}
\label{chap2:compilerl}

{\cmm}コンパイラのソースコードは
\url{https://github.com/tctsigemura/C--/}から入手します。
ダウンロードした配布物を展開し以下の順にインストールします。

\subsection{コンパイラ本体のインストール}

%コンパイラ本体をインストールします。
{\tt C--/src}ディレクトリに移動し以下のように操作します。

\begin{mylist}
\begin{verbatim}
$ make
cc -std=c99 -Wall  -DDATE="\"`date`\"" -DVER=\"3.1.7\" ...
...
$ sudo make install
Password:
install -d -m 755 /usr/local/bin
install -m 755 c-- /usr/local/bin
install -m 755 vm-c-- /usr/local/bin
...
\end{verbatim}
\end{mylist}

以上で、{\cme}、{\cmc}、{\cmi}、{\cmv}、{\cmmc}、{\ccmmc}、{\icmmc}、{\vcmmc}の
八つのプログラムが作成され\verb;/usr/local/bin;にインストールされました。
{\cme}、{\cmc}、{\cmi}、{\cmv}は、
{\cmml}をコンパイルするために使用するシェルスクリプトです。
これらのシェルスクリプトは、
与えられたファイルの拡張子から処理するべき手順を判断し、
コンパイラやユーティリティを自動的に実行します。

\begin{itemize}
\item {\cme}は、
{\cmm}プログラムを{\tac}で実行できる「{\tt .exe}」ファイルに変換します。
\item {\cmc}は、
{\cmm}プログラムをUNIXやMacで実行できるファイルに変換します。
\item {\cmi}は、
{\cmm}プログラムを中間コードに変換します。
\item {\cmv}は、
{\cmm}プログラムをスタックマシンのニーモニックに変換します。
\item {\cmmc}は、{\tac}用の{\cmml}コンパイラ本体です。
{\cmm}プログラムを{\tac}のアセンブリ言語に変換します。
通常{\cme}から呼び出され、ユーザが直接使用することはありません。
\item {\ccmmc}は、{\cmm}プログラムを{\cl}に変換するトランスレータです。
変換された{\cl}プログラムはUNIXやMacで実行できます。
通常{\cmc}から呼び出され、ユーザが直接使用することはありません。
\item {\icmmc}は、コンパイラの勉強をしたい人のために、
中間言語(\pageref{app:vm}ページ)を出力して見せるコンパイラです。
通常{\cmi}から呼び出され、ユーザが直接使用することはありません。
\item {\vcmmc}は、コンパイラの勉強をしたい人のために、
中間言語(\pageref{app:vm}ページ)をよく反映した、
仮想スタックマシンのニーモニックを出力して見せるコンパイラです。
通常{\cmv}から呼び出され、ユーザが直接使用することはありません。
\end{itemize}

なお、これら八つのプログラムの使用方法は、
コマンドリファレンス(\pageref{app:command}ページ)で紹介します。

\subsection{ヘッダファイルのインストール}

{\cmml}プログラム用のヘッダファイルをインストールします。
\verb;C--/include;ディレクトリに移動し以下のように操作します。
\verb;/usr/local/cmmInclude;にインストールされます。

\begin{mylist}
\begin{verbatim}
$ sudo make install
Password:
install -d -m 755 /usr/local/cmmInclude
rm -f /usr/local/cmmInclude/*.hmm
install -m 644 *.hmm /usr/local/cmmInclude
\end{verbatim}
\end{mylist}

\subsection{ライブラリのインストール}

\verb;C--/lib;ディレクトリに移動しまずライブラリをコンパイルします。
以下のように操作します。

\begin{mylist}
\begin{verbatim}
$ make
as-- crt0.s
cm2e -c  ctype.cmm
cm2e -c  stdio.cmm
cm2e -c  printf.cmm
cm2e -c  stdlib.cmm
cm2e -c  string.cmm
as-- syslib.s
ld-- libtac.o crt0.o ctype.o stdio.o printf.o stdlib.o ...
\end{verbatim}
\end{mylist}

以上で、\verb;libtac.o;が作成されました。
これが、{\tacos}のアプリケーションプログラムにリンクされるライブラリです。
次に、インストールします。操作は次の通りです。

\begin{mylist}
\begin{verbatim}
$ sudo make install
Password:
install -d -m 755 /usr/local/cmmLib
install -m 644 libtac.o /usr/local/cmmLib
install -m 644 cfunc.hmm /usr/local/cmmLib
install -m 644 wrapper.c /usr/local/cmmLib
install -m 644 wrapper.h /usr/local/cmmLib
ln -fs wrapper.h /usr/local/cmmLib/crt0.h
\end{verbatim}
\end{mylist}

\verb;/usr/local/cmmLib;ディレクトリに、
いくつかのファイルがインストールされました。
\verb;libtac.o;は{\cmml}で記述した{\tacos}用のライブラリ関数です。
\verb;cfunc.hmm;は{\cmm}プログラムがパソコンで実行される際に
{\cmml}ライブラリ関数の代用として使用される{\cl}ライブラリ関数を
定義(\verb;#define;)します。
\verb;wrapper.c;は{\cmm}プログラムがパソコンで実行される際に使用される
{\cl}で記述された{\cmm}ライブラリ関数です。
\verb;cfunc.hmm;による置き換えができない関数が記述してあります。

\section{サンプルプログラムのコンパイル}

\verb;C--/samples/hello;ディレクトリに移動します。
{\tt hello.cmm}プログラムが準備してあります。
これをコンパイルして実行してみましょう。
{\tt hello.cmm}プログラムは以下のような{\cmml}プログラムです。

\begin{mylist}
\begin{verbatim}
//
// hello.cmm : C--のサンプルプログラム
//
#include <stdio.hmm>

public int main() {
  printf("hello,world\n");
  return 0;
}
\end{verbatim}
\end{mylist}

\subsection{コンパイルしてPCで実行}

\verb;C--/samples/hello;ディレクトリで以下のように操作すると、
{\tt hello.cmm}プログラムをコンパイルしてPCで実行することができます。
なお、{\cmc}は、「\ref{chap2:compilerl} コンパイラのインストール」で
インストールしたシェルスクリプトです。
コンパイラとユーティリティを自動的に起動して、
{\cmml}プログラムをPCで実行可能なプログラムに変換します。

\begin{mylist}
\begin{verbatim}
$ cm2c -o hello hello.cmm 
$ ./hello
hello,world
\end{verbatim}
\end{mylist}

\subsection{コンパイルしてTaCで実行}

\verb;C--/samples/hello;ディレクトリで以下のように操作すると、
{\tt hello.cmm}プログラムをコンパイルして{\tac}で実行可能な
{\tt hello.exe}を作ることができます。
なお、{\cme}は、「\ref{chap2:compilerl} コンパイラのインストール」で
インストールしたシェルスクリプトです。
コンパイラとユーティリティを自動的に起動して、
{\cmml}プログラムを{\tac}で実行可能なプログラムに変換します。
{\tt hello.exe}をメモリカードにコピーすると{\tac}で実行できます。

\begin{mylist}
\begin{verbatim}
$ cm2e -o hello hello.cmm 
$ ls -l
...
-rw-r--r--  1 sigemura  staff   138 May  5 01:03 hello.cmm
-rw-r--r--  1 sigemura  staff  8164 May 28 23:28 hello.exe
-rw-r--r--  1 sigemura  staff  5862 May 28 23:28 hello.map
\end{verbatim}
\end{mylist}

\subsection{いろいろな中間ファイル}

上の例で示した他に、{\tt hello.cmm}から数種類のファイルを作ることができます。

\begin{itemize}

\item {\tt hello.c}は、
{\cmc}を{\tt -S}オプション付きで実行して
{\tt hello.cmm}を変換した{\cl}ファイルです。
「\verb/$ cm2c -S hello.cmm/」のように実行します。
上記の{\tt hello}は、これを{\cl}コンパイラでコンパイルしたものです。

\begin{mylist}
\begin{verbatim}
#include <stdio.h>
#define _cmm_0S "hello,world\n"
int main(){
printf(_cmm_0S);
return 0;
}
\end{verbatim}
\end{mylist}

\item {\tt hello.s}は、
{\cme}を{\tt -S}オプション付きで実行して
{\tt hello.cmm}を変換した{\tac}のアセンブリ言語プログラムです。
「\verb/$ cm2e -S hello.cmm/」のように実行します。
上記の{\tt hello.exe}は、
このファイルから{\cmm}コンパイラ用のユーティリティ({\tt Util--})を
用いて作成されました。

\begin{mylist}
\begin{verbatim}
_stdin  WS      1
_stdout WS      1
_stderr WS      1
.L1     STRING  "hello,world\n"
_main   PUSH    FP
        LD      FP,SP
        CALL    __stkChk
        LD      G0,#.L1
        PUSH    G0
        CALL    _printf
        ADD     SP,#2
        LD      G0,#0
        POP     FP
        RET
\end{verbatim}
\end{mylist}

\item {\tt hello.i}は、{\tt hello.cmm}を{\cmi}により
中間言語に変換したプログラムです。\\
「\verb/$ cm2i hello.cmm/」のように実行します。
中間言語の詳細は、
\pageref{app:vm}ページで紹介します。

\begin{mylist}
\begin{verbatim}
vmNam(25)
vmWs(1)
vmNam(26)
vmWs(1)
vmNam(27)
vmWs(1)
vmLab(1)
vmStr("hello,world\n")
vmNam(65)
vmEntry(0)
vmLdLab(1)
vmArg()
vmCallF(1, 62)
vmPop()
vmLdCns(0)
vmMReg()
vmRet()
\end{verbatim}
\end{mylist}

\item {\tt hello.v}は、{\tt hello.cmm}を{\cmv}により
仮想スタックマシンのニーモニックに変換したプログラムです。
中間言語とニーモニックは一対一に対応しています。\\
「\verb/$ cm2v hello.cmm/」のように実行します。
仮想スタックマシンのニーモニックの詳細は、
\pageref{app:vm}ページで紹介します。

\begin{mylist}
\begin{verbatim}
_stdin
        WS      1
_stdout
        WS      1
_stderr
        WS      1
.L1
        STRING  "hello,world\n"
_main
        ENTRY   0
        LDC     .L1
        ARG
        CALLF   1,_printf
        POP
        LDC     0
        MREG
        RET
\end{verbatim}
\end{mylist}
\end{itemize}

\subsection{Makefileファイル}

{\tt Makefile}は、
コンパイル手順を記述したファイルです。
数種類の例を示しますので参考にしてください。

\begin{mylist}
\begin{verbatim}
#
# Makefile.tac : C--言語からTacの実行形式に変換する手順
#

hello.exe: hello.cmm
        cm2e -o hello hello.cmm

clean:
        rm -f hello.exe hello.map
\end{verbatim}
\end{mylist}

\begin{mylist}
\begin{verbatim}
#
# Makefile.unix : C--言語からMacOSやUNIXの実行形式に変換する手順
#

hello: hello.cmm
        cm2c -o hello hello.cmm

clean:
        rm -f hello
\end{verbatim}
\end{mylist}

\begin{mylist}
\begin{verbatim}
#
# C-- 言語から a.out、.exe、 .v ファイルを作る手順
#

all: hello hello.exe hello.v

# UNIX, Mac の a.out へ変換
hello: hello.cmm
        cm2c -o hello hello.cmm

# TacOS の実行形式を作る
hello.exe: hello.cmm
        cm2e -o hello.exe hello.cmm

# C--コンパイラの中間言語に変換する
hello.v: hello.cmm
        cm2v hello.cmm

clean:
        rm -f hello hello.c hello.s hello.v *.lst *.sym *.map \
	*.o *.exe *~

\end{verbatim}
\end{mylist}
